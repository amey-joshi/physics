\chapter{Some mathematical results}
\section{Green's theorem}\label{c2sa1}
We will prove Green's theorem
\begin{equation}\label{c2sae1}
\int\left(\phi\nabla^2\psi - \psi\nabla^2\phi\right)dV = \int\left(\phi\grad{\psi} - \psi\grad{\phi}\right)\cdot\un dA
\end{equation}
Let $\vec{A} = \phi\grad{\psi}$ and $\vec{B} = \psi\grad{\psi}$. Then,
\[
\int\left(\vec{A} - \vec{B}\right)\cdot\un dA = \int\dive{(\vec{A} - \vec{B})}dV
\]
Now, $\dive{\vec{A}} = \dive{(\phi\grad{\psi})} = \grad{\phi}\cdot\grad{\psi} + \phi\nabla^2\psi$. Similarly, $\dive{\vec{B}} = \grad{\psi}\cdot\grad{\phi} + \psi\nabla^2\phi$. Therefore,
\[
\int(\vec{A} - \vec{B})\cdot\un dA = \int\left(\phi\nabla^2\psi - \psi\nabla^2\phi\right)dV
\]

\section{Solution of Poisson equation}\label{c2sa2} 
We will show that the solution of the Poisson's equation $\nabla^2\phi = \Delta$ is
\[
\phi(\vec{x}) = -\frac{1}{4\pi}\int\frac{\Delta(\vec{x}^\op)}{|\vec{x} - \vec{x}^\op|} dV(\vec{x}^\op)
\]
We begin with Fourier transform of $\nabla^2\phi = \Delta(\vec{x})$. Thus,
\[
\frac{1}{(2\pi)^{3/2}}\int_{-\infty}^{\infty} e^{-i\vec{k}\cdot\vec{x}}\nabla^2\phi dV = \frac{1}{(2\pi)^{3/2}}\int_{-\infty}^{\infty} e^{-i\vec{k}\cdot\vec{x}}\Delta(\vec{x}) dV
\]
Use Green's theorem \eqref{c2sae1} on the left hand side to get,
\begin{eqnarray*}
\frac{1}{(2\pi)^{3/2}}\int_{-\infty}^{\infty} \phi\nabla^2e^{-i\vec{k}\cdot\vec{x}} dV &=& \frac{1}{(2\pi)^{3/2}}\int_{-\infty}^{\infty} e^{-i\vec{k}\cdot\vec{x}}\Delta(\vec{x}) dV \\
+ \frac{1}{(2\pi)^{3/2}}\int_{-\infty}^{\infty} e^{-i\vec{k}\cdot\vec{x}} \grad{\phi}\cdot\un dA & & \\
- \frac{1}{(2\pi)^{3/2}}\int_{-\infty}^{\infty} \phi \grad{e^{-i\vec{k}\cdot\vec{x}}}\cdot\un dA
\end{eqnarray*}
The second and the third term on the left hand side can be written as
\[
\lim_{R \rightarrow \infty} \frac{R^2}{(2\pi)^{3/2}}\int_0^\pi\int_0^{2\pi}\left(e^{-i\vec{k}\cdot\vec{x}} \grad{\phi} - \phi \grad{e^{-i\vec{k}\cdot\vec{x}}}\right)\sin\theta 
d\theta d\varphi
\]
For a function $\phi$ that goes to zero faster than $1/r^2$ as $r \rightarrow \infty$, the above expression will be zero. Therefore,
\begin{equation}\label{c2sae2}
\frac{1}{(2\pi)^{3/2}}\int_{-\infty}^{\infty} \phi\nabla^2e^{-i\vec{k}\cdot\vec{x}} dV = \frac{1}{(2\pi)^{3/2}}\int_{-\infty}^{\infty} e^{-i\vec{k}\cdot\vec{x}}\Delta(\vec{x}) dV 
\end{equation}
To calculate $\nabla^2e^{-i\vec{k}\cdot\vec{x}}$, we assume, without loss of generality, that $\vec{k}$ is along the positive $z$ axis so that $\vec{k}\cdot\vec{x} = kr\cos\theta$.
Therefore,
\begin{eqnarray*}
\nabla^2e^{-i\vec{k}\cdot\vec{x}} &=& \frac{1}{r^2}\frac{\partial}{\partial r}\left(r^2\frac{\partial}{\partial r} e^{-ikr\cos\theta}\right) + \\
 & & \frac{1}{r^2\sin\theta}\frac{\partial}{\partial\theta}\left(\sin\theta\frac{\partial}{\partial\theta} e^{-ikr\cos\theta}\right) + \\
 & & \frac{1}{r^2\sin^2\theta}\frac{\partial^2}{\partial\phi^2}e^{-ikr\cos\theta}
\end{eqnarray*}
or
\[
\nabla^2e^{-i\vec{k}\cdot\vec{x}} = -k^2e^{-ikr\cos\theta} = -k^2e^{-i\vec{k}\cdot\vec{x}}
\]
Therefore, equation \eqref{c2sae2} becomes,
\[
-\frac{k^2}{(2\pi)^{3/2}}\int_{-\infty}^{\infty} \phi e^{-i\vec{k}\cdot\vec{x}} dV = \frac{1}{(2\pi)^{3/2}}\int_{-\infty}^{\infty} e^{-i\vec{k}\cdot\vec{x}}\Delta(\vec{x}) dV 
\]
If $\hat{f}(\vec{k})$ denotes the Fourier transform of $f(\vec{x})$, the above equation means,
\[
-k^2\hat{\phi}(\vec{k}) = \hat{\Delta}(\vec{k}) \Rightarrow \hat{\phi}(\vec{k}) = -\frac{\hat{\Delta}(\vec{k})}{k^2}
\]
Therefore,
\[
\phi(\vec{x}) = \phi_h(\vec{x}) - \frac{1}{(2\pi)^{3/2}}\int_{-\infty}^{\infty} \frac{\hat{\Delta}(\vec{k})}{k^2} e^{i\vec{k}\cdot\vec{x}} dV_k
\]
where $dV_k$ denotes the volume element in the $\vec{k}$-space and $\phi_h(\vec{x})$ is any solution of the homogeneous equation $\nabla^2\phi = 0$. Using the expression for 
$\hat{\Delta}(\vec{k})$, we get
\[
\phi(\vec{x})=\phi_h(\vec{x})-\frac{1}{(2\pi)^{3}}\int_{-\infty}^{\infty}\int_{-\infty}^{\infty} \Delta(\vec{x}^\op)\frac{e^{i\vec{k}\cdot(\vec{x}-\vec{x}^\op)}}{k^2}dV^\op dV_k,
\]
where $dV^\op$ is integration over $\vec{x}^\op$. Let,
\[
G(\vec{x}, \vec{x}^\op) = -\frac{1}{(2\pi)^{3}}\int_{-\infty}^{\infty}\frac{e^{i\vec{k}\cdot(\vec{x}-\vec{x}^\op)}}{k^2}dV_k
\]
be the Green function so that,
\begin{equation}\label{c2sae3}
\phi(\vec{x})=\phi_h(\vec{x}) + \int_{-\infty}^{\infty}G(\vec{x}, \vec{x}^\op)\Delta(\vec{x}^\op) dV^\op
\end{equation}
The Green function is an integral in $\vec{k}$-space. Therefore, without loss of generality, we can orient the $k_z$ axis parallel 
to the vector $\vec{x}-\vec{x}^\op$ so that
\[
G(\vec{x}, \vec{x}^\op) = -\frac{1}{(2\pi)^{3}}\int_{0}^{\infty}\int_0^\pi\int_0^{2\pi}\frac{e^{ik|\vec{x}-\vec{x}^\op|\cos\theta}}{k^2}k^2\sin\theta dk d\theta d\varphi
\]
or
\[
G(\vec{x}, \vec{x}^\op) = -\frac{1}{(2\pi)^{2}}\int_{0}^{\infty}\int_0^\pi e^{ik|\vec{x}-\vec{x}^\op|\cos\theta}\sin\theta dk d\theta
\]
Put $t = \cos\theta$ so that,
\[
G(\vec{x}, \vec{x}^\op) = \frac{1}{(2\pi)^{2}}\int_{0}^{\infty}\int_1^{-1} e^{ik|\vec{x}-\vec{x}^\op| t} dt dk
\]
or
\[
G(\vec{x}, \vec{x}^\op) = \frac{1}{(2\pi)^{2}}\int_{0}^{\infty}\left( \frac{e^{ik|\vec{x}-\vec{x}^\op| t}}{ik|\vec{x}-\vec{x}^\op|}\Big|_{1}^{-1}\right)dk
\]
or
\[
G(\vec{x}, \vec{x}^\op) = \frac{i}{(2\pi)^{2}}\int_{0}^{\infty} \frac{e^{ik|\vec{x}-\vec{x}^\op|} - e^{-ik|\vec{x}-\vec{x}^\op|}}{k|\vec{x}-\vec{x}^\op|} dk
\]
or,
\[
G(\vec{x}, \vec{x}^\op) = \frac{i}{(2\pi)^{2}}\frac{1}{|\vec{x}-\vec{x}^\op|}\int_{0}^{\infty} \frac{e^{ik|\vec{x}-\vec{x}^\op|} - e^{-ik|\vec{x}-\vec{x}^\op|}}{k} dk
\]
Now,
\begin{eqnarray*}
\int_{0}^{\infty} \frac{e^{ik|\vec{x}-\vec{x}^\op|} - e^{-ik|\vec{x}-\vec{x}^\op|}}{k} dk &=& \\
\int_{0}^{\infty} \frac{e^{ik|\vec{x}-\vec{x}^\op|}}{k} dk - \int_{0}^{\infty} \frac{e^{-ik|\vec{x}-\vec{x}^\op|}}{k} dk &=& \\
\int_{0}^{\infty} \frac{e^{ik|\vec{x}-\vec{x}^\op|}}{k} dk - \int_{0}^{-\infty} \frac{e^{ik|\vec{x}-\vec{x}^\op|}}{(-k)} d(-k) &=& \\
\int_{0}^{\infty} \frac{e^{ik|\vec{x}-\vec{x}^\op|}}{k} dk + \int_{-\infty}^{0} \frac{e^{ik|\vec{x}-\vec{x}^\op|}}{k} dk &=& \\
\int_{-\infty}^{\infty} \frac{e^{ik|\vec{x}-\vec{x}^\op|}}{k} dk
\end{eqnarray*}
so that,
\begin{equation}\label{c2sae4}
G(\vec{x}, \vec{x}^\op) = \frac{i}{(2\pi)^{2}}\frac{1}{|\vec{x}-\vec{x}^\op|}\int_{-\infty}^{\infty} \frac{e^{ik|\vec{x}-\vec{x}^\op|}}{k} dk
\end{equation}
We can evaluate the integral using residue theorem. Consider a semi-circular contour with diameter along the real axis, from $-R$ to $R$ and arc in the first two quadrants. Now the
integrand has a singularity the origin, a point on the contour. Therefore, we have to take the Cauchy principal value of the integrand. To that end, indent the contour by a small 
semi-circulararc with diameter along the real axis, from $-a$ to $a$ and arc in the first two quadrants. The indented contour now no longer has the singularity in its interior and
hence by Cauchy theorem, the integral around it is zero. If $C_1$ denotes the larger arc and $C_2$ the smaller one, then
\begin{equation}\label{c2sae5}
\left(\int_{C_1} + \int_{-R}^{-a} + \int_{C_2} + \int_{a}^R\right)\frac{e^{ik|\vec{x}-\vec{x}^\op|}}{k} dk = 0
\end{equation}
In the limit $R \rightarrow \infty$, the integral along $C_1$ vanishes because the integrand itself goes to zero. Along $C_2$, $k = \rho e^{i\theta}$, $dk = i\rho e^{i\theta}d\theta$ and
the limits of the integral are from $\pi$ to $0$. Thus,
\begin{eqnarray*}
\int_{C_2}\frac{e^{ik|\vec{x}-\vec{x}^\op|}}{k} dk &=& \int_{\pi}^0 \exp\left(i\rho|\vec{x}-\vec{x}^\op|(\cos\theta+i\sin\theta)\right)\frac{i\rho e^{i\theta}d\theta}{\rho e^{i\theta}} \\
 &=& i\int_{\pi}^0 \exp\left(i\rho|\vec{x}-\vec{x}^\op|(\cos\theta+i\sin\theta)\right) d\theta
\end{eqnarray*}
In the limit $\rho \rightarrow 0$, we can approximate the integrand by its Maclaurin series so that,
\begin{eqnarray*}
\int_{C_2}\frac{e^{ik|\vec{x}-\vec{x}^\op|}}{k} dk &=& i\int_{\pi}^0 \left(1 + i\rho|\vec{x}-\vec{x}^\op|(\cos\theta+i\sin\theta)\right) d\theta \\
 &=& -i\pi + 2\rho|\vec{x}-\vec{x}^\op|,
\end{eqnarray*}
which, in the limit of vanishing $\rho$ gives $-i\pi$. Therefore, equation \eqref{c2sae5} becomes,
\[
\int_{-\infty}^\infty \frac{e^{ik|\vec{x}-\vec{x}^\op|}}{k} dk = -\int_{C_2}\frac{e^{ik|\vec{x}-\vec{x}^\op|}}{k} dk = i\pi
\]
Therefore, equation \eqref{c2sae4} becomes,
\[
G(\vec{x}, \vec{x}^\op) = -\frac{1}{4\pi}\frac{1}{|\vec{x}-\vec{x}^\op|}
\]
Therefore from \eqref{c2sae3}
\[
\phi(\vec{x})=\phi_h(\vec{x}) - \frac{1}{4\pi}\int_{-\infty}^{\infty}\frac{\Delta(\vec{x}^\op)}{|\vec{x}-\vec{x}^\op|} dV^\op
\]
Since $\phi_h(r) = 0$ is a solution of the Laplace's equation,
\[
\phi(\vec{x}) = - \frac{1}{4\pi}\int_{-\infty}^{\infty}\frac{\Delta(\vec{x}^\op)}{|\vec{x}-\vec{x}^\op|} dV^\op
\]
is a solution of the Poisson's equation $\nabla^2\phi = \Delta(\vec{x})$.

\section{Another form of Stokes's theorem}\label{c2sa3} 
We will prove an analog of Stokes' theorem for a scalar field,
\begin{equation}\label{c2sae6}
\oint f d\vec{l} = -\int(\grad{f} \vp \un)dA
\end{equation}
Let $\vec{F}(\vec{x}) = \vec{\alpha} f(\vec{x})$, where $\vec{\alpha}$ is a constant vector. Therefore, Stokes' theorem for $\vec{F}$ is
\[
\oint\vec{F}\cdot d\vec{l} = \int\curl{\vec{F}}\cdot\un dA
\]
that is
\[
\vec{\alpha}\cdot\oint f(\vec{x}) d\vec{l} = \int \left(\grad{f(\vec{x})} \vp \vec{\alpha}\right)\cdot\un dA,
\]
or,
\[
\vec{\alpha}\cdot\oint f(\vec{x}) d\vec{l} = -\int \left(\vec{\alpha} \vp \grad{f(\vec{x})}\right)\cdot\un dA,
\]
Interchanging the dot and the cross products on the right hand side,
\[
\vec{\alpha}\cdot\oint f(\vec{x}) d\vec{l} = -\vec{\alpha}\cdot\int\left(\grad{f(\vec{x})} \vp \un\right)dA
\]
or
\[
\vec{\alpha}\cdot\left\{\oint f(\vec{x}) d\vec{l} + \int\left(\grad{f(\vec{x})} \vp \un\right)dA\right\} = 0
\]
Since $\vec{\alpha}$ is an arbitrary vector,
\[
\oint f(\vec{x}) d\vec{l} + \int\left(\grad{f(\vec{x})} \vp \un\right)dA = 0
\]

\section{Multipole expansion}\label{c2sa4} 
We will demonstrate the multipole expansion for an electrostatic potential. If $\rho(\vec{x})$ is an arbitrary charge distribution then the potential due to it at a 
point $\vec{x}$, in gaussian units, is
\[
\phi(\vec{x}) = \int\frac{\rho(\vec{x}^\op)}{|\vec{x} - \vec{x}^\op|}dV(\vec{x}^\op)
\]
If $\vec{s} = \vec{x} - \vec{x}^\op$ then the Taylor expansion for $s^{-1}$ is
\[
\frac{1}{s} = \frac{1}{r} + x_i^\op\frac{\partial}{\partial x_i}\left(\frac{1}{s}\right) + \frac{x_i^\op x_j^\op}{2!}\frac{\partial^2}{\partial x_i \partial x_j}\left(\frac{1}{s}\right)
+ \frac{x_i^\op x_j^\op x_k^\op}{3!} \frac{\partial^3}{\partial x_i \partial x_j \partial x_k}\left(\frac{1}{s}\right) + \cdots
\]
Therefore,
\[
\phi(\vec{x}) = \frac{Q^{(0)}}{r} + Q_i^{(1)}\frac{\partial}{\partial x_i}\left(\frac{1}{s}\right) + 
\frac{Q_{ij}^{(2)}}{2!}\frac{\partial^2}{\partial x_i \partial x_j}\left(\frac{1}{s}\right) +
\frac{Q_{ijk}^{(3)}}{3!}\frac{\partial^3}{\partial x_i \partial x_j \partial x_k}\left(\frac{1}{s}\right) + \cdots
\]
where
\begin{eqnarray*}
Q^{(0)} &=& \int \rho(\vec{x}^\op)dV(\vec{x}^\op) \\
Q_i^{(1)} &=& \int x_i \rho(\vec{x}^\op)dV(\vec{x}^\op) \\
Q_{ij}^{(2)} &=& \int x_i x_j \rho(\vec{x}^\op)dV(\vec{x}^\op) \\
Q_{ijk}^{(3)} &=& \int x_i x_j x_k \rho(\vec{x}^\op)dV(\vec{x}^\op) \\
\end{eqnarray*}
are the multipoles of charge density. The zeroth order term $Q^{(0)}$ is the total charge, the first order term $Q_i^{(1)}$ is the dipole moment, the second order term $Q_{ij}^{(2)}$ is 
the quadrupole moment and $Q_{ijk}^{(3)}$ is the octupole moment of the charge distribution. If we write $s^2 = \sum_{a}(x_a - x_a^\op)^2$ then we observe that
\begin{eqnarray*}
\frac{\partial}{\partial x_i}\left(\frac{1}{s}\right) &=& -\frac{x_i - x_i^\op}{s^3} \\
\frac{\partial^2}{\partial x_i \partial x_j}\left(\frac{1}{s}\right) &=& 3\frac{(x_i - x_i^\op)(x_j - x_j^\op)}{s^5} -\frac{\delta_{ij}}{s^3} \\
\frac{\partial^3}{\partial x_i \partial x_j \partial x_k}\left(\frac{1}{s}\right) &=& -15\frac{(x_i - x_i^\op)(x_j - x_j^\op)(x_k - x_k^\op)}{s^7} + \\
 & & \frac{3}{s^5}\left[(x_i - x_i^\op)\delta_{jk} + (x_j - x_j^\op)\delta_{ik} + (x_k - x_k^\op)\delta_{ij}\right]
\end{eqnarray*}
Thus,
\begin{eqnarray*}
\phi(\vec{x}) &=& \frac{Q^{(0)}}{r} - \frac{(x_i - x_i^\op)Q_i^{(1)}}{s^3} + \frac{3(x_i - x_i^\op)(x_j - x_j^\op)Q_{ij}^{(2)} - Q_{ii}^{(2)}}{s^5} -\\
 & & \frac{15(x_i - x_i^\op)(x_j - x_j^\op)(x_k - x_k^\op)Q_{ijk}^{(3)} + 3(x_i - x_i^\op)Q_{ijj}^{(3)}}{s^7} + \\ 
 & & O\left(\frac{1}{s^5}\right),
\end{eqnarray*}
where we have used the symmetry of the octupole moment tensor to get
\[
(x_i - x_i^\op)Q_{ijj}^{(3)} + (x_j - x_j^\op)Q_{iji}^{(3)} + (x_k - x_k^\op)Q_{iik}^{(3)} = 3(x_i - x_i^\op)Q_{ijj}^{(3)}
\]
The $n$th term of the expansion of $\phi$ goes as $r^{n}$. We will not get an expression for the field due to an arbitrary charge distribution. Since $\vec{E} = -\grad{\phi}$,
\begin{eqnarray*}
E_a &=& - Q^{(0)}\pdt{(r^{-1})}{x_a} + Q_i^{(1)}\frac{\partial}{\partial x_a}\left(\frac{(x_i - x_i^\op)}{s^3}\right) - \\
 & & Q_{ij}^{(2)}\frac{\partial}{\partial x_a}\left(\frac{3(x_i - x_i^\op)(x_j - x_j^\op) - \delta_{ij}}{s^5}\right) + \\
 & & Q_{ijk}^{(3)}\frac{\partial}{\partial x_a}\left(\frac{15(x_i - x_i^\op)(x_j - x_j^\op)(x_k - x_k^\op) + 3(x_i - x_i^\op)\delta_{jk}}{s^7}\right) - \\
 & & \cdots
\end{eqnarray*}
Thus, the field due to a monopole goes as $r^{-2}$, that due to a dipole as $r^{-3}$, quadrupole as $r^{-4}$ and so on.

\section{Solid harmonics}\label{c2sa5}
We consider the solution of Laplace's equation using the method of separation of variables. Let $\Phi(r, \varphi, \theta) = R(r)S(\varphi, \theta)$ be the solution of $\nabla^2\Phi=0$. 
Writing the Laplacian in spherical polar coordinates,
\[
\frac{S}{r^2}\frac{d}{dr}\left(r^2\td{R}{r}\right) + 
\frac{R}{r^2}\left[\frac{1}{\sin\theta}\frac{\partial}{\partial\theta}\left(\sin\theta\pdt{S}{\theta}\right) + \frac{1}{\sin^2\theta}\frac{\partial^2S}{\partial\varphi^2}\right] = 0
\]
Multiplying both sides by $r^2/\Phi$,
\[
\frac{1}{R}\frac{d}{dr}\left(r^2\td{R}{r}\right) =
-\frac{1}{S}\left[\frac{1}{\sin\theta}\frac{\partial}{\partial\theta}\left(\sin\theta\pdt{S}{\theta}\right) + \frac{1}{\sin^2\theta}\frac{\partial^2S}{\partial\varphi^2}\right]
\]
Since the left hand side is a function of $r$ alone while the right hand side is a function of $\varphi$ and $\theta$, each side is a constant. We write it as $n(n+1)$ in view of the
fact that the $S$ is a spherical harmonic. Thus, the Laplace's equation splits into
\begin{eqnarray*}
\frac{1}{R}\frac{d}{dr}\left(r^2\td{R}{r}\right) &=& n(n+1) \\
\frac{1}{S}\left[\frac{1}{\sin\theta}\frac{\partial}{\partial\theta}\left(\sin\theta\pdt{S}{\theta}\right) + \frac{1}{\sin^2\theta}\frac{\partial^2S}{\partial\varphi^2}\right] &=& -n(n+1)
\end{eqnarray*}
The second equation can be written as
\[
\frac{1}{\sin\theta}\frac{\partial}{\partial\theta}\left(\sin\theta\pdt{S_n}{\theta}\right) + \frac{1}{\sin^2\theta}\frac{\partial^2S_n}{\partial\varphi^2} + n(n+1)S_n = 0
\]
where we add a subscript $n$ to $S$ following the convention to indicate the eigenvalue. Since $S_n$ is not a function of $r$, we can as well write it as
\[
r^2\nabla^2S_n + n(n+1)S_n = 0
\]
We can readily verify that $R(r) = r^n$ is a solution of
\[
\frac{1}{R}\frac{d}{dr}\left(r^2\td{R}{r}\right) = n(n+1)
\]
Thus, $\Phi_n(r, \varphi, \theta) = r^nS_n(\varphi, \theta)$ is a solution of $\nabla^2\Phi = 0$. The functions $\Phi_n$ are called solid harmonics. We will now show that if $\Phi_n$
solves the Laplace's equation then so does $\Phi_{-n-1}$. To do so, let us find the conditions for $r^m\Phi_n$ to be a solution of the Laplace's equation, where $n$ is fixed. We
observe that
\begin{eqnarray*}
\nabla^2(r^m\Phi_n) &=& (m + n)(m + n + 1)r^{m + n - 2}S_n - n(n + 1)r^{m + n - 2}S_n \\
 &=& m(m + 2n + 1)r^{m - 2}\Phi_n,
\end{eqnarray*}
where we used the fact that $S_n$ is a spherical harmonic. Clearly, $r^m\Phi_n$ is a solution of Laplace's equation only if $m = -(2n + 1)$. Thus $r^{-(2n + 1)}\Phi_n = 
r^{-(2n + 1)}r^nS_n = r^{-n-1}S_n$ is a solution of Laplace's equation.

The only spherically symmetric solutions which vanish at infinity are the ones with $n = -1$. They are
\[
\Phi = \frac{A}{r},
\]
where $A$ is a suitably chosen constant. If $\Phi$ is a solution of Laplace's equation then so are its derivatives. We consider only these solutions in the book.

\section{Green function of Laplacian}
\subsection{Method 1}\label{c2sa6} 
Let $f(x_1, x_2, \ldots, x_n) = g(r)$ where $r = \sqrt{x_1^2 + x_2^2 + \cdots + x_n^2}$. Therefore,
\begin{eqnarray*}
f_{x_1}    &=& g^\op(r) r_x \\
		  &=& g^\op(r)\frac{x_1}{\sqrt{x_1^2 + x_2^2 + \cdots + x_n^2}} \\
f_{x_1x_1} &=& g^\tp(r)\frac{x_1^2}{x_1^2 + x_2^2 + \cdots + x_n^2} + g^\op(r)\frac{1}{\sqrt{x_1^2 + x_2^2 + \cdots + x_n^2}} - \\
           & & g^\op(r)\frac{x_1^2}{(x_1^2 + x_2^2 + \cdots + x_n^2)^{3/2}} \\
f_{x_1x_1} &=& g^\tp(r)\frac{x_1^2}{x_1^2 + x_2^2 + \cdots + x_n^2} + g^\op(r)\frac{x_2^2 + x_3^2 + \cdots + x_n^2}{(x_1^2 + x_2^2 + \cdots + x_n^2)^{3/2}} \\
\end{eqnarray*}
Therefore,
\[
f_{x_1x_1} + f_{x_2x_2} + \cdots + f_{x_nx_n} = g^\tp(r) + (n - 1)\frac{g^\op(r)}{\sqrt{x_1^2 + x_2^2 + \cdots + x_n^2}} 
\]
or,
\[
f_{x_1x_1} + f_{x_2x_2} + \cdots + f_{x_nx_n} = g^\tp(r) + (n - 1)\frac{g^\op(r)}{r} 
\]
If $f_{x_1x_1} + f_{x_2x_2} + \cdots + f_{x_nx_n} = 0$ then 
\[
g^\tp(r) = -(n - 1)\frac{g^\op(r)}{r} \Rightarrow \frac{g^\tp(r)}{g^\op(r)} = -(n - 1)\frac{1}{r}
\]
which implies
\[
\ln g^\op(r) = -(n - 1)\ln r + \ln\alpha,
\]
where $\alpha$ is a constant of integration. Therefore $g^\op(r)r^{n - 1} = \alpha$. We solve this ordinary differential equation for the following three cases
\begin{enumerate}
\item If $n = 2$, $g(r) = \alpha\ln r + \beta$ and
\item If $n \ne 2$, 
\[
g(r) = \frac{\alpha}{r^{n - 2}} + \beta,
\]
\end{enumerate}
where $\alpha$ and $\beta$ are constants of integration. For $n \ge 2$, the function $g$ is singular at the origin. Therefore, it is also the Green function of Laplacian. If 
$r = |\vec{x} - \vec{x}^\op|$ then
\[
G(\vec{x}, \vec{x}^\op) = \begin{cases}
\alpha\ln|\vec{x} - \vec{x}^\op| + \beta & \text{ if } n = 2 \\
\alpha|\vec{x} - \vec{x}^\op|^{-(n - 2)} + \beta & \text{ if } n \ne 2
\end{cases}
\]
 
\subsection{Method 2}\label{c2sa7} 
Green function of a Laplacian is defined as
\[
\nabla^2 G(\vec{x}, \vec{x}^\op) = \begin{cases}
\delta(\vec{x} - \vec{x}^\op) & \text{ on } V \\
0 & \text{ on } S
\end{cases}
\]
where $V$ is the volume enclosed by a surface $S$. When Neumann boundary conditions are specified, the second condition is usually $\grad{G(\vec{x}, \vec{x}^\op)}\cdot\un = A^{-1}$, 
where $A$ is the area of a \enquote*{sphere} of unit radius centerd at $\vec{x}^\op$. To get the Green function, we will integrate $\nabla^2 G(\vec{x}, \vec{x}^\op) = 
\delta(\vec{x} - \vec{x}^\op)$. In each case, the Green function depends only on $|\vec{x} - \vec{x}^\op|$ because of the \enquote*{spherical} symmetry of the problem.
\begin{enumerate}
\item In three dimensions,
\[
\int\nabla^2 G(\vec{x}, \vec{x}^\op) dV = \int \delta(\vec{x} - \vec{x}^\op) dV
\]
or
\[
\int\dive{(\grad{G(\vec{x}, \vec{x}^\op)})}dV = 1
\]
Using divergence theorem,
\[
\int \grad{G(\vec{x}, \vec{x}^\op})\cdot\un dS = 1
\]
If the volume of integration were a sphere of radius $r$, since $G$ is a function of $|\vec{x} - \vec{x}^\op|$ alone, $\grad{G(\vec{x}, \vec{x}^\op}) = G^\op(r)\uvec{r}$. Further, 
since the unit normal $\un$ to the sphere is in the radial direction, $\grad{G(\vec{x}, \vec{x}^\op})\cdot\un = G^\op(r)$. Hence,
\begin{equation}\label{c2sa7e1}
\int G^\op(r) dS = 1
\end{equation}
On a sphere, $G^\op(r)$ is constant, Therefore,
\[
G^\op(r)\int dS = 1 \Rightarrow G^\op 4\pi r^2 = 1
\]
or
\[
G(r) = -\frac{1}{4\pi r} + c,
\]
where $c$ is a constant of integration. Since $r = |\vec{x} - \vec{x}^\op|$,
\[
G(|\vec{x} - \vec{x}^\op|) = -\frac{1}{4\pi|\vec{x} - \vec{x}^\op|}
\]
if we also choose $c = 0$.

\item In two dimensions, the treatment up to \eqref{c2sa7e1} is same as in the case of three dimensions, except that the volume integral is really over a disk and the surface integral is 
over a circle. Therefore,
\[
G^\op(r)\int dS = 1 \Rightarrow G^\op 2\pi r = 1
\]
or
\[
G(r) = \frac{1}{2\pi}\ln r + c,
\]
where $c$ is a constant of integration. Since $r = |\vec{x} - \vec{x}^\op|$,
\[
G(|\vec{x} - \vec{x}^\op|) = \frac{1}{2\pi}\ln|\vec{x} - \vec{x}^\op|
\]
if we also choose $c = 0$.

\item For $n \ge 3$ dimensions, the equation \eqref{c2sa7e1} evaluates to
\[
G^\op(r)\int dS = 1 \Rightarrow G^\op \frac{n \pi^{n/2} r^{n - 1}}{\Gamma(1 + n/2)} = 1,
\]
where we have used the result that the surface area of a $n$-sphere of radius $r$ is{\footnote{Equation 18.77 in \enquote*{Statistical Mechanics} by K. Huang, 2nd edition.}}
\[
S_n(r) = \frac{n \pi^{n/2} r^{n - 1}}{\Gamma(1 + n/2)}
\]
Therefore,
\[
G^\op(r) = \Gamma\left(1 + \frac{n}{2}\right)\frac{1}{n \pi^{n/2}}\frac{1}{r^{n-1}}
\]
or
\[
G^\op(r) = -\Gamma\left(1 + \frac{n}{2}\right)\frac{1}{n(n - 2) \pi^{n/2}}\frac{1}{r^{n - 2}} + c,
\]
where $c$ is a constant of integration. Since $r = |\vec{x} - \vec{x}^\op|$,
\begin{equation}\label{c2sa7e2}
G(|\vec{x} - \vec{x}^\op|) = -\Gamma\left(1 + \frac{n}{2}\right)\frac{1}{n(n - 2) \pi^{n/2}}\frac{1}{|\vec{x} - \vec{x}^\op|^{n - 2}}
\end{equation}
if we also choose $c = 0$. If $n = 3$, we get 
\[
G(|\vec{x} - \vec{x}^\op|) = -\frac{1}{4\pi|\vec{x} - \vec{x}^\op|}
\]

\item Interestingly, the \eqref{c2sa7e2} is valid even for $n = 1$, when we get
\[
 G(|\vec{x} - \vec{x}^\op|) = \frac{1}{2}|\vec{x} - \vec{x}^\op|
\]
\end{enumerate}

\subsection{Method 3}\label{c2sa8}
The definition of Green function is
\begin{equation}\label{c2sa8e1}
\nabla^2 G(\vec{x}, \vec{x}^\op) = \delta^{(n)}(\vec{x} - \vec{x}^\op)
\end{equation}
where $n$ could be any dimension. Thus, in general,
\[
\nabla^2 \equiv \sum_{k=1}^n \frac{\partial^2}{\partial x_k^2}
\]
Without loss of generality, choose $\vec{x}^\op = 0$ and take Fourier transform of \eqref{c2sa8e1} to get
\[
k^2\tilde{G}(\vec{k}) = 1
\]
Thus, the Fourier transform of Green function of Laplacian of any dimensions, in the momentum space is $1/k^2$. Finding an expression in position space is just a matter of finding an
inverse Fourier transform suitable for that space. This technique works for all dimensions except $n = 2$. We know that in two dimensions the Green function of a Laplacian is $\ln r$. 
It diverges as $r \rightarrow \infty$. The theory of Fourier transforms needs the functions to be localized and decaying to zero at infinity. Therefore, the usual techniques of 
integration do not work in this case. Refer to the \href{http://math.stackexchange.com/questions/847706/does-the-integral-in-the-formal-2d-fourier-transform-of-the-logarithm-converge}
{StackExchange} conversation for more details{\footnote{Dr. Robert E. Hunt of Cambridge University, in a private correspondence, explained this point to me and directed me to the
StackExchange conversation}}.

\section{Divergence theorem in a plane}\label{c2sa9} 
Consider a vector field $\vec{F}(x, y)$ defined over a closed region $S$ whose boundary is denoted by $\partial S$. If $ds$ denotes a small displacement along the boundary, the tangent
$\ut$ and the normal $\un$ are defined as
\begin{eqnarray*}
\ut &=& \td{x}{s}\uvec{x} + \td{y}{s}\uvec{y} \\
\un     &=& \td{y}{s}\uvec{x} - \td{x}{s}\uvec{y}
\end{eqnarray*}
Refer to \href{http://mathworld.wolfram.com/NormalVector.html}{Mathworld} for more details. The circulation of $\vec{F} = L\uvec{x} + M\uvec{y}$ is defined as
\[
\oint\vec{F}\cdot\ut ds = \oint\left(L\td{x}{s} + M\td{y}{s}\right)ds = \oint(Ldx + Mdy)
\]
while the flux through $\partial S$ is
\[
\oint\vec{F}\cdot\un ds = \oint\left(L\td{y}{s} - M\td{x}{s}\right)ds = \oint(Ldy - Mdx)
\]
Green's theorem in a plane is
\[
\oint (Ldx + Mdy) = \iint \left(\pdt{M}{x} - \pdt{L}{y}\right) dxdy
\]
Therefore,
\begin{eqnarray*}
\oint\vec{F}\cdot\ut ds &=& \iint\left(\pdt{M}{x} - \pdt{L}{y}\right)dxdy \\
\oint\vec{F}\cdot\un ds     &=& \iint\left(\pdt{L}{x} + \pdt{M}{y}\right)dxdy
\end{eqnarray*}
However, the circulation is a line integral of the curl while the flux is a \enquote*{surface} integral of the field. Therefore, the above equations can as well be written as
\begin{eqnarray*}
\oint\vec{F}\cdot\ut ds &=& \iint\left(\curl{F}\right)_zdxdy \\
\oint\vec{F}\cdot\un ds     &=& \iint(\dive{\vec{F}})dxdy
\end{eqnarray*}

\section{Circular harmonics}\label{c2sa10} 
Consider Laplace's equation in plane polar coordinates for $\Phi(r, \theta)$,
\[
\frac{1}{r}\frac{\partial}{\partial r}\left(r\pdt{\Phi}{r}\right) + \frac{1}{r^2}\pdts{\Phi}{\theta} = 0
\]
Let $\Phi(r, \theta) = R(r)S(\theta)$ so that
\[
\frac{r}{R}\frac{d}{dr}\left(r\td{R}{r}\right) = -\frac{1}{S}\frac{d^2S}{d\theta^2}
\]
Since the left side depends on $r$ alone while the right hand side depends on $\theta$, each side is equal to a constant, say $m$. Thus,
\[
-\frac{1}{S}\frac{d^2S}{d\theta^2} = m \Rightarrow \frac{d^2S}{d\theta^2} + mS = 0
\]
Since $S(\theta) = S(\theta + 2\pi)$, we require $m$ to be square of an integer. Let us therefore write $m = n^2$. The $R$ equation is
\[
\frac{r}{R}\frac{d}{dr}\left(r\td{R}{r}\right) = m^2
\]
It is easy to check that if $m^2 \ne 0$ then $R = r^l$ is a solution while if $m = 0$, $R = \ln r$ is a solution. Thus, the complete solution of Laplace's equation is
\begin{eqnarray*}
\Phi(r, \theta) &=& (a_0 + b_0\ln r)(c_0 + d_0\theta) + \\
 & & \sum_{k \ne 0}\left(a_k r^k + b_k r^{-k}\right)\left(c_ke^{ik\theta} + d_ke^{-ik\theta}\right),
\end{eqnarray*}
where $a_k, b_k, c_k, d_k$ are constants. $\Phi(r, \theta) = \Phi(r, \theta + 2\pi)$ requires $d_0 = 0$ so that,
\[
\Phi(r, \theta) = (a_0 + b_0\ln r) + \sum_{k \ne 0}\left(a_k r^k + b_k r^{-k}\right)\left(c_ke^{ik\theta} + d_ke^{-ik\theta}\right),
\]
If the solution is required to be \enquote*{circularly symmetric}, that is independent of polar angle $\theta$ then we have to choose $a_k, b_k, c_k, d_k = 0$ for all $k \ne 0$, in
which case,
\[
\Phi(r, \theta) = a_0 + b_0\ln r
\]
Solutions of Laplace's equation in plane polar coordinates are called circular harmonics. In the book considers only those which are \enquote*{circularly symmetric}.

Recall that if $f(r)$ is a solution of Laplace's equation then so are all derivatives of $f$ with respect to $r$. 

\section{A property of harmonic functions}\label{c2sa11}
Let $\phi_1(\vec{x})$ and $\phi_2(\vec{x})$ be two solution of Laplace's equation. That it, $\nabla^2\phi_1 = 0$ and $\nabla^2\phi_2 = 0$. Consider
\[
\dive\left(\phi_1\grad\phi_2\right) = \grad\phi_1\cdot\grad\phi_2 + \phi_1\nabla^2\phi_2 = \grad\phi_1\cdot\grad\phi_2,
\]
because $\nabla^2\phi_2 = 0$. Similarly,
\[
\dive\left(\phi_2\grad\phi_1\right) = \grad\phi_2\cdot\grad\phi_1 + \phi_2\nabla^2\phi_1 = \grad\phi_1\cdot\grad\phi_2,
\]
because $\nabla^2\phi_1 = 0$. Therefore, $\dive\left(\phi_1\grad\phi_2\right) = \dive\left(\phi_2\grad\phi_1\right)$. Taking a volume integral of the equation,
\[
\int\dive\left(\phi_1\grad\phi_2\right)dV = \int\dive\left(\phi_2\grad\phi_1\right)dV,
\]
after using divergence theorem,
\begin{equation}\label{c2sa11e1}
\int\phi_1\grad\phi_2\cdot\un dA = \int\phi_2\grad\phi_1\cdot\un dA
\end{equation}

\section{Proof of \texorpdfstring{$\int n_i n_j d\Omega = (4\pi/3)\delta_{ij}$}{}}
Consider the integral,
\[
I_{ij} = \int n_i n_j d\Omega
\]
where $n_i, n_j$ are unit normals and $d\Omega$ denotes the solid angle. We will evaluate this integral for the following sub-cases, which are all the ones that are possible.
\begin{itemize}
\item $i = j = x$. In this case, putting $n_i = n_j = \sin\theta\cos\varphi$,
\[
I_{xx} = \int_0^\pi\int_0^{2\pi} \sin^2\theta \cos^2\varphi \sin\theta d\theta d\varphi
\]
We observe that
\[
\int_0^{2\pi} \cos^2\varphi d\varphi = \int_0^{2\pi}\left(\frac{1 + \cos(2\varphi)}{2}\right)d\varphi = \pi
\]
and
\begin{eqnarray*}
\int_0^{\pi}\sin^3\theta d\theta &=& \int_0^\pi(1 - \cos^2\theta)\sin\theta d\theta \\
 &=& -\int_{1}^{-1} (1 - u^2)(-du) \\
 &=& \int_{-1}^1(1 - u^2)du \\
 &=& 2 - \frac{2}{3} = \frac{4}{3}
\end{eqnarray*}
Therefore,
\[
I_{xx} = \frac{4\pi}{3}
\]

\item $i = j = y$. In this case, put $n_i = n_j = \sin\theta\sin\varphi$, so that
\[
I_{yy} = \int_0^\pi\int_0^{2\pi} \sin^2\theta \sin^2\varphi \sin\theta d\theta d\varphi
\]
Since
\[
\int_0^{2\pi} \sin^2\varphi d\varphi = \int_0^{2\pi}\left(\frac{1 - \cos(2\varphi)}{2}\right)d\varphi = \pi,
\]
we get
\[
I_{yy} = \frac{4\pi}{3}
\]

\item $i = j = z$. In this case, put $n_i = n_j = \cos\theta$, so that
\[
I_{zz} = \int_0^\pi\int_0^{2\pi} \cos^2\theta \sin\theta d\theta d\varphi = 2\pi\int_{-1}^1 u^2 (-du) = \frac{4\pi}{3}
\]

\item $i = x, j = y$. In this case, put $n_i = \sin\theta\cos\varphi$ and $n_j = \sin\theta\sin\varphi$, so that
\[
I_{xy} = \int_0^{\pi}\sin^3\theta d\theta \int_0^{2\pi} \sin\varphi\cos\varphi d\varphi
\]
The $\varphi$ integral is zero making $I_{xy} = 0$.

\item $i = y, j = z$. In this case, put $n_i = \sin\theta\sin\varphi$ and $n_j = \cos\theta$, so that
\[
I_{yz} = \int_0^{\pi}\sin^2\theta\cos\theta d\theta \int_0^{2\pi} \sin\varphi d\varphi
\]
The $\varphi$ integral is zero making $I_{yz} = 0$.

\item $i = z, j = x$. In this case, put $n_i = \cos\theta\sin\varphi$ and $n_j = \sin\theta\cos\phi$, so that
\[
I_{zx} = \int_0^{\pi}\sin^2\theta\cos\theta d\theta \int_0^{2\pi} \cos\varphi d\varphi
\]
The $\varphi$ integral is zero making $I_{zx} = 0$.
The six cases above can be summarized as
\begin{equation}\label{c3sae1}
\int n_i n_j d\Omega = \frac{4\pi}{3}\delta_{ij}
\end{equation}
\end{itemize}

\section{Proof of \texorpdfstring{$\int n_i n_j n_k n_l d\Omega = (4\pi/15)(\delta_{ij}\delta_{kl}+\delta_{ik}\delta_{jl}+\delta_{il}\delta_{jk})$}{}}
We will consider the following cases,
\begin{itemize}
\item All indices are the same. In this case, the integral is one of $x^4$, $y^4$ or $z^4$, where $x = \sin\theta\cos\varphi$, $y = \sin\theta\sin\varphi$ and $z = \cos\theta$.
We first consider,
\begin{eqnarray*}
I_{xxxx} &=& \int_0^\pi \int_0^{2\pi} \sin^4\theta \cos^4\varphi \sin\theta d\theta d\varphi \\
 &=& \int_0^\pi\sin^5\theta d\theta \int_0^{2\pi}\cos^4\varphi d\varphi
\end{eqnarray*}
Now,
\begin{eqnarray*}
\int_0^\pi\sin^5\theta d\theta &=& \int_0^\pi\sin^4\theta \sin\theta d\theta \\
 &=& \int_0^\pi(1 - \cos^2\theta)^5 \sin\theta d\theta \\
 &=& -\int_1^{-1}(1 - 2u^2 + u^4)du \\
 &=& \frac{16}{15}
\end{eqnarray*}
and
\begin{eqnarray*}
\int_0^{2\pi}\cos^4\varphi d\varphi &=& \int_0^{2\pi}\left(\frac{3}{8} + \frac{\cos 2\varphi}{2} + \frac{\cos 4\varphi}{8}\right)d\varphi \\
 &=& \frac{3\pi}{4}
\end{eqnarray*}
Therefore,
\[
I_{xxxx} = \frac{4\pi}{5}
\]
Now consider,
\begin{eqnarray*}
I_{yyyy} &=& \int_0^\pi \int_0^{2\pi} \sin^4\theta \sin^4\varphi \sin\theta d\theta d\varphi \\
 &=& \int_0^\pi\sin^5\theta d\theta \int_0^{2\pi}\sin^4\varphi d\varphi
\end{eqnarray*}
We evaluate
\begin{eqnarray*}
\int_0^{2\pi}\sin^4\varphi d\varphi &=& \int_0^{2\pi}\left(\frac{3}{8} - \frac{\cos 2\varphi}{2} + \frac{\cos 4\varphi}{8}\right)d\varphi \\
 &=& \frac{3\pi}{4}
\end{eqnarray*}
Therefore,
\[
I_{yyyy} = \frac{4\pi}{5}
\]
Finally, consider
\begin{eqnarray*}
I_{zzzz} &=& \int_0^\pi \int_0^{2\pi} \cos^4\theta \sin\theta d\theta d\varphi \\
 &=& \int_0^\pi \cos^4\theta \sin\theta \int_0^{2\pi}d\varphi \\
 &=& 2\pi\int_1^{-1}u^4(-du) \\
 &=& \frac{4\pi}{5}
\end{eqnarray*}
Thus, this point covers $3$ cases.

\item Three indices are same, one is different. This case includes integrands of the form $xyyy, xzzz, yzzz, yxxx, zyyy, zxxx$. We evaluate each on of them.
\begin{eqnarray*}
I_{xyyy} &=& \int_0^\pi\sin^5\theta d\theta \int_0^{2\pi}\sin^3\varphi\cos\varphi d\varphi \\
 &=& \frac{16}{15} \times 0 \\
 &=& 0 \\
I_{xzzz} &=& \int_0^\pi\cos^3\theta\sin^2\theta d\theta \int_0^{2\pi}\cos\varphi d\varphi \\
 &=& \int_0^\pi\cos^3\theta\sin^2\theta d\theta \times 0 \\
 &=& 0 \\
I_{yzzz} &=& \int_0^\pi\cos^3\theta\sin^2\theta d\theta \int_0^{2\pi}\sin\varphi d\varphi \\
 &=& \int_0^\pi\cos^3\theta\sin^2\theta d\theta \times 0 \\
 &=& 0 \\
I_{yxxx} &=& \int_0^\pi\sin^5\theta d\theta \int_0^{2\pi}\sin\varphi\cos^3\varphi d\varphi \\
 &=& \frac{16}{15} \times 0 \\
 &=& 0 \\
I_{zxxx} &=& \int_0^\pi\sin^3\theta \cos\theta d\theta \int_0^{2\pi}\sin^3\varphi d\varphi \\
 &=& 0 \times \int_0^{2\pi}\sin^3\varphi d\varphi \\
 &=& 0\\
I_{zyyy} &=& \int_0^\pi\sin^3\theta \cos\theta d\theta \int_0^{2\pi}\cos^3\varphi d\varphi \\
 &=& 0 \times \int_0^{2\pi}\cos^3\varphi d\varphi \\
 &=& 0
\end{eqnarray*}
Note that 
\[
I_{xyyy} = I_{yxyy} = I_{yyxy} = I_{yyyx} = 0
\]
and similarly for other cases. Thus, this point covers $6 \times 4 = 24$ cases.

\item Two of them of one kind and two of the other. We therefore, consider the cases
\begin{eqnarray*}
I_{xxyy} &=& \int_0^\pi\sin^5\theta d\theta \int_0^{2\pi}\sin^2\varphi\cos^2\varphi d\varphi \\
 &=& \frac{16}{15}\int_0^{2\pi}\frac{\cos^2(2\varphi)}{4}d\varphi \\
 &=& \frac{16}{15}\int_0^{2\pi}\frac{1 + \cos(4\varphi)}{8}d\varphi \\
 &=& \frac{16}{15}\times\frac{\pi}{4} \\
 &=& \frac{4\pi}{15} \\
I_{xxzz} &=& \int_0^\pi\sin^3\theta\cos^2\theta d\theta \int_0^{2\pi}\cos^2\varphi d\varphi \\
 &=& \left(\int_0^\pi\sin^3\theta d\theta - \int_0^\pi\sin^5\theta d\theta\right)\int_0^{2\pi}\frac{1 + \cos(2\varphi)}{2}d\varphi \\
 &=& \left(\frac{4}{3} - \frac{16}{15}\right)\pi \\
 &=& \frac{4\pi}{15} \\
I_{yyzz} &=& \int_0^\pi\sin^3\theta\cos^2\theta d\theta \int_0^{2\pi}\sin^2\varphi d\varphi \\
 &=& \left(\int_0^\pi\sin^3\theta d\theta - \int_0^\pi\sin^5\theta d\theta\right)\int_0^{2\pi}\frac{1 - \cos(2\varphi)}{2}d\varphi \\
 &=& \left(\frac{4}{3} - \frac{16}{15}\right)\pi \\
 &=& \frac{4\pi}{15} \\
\end{eqnarray*}
Note that
\[
I_{xxyy} = I_{xyyx} = I_{yxyx} = I_{xyxy} = I_{yxxy} = I_{yyxx} = \frac{4\pi}{15}
\]
and similarly for other cases. Thus, this point covers $3 \times 6 = 18$ cases.

\item Two of them of one kind and the remaining of two different kinds. We therefore, consider the cases
\begin{eqnarray*}
I_{xxyz} &=& \int_0^\pi\sin^4\theta \cos\theta d\theta \int_0^{2\pi}\cos^2\varphi \sin\varphi d\varphi \\
 &=& 0 \times 0 \\
 &=& 0 \\
I_{yyxz} &=& \int_0^\pi\sin^4\theta \cos\theta d\theta \int_0^{2\pi}\cos\varphi \sin^2\varphi d\varphi \\
 &=& 0 \times 0 \\
 &=& 0 \\
I_{zzxy} &=& \int_0^\pi\sin^2\theta \cos^3\theta d\theta \int_0^{2\pi}\cos\varphi\sin\varphi d\varphi \\
 &=& \int_0^\pi\sin^2\theta \cos^3\theta d\theta \times 0 \\
 &=& 0
\end{eqnarray*}
Note that
\[
I_{xxyz} = I_{xyzx} = I_{xyxz} = I_{xzyx} = I_{xzxy} = I_{xxzy} = I_{yzxx} = I_{yxxz} = 
\]
\[
I_{zxxy} = I_{zxyx} = I_{zyxx} = I_{yxzx}
\]
and similarly for the other cases. Thus, this point covers $3 \times 12 = 36$ cases.
\end{itemize}
The four points cover $3 + 24 + 18 + 36 = 81$ cases, which are all the cases possible. We can summarize all of them in the equation,
\begin{equation}\label{c3sae2}
\int n_i n_j n_k n_l d\Omega = \frac{4\pi}{15}(\delta_{ij}\delta_{kl}+\delta_{ik}\delta_{jl}+\delta_{il}\delta_{jk})
\end{equation}

\section{Isotropic tensors}\label{mr12}
We will examine the nature of isotropic tensors of orders $0, 1, 2, 3$ and $4$. A tensor is isotropic if it is invariant under a rotation of axes. The treatment in this section follows
that in H. Jeffrey's \enquote*{Cartesian Tensors}\cite{jeffreys1961cartesian}.

\begin{itemize}
\item A tensor of zero order is a scalar and it is invariant under a rotation of coordinate axes. 

\item We will show that there are no isotropic tensors of order $1$. A rotation of coordinate axes by and angle $\theta$ about the $z$ axis is represented by a the orthogonal matrix,
\[
\begin{pmatrix}
\cos\theta & \sin\theta & 0 \\
-\sin\theta & \cos\theta & 0 \\
0 & 0 & 1
\end{pmatrix}
\]
For small angles of rotation, it can be approximated as
\[
\begin{pmatrix}
1 & \theta & 0 \\
-\theta & 1 & 0 \\
0 & 0 & 1
\end{pmatrix}
\]
and decomposed as $\delta_{ij} + c_{ij}$, where the anti-symmetric matrix $\{c_{ij}\}$ is
\[
\begin{pmatrix}
0 & \theta & 0 \\
-\theta & 0 & 0 \\
0 & 0 & 0
\end{pmatrix}
\]
We have thus shown that any small rotation can be represented by $\delta_{ij} + c_{ij}$, where $\{c_{ij}\}$ is an anti-symmetric matrix. A vector $u_i$ under a small rotation is
transformed to
\[
u_j^\op = (\delta_{ij} + c_{ij})u_i
\]
It will remain unvariant under the rotation only if $c_{ij}u_i = 0$. Writing in full,
\begin{eqnarray*}
 0.u_1 + c_{12}u_2 + c_{13}u_3 &=& 0 \\
-c_{12}u_1 + 0.u_2 + c_{23}u_3 &=& 0 \\
-c_{13}u_1 - c_{23}u_2 + 0.u_3 &=& 0
\end{eqnarray*}
Since
\[
\begin{vmatrix}
 0 & c_{12} & c_{13} \\
-c_{12} & 0 & c_{23} \\
-c_{13} & -c_{23} & 0
\end{vmatrix} = 0
\]
the only solution to $c_{ij}u_i = 0$ is $u_i = 0$. Thus, the only isotropic vector is the null vector.

\item Let $u_{ij}$ be a second order tensor. Then $u_{kl}^\op = (\delta_{ik} - c_{ik})u_{ij}(\delta_{jl} - c_{jl})$. Up to first order in $c_{ij}$,
\[
u_{kl}^\op = u_{kl} - u_{kj}c_{jl} - u_{il}c_{ik}
\]
The tensor, $u_{ij}$ will be isotropic only if $u_{kj}c_{jl} + u_{il}c_{ik} = 0$. If $k \ne l$, choose $k = 1$ and $l = 2$ so that $u_{1j}c_{j2} + u_{i2}c_{i1} = 0$ or
\[
u_{11}c_{12} + u_{12}c_{22} + u_{12}c_{32} + u_{12}c_{11} + u_{22}c_{21} + u_{32}c_{31} = 0
\]
Using symmetry properties of $\{c_{ij}\}$,
\[
u_{11}c_{12} - u_{12}c_{23} - u_{22}c_{12} - u_{32}c_{13} = 0
\]
or
\[
(u_{11} - u_{22})c_{12} - u_{12}c_{23} - u_{32}c_{13} = 0
\]
This can be true, in general, only if $u_{11} = u_{22}$ and $u_{12} = u_{32} = 0$. Similarly, by choosing $k = 1$ and $l = 3$, we get $u_{11} = u_{33}$ and $u_{12} = u_{23} = 0$. 
Further, choosing $k = 2$ and $l = 3$ gives $u_{22} = u_{33}$ and $u_{21} = u_{13} = 0$.

If we choose $k = l = 1$, $u_{1j}c_{j1} + u_{i1}c_{i1} = 0$ or
\[
u_{11}c_{11} + u_{12}c_{21} + u_{13}c_{31} + u_{11}c_{11} + u_{21}c_{21} + u_{31}c_{31} = 0
\]
Using symmetry properties of $\{c_{ij}\}$,
\[
(u_{12} + u_{21})c_{21} + (u_{13} + u_{31})c_{31} = 0
\]
which can be true, in general, only if $u_{12} = -u_{21}$ and $u_{13} = -u_{31}$. But we have already shown that $u_{12} = 0$ and $u_{13} = 0$. Thus, $u_{21}$ and $u_{31}$ also vanish.
Choosing $k = l = 3$ we can show that $u_{23} + u_{32} = 0$, which since $u_{23} = 0$ implies vanishing of both. We have thus shown that the only isotropic tensor is a multiple of the
Kronecker delta $\delta_{ij}$.

\item Let $u_{ijk}$ be a third order tensor. Then
\[
u_{lmn}^\op = (\delta_{il} - c_{il})(\delta_{jm} - c_{jm})(\delta_{kn} - c_{kn})u_{ijk}
\]
Up to first order in $c_{ij}$,
\[
u_{lmn}^\op = u_{lmn} - \delta_{il}\delta_{jm}c_{kn}u_{ijk} - \delta_{il}\delta_{kn}c_{jm}u_{ijk} - \delta_{jm}\delta_{kn}c_{il}u_{ijk},
\]
or
\[
u_{lmn}^\op = u_{lmn} - c_{kn}u_{lmk} - c_{jm}u_{ljn} - c_{il}u_{imn}
\]
The tensor $u_{ijk}$ will be isotropic only if $u_{lmn}^\op = u_{lmn}$ or, if
\[
c_{kn}u_{lmk} + c_{jm}u_{ljn} + c_{il}u_{imn} = 0
\]
If $l = m = 1$, the above equation becomes $c_{kn}u_{11k} + c_{j1}u_{1jn} + c_{i1}u_{i1n} = 0$. Summing it over all indices, except $n$, we get
\begin{eqnarray*}
c_{1n}u_{111} + c_{2n}u_{112} + c_{3n}u_{113} + c_{11}u_{11n} + c_{21}u_{12n}  + & & \\
c_{31}u_{13n} + c_{11}u_{11n} + c_{21}u_{21n} + c_{31}u_{31n} &=& 0
\end{eqnarray*}
Using symmetry properties of $\{c_{ij}\}$,
\[
c_{1n}u_{111} + c_{2n}u_{112} + c_{3n}u_{113} + c_{21}(u_{12n} + u_{21n}) + c_{31}(u_{13n} + u_{31n})= 0
\]
In the particular case of $n = 2$, we get
\[
c_{12}(u_{111} - u_{122} - u_{212}) + c_{32}u_{113} + c_{31}(u_{132} + u_{312}) = 0
\]
This can be true, in general, only of
\begin{eqnarray*}
u_{122} + u_{212} &=& u_{111} \\
u_{132} + u_{312} &=& 0 \\
u_{113} &=& 0
\end{eqnarray*}
From the last of the above three equations, $u_{ijk} = 0$ if any two of the three indices are same. This fact, coupled with the first of the three equations tells that $u_{ijk} = 0$
if all indices are equal. Finally, the second of the above equations tells that $u_{ijk} = -u_{jik}$. Thus, the tensor entry for $(i, j, k)$ changes sign under an odd permutation of
indices. Thus, the only isotropic tensor of third order is a scalar multiple of Levi-Civita tensor $\epsilon_{ijk}$.

\item Let $u_{ijkl}$ be a fourth order tensor and let
\[
u_{pqrs}^\op = (\delta_{ip} - c_{ip})(\delta_{jq} - c_{jq})(\delta_{kr} - c_{kr})(\delta_{ls} - c_{ls})u_{ijkl}
\]
Then, up to first order in $c_{ij}$,
\begin{eqnarray*}
u_{pqrs}^\op &=& \delta_{ip}\delta_{jq}\delta_{kr}\delta_{ls}u_{ijkl} - c_{ip}\delta_{jq}\delta_{kr}\delta_{ls}u_{ijkl} \\
 & & - \delta_{ip}c_{jq}\delta_{kr}\delta_{ls}u_{ijkl} - \delta_{ip}\delta_{jq}c_{kr}\delta_{ls}u_{ijkl} -\delta_{ip}\delta_{jq}\delta_{kr}c_{ls}u_{ijkl} \\
 &=& u_{pqrs} - c_{ip}u_{iqrs} - c_{jq}u_{pjrs} - c_{kr}u_{pqks} - c_{ls}u_{pqrl}
\end{eqnarray*}
The tensor $u_{ijkl}$ will be isotropic only if
\begin{equation}\label{mr12e1}
c_{ip}u_{iqrs} + c_{jq}u_{pjrs} + c_{kr}u_{pqks} + c_{ls}u_{pqrl} = 0
\end{equation}
There are three possible values for the four indices $i, j, k, l$. They give rise to four possibilities:
\begin{enumerate}
\item Two indices are equal and the other two are unequal. Choose $p = q = 1$, $r = 2$ and $s = 3$ so that \eqref{mr12e1} becomes
\[
c_{i1}u_{i123} + c_{j1}u_{1j23} + c_{k2}u_{11k3} + c_{l3}u_{112l} = 0
\]
which is same as
\begin{eqnarray*}
c_{11}u_{1123} + c_{21}u_{2123} + c_{31}u_{3123} + c_{11}u_{1123} &+& = 0 \\ 
c_{21}u_{1223} + c_{31}u_{1323} + c_{12}u_{1113} + c_{22}u_{1123} &+& \\
c_{32}u_{1133} + c_{13}u_{1121} + c_{23}u_{1122} + c_{33}u_{1123} & &
\end{eqnarray*}
Using symmetry properties of $\{c_{ij}\}$,
\begin{eqnarray*}
c_{12}(u_{1113} - u_{2123} - u_{1223}) + c_{13}(u_{1121} - u_{3123} - u_{1323}) + &=& 0 \\
c_{23}(u_{1122} - u_{1133}) & &
\end{eqnarray*}
This equation will be valid for an arbitrary $c_{ij}$ only if
\begin{eqnarray}
u_{1113} - u_{2123} - u_{1223} &=& 0 \label{mr12e2} \\
u_{1121} - u_{3123} - u_{1323} &=& 0 \label{mr12e3} \\
u_{1122} - u_{1133} &=& 0 \label{mr12e4}
\end{eqnarray}
The last of the above equations gives $u_{1122} = u_{1133}$. Choosing $p = r = 1$, $q = 2$ and $s = 3$ will give $u_{1212} = u_{1313}$. Other combinations will similar identities. In
general,
\begin{equation}\label{mr12e5}
u_{1122} = u_{1133} = u_{2211} = u_{2233} = u_{3311} = u_{3322}
\end{equation}
\begin{equation}\label{mr12e6}
u_{1212} = u_{1313} = u_{2121} = u_{2323} = u_{3131} = u_{3232}
\end{equation}
\begin{equation}\label{mr12e7}
u_{1221} = u_{1331} = u_{2112} = u_{2332} = u_{3113} = u_{3223}
\end{equation}
Thus, all terms of $u_{ijkl}$ where two indices are of one kind and the remaining two are of the other kind are equal.

\item Three indices are equal and the remaining one is different from them. Choose $p = q = r = 1$ and $s = 2$ so that \eqref{mr12e1} becomes
\[
c_{i1}u_{i112} + c_{j1}u_{1j12} + c_{k1}u_{11k2} + c_{l2}u_{111l} = 0
\]
which is same as
\begin{eqnarray*}
c_{11}u_{1112} + c_{21}u_{2112} + c_{31}u_{3112} + c_{11}u_{1112} &+& = 0 \\
c_{21}u_{1212} + c_{31}u_{1312} + c_{11}u_{1112} + c_{21}u_{1122} &+& \\
c_{31}u_{1132} + c_{12}u_{1111} + c_{22}u_{1112} + c_{32}u_{1113} & & 
\end{eqnarray*}
Using symmetry properties of $\{c_{ij}\}$,
\begin{eqnarray*}
c_{12}(u_{1111} - u_{2112} - u_{1212} - u_{1122}) - &=& 0 \\
c_{13}(u_{3112} + u_{1312} + u_{1132}) - c_{23}u_{1113} & &
\end{eqnarray*}
This equation can be true for arbitrary $c_{ij}$ only if
\begin{eqnarray}
u_{1111} &=& u_{2112} + u_{1212} + u_{1122} \label{mr12e8} \\
0 &=& u_{3112} + u_{1312} + u_{1132} \label{mr12e9} \\
0 &=& u_{1113} \label{mr12e10}
\end{eqnarray}
The last of the above equations tells that all terms of $u_{ijkl}$ where three indices are same and one differs from the rest are zero. Using this fact in \eqref{mr12e2} and 
\eqref{mr12e3}, we get
\begin{eqnarray}
u_{2123} + u_{1223} &=& 0 \label{mr12e11} \\
u_{3123} + u_{1323} &=& 0 \label{mr12e12}
\end{eqnarray}
\eqref{mr12e11} under the transformation $1 \mapsto 3, 2\mapsto 1, 3 \mapsto 2$ becomes
\[
u_{1312} + u_{3112} = 0,
\]
which put with \eqref{mr12e9} gives $u_{3112} = 0$. By symmetry, all components of $u_{ijkl}$ whose two indices are same and the rest two are different are zero. Further, equation 
\eqref{mr12e8} tells that all components of $u_{ijkl}$, all whose indices are equal, are written as sums of components two of whose indices are of one value and the rest are of other
value.
\end{enumerate}
To summarize:
\begin{itemize}
\item Components of $u_{ijkl}$ where three indices are same and one differs from the rest are zero.
\item Components of $u_{ijkl}$ whose two indices are same and the rest two are different are zero.
\item Components of $u_{ijkl}$, all whose indices are equal, are written as sums of components, two of whose indices are of one value and the rest are of other value.
\item Components of $u_{ijkl}$, two of whose values are equal and rest are of other value satisfy equations \eqref{mr12e5}, \eqref{mr12e6} and \eqref{mr12e7}.
\end{itemize}
Let
\[
u_{1122} = u_{1133} = u_{2211} = u_{2233} = u_{3311} = u_{3322} = \gamma 
\]
\[
u_{1212} = u_{1313} = u_{2121} = u_{2323} = u_{3131} = u_{3232} = a
\]
\[
u_{1221} = u_{1331} = u_{2112} = u_{2332} = u_{3113} = u_{3223} = b, 
\]
where $\gamma, a, b$ are scalars. For sake of convenience, we can introduce two new scalars
\begin{eqnarray*}
\alpha &=& \frac{a + b}{2} \\
\beta &=& \frac{a - b}{2}
\end{eqnarray*}
so that
\begin{equation}\label{mr12e13}
u_{1122} = u_{1133} = u_{2211} = u_{2233} = u_{3311} = u_{3322} =  \gamma
\end{equation}
\begin{equation}\label{mr12e14}
u_{1212} = u_{1313} = u_{2121} = u_{2323} = u_{3131} = u_{3232} = \alpha + \beta
\end{equation}
\begin{equation}\label{mr12e15}
u_{1221} = u_{1331} = u_{2112} = u_{2332} = u_{3113} = u_{3223} = \alpha - \beta 
\end{equation}
From \eqref{mr12e8},
\[
u_{1111} = (\alpha - \beta) + (\alpha + \beta) + \gamma = \gamma + 2\alpha
\]
By symmetry, we conclude that
\begin{equation}\label{mr12e16}
u_{1111} = u_{2222} = u_{3333} = \gamma + 2\alpha
\end{equation}

From \eqref{mr12e14} and \eqref{mr12e15}, we observe that $u_{1212} = \gamma + \beta$ and $u_{1221} = \gamma - \beta$ so that
\begin{eqnarray*}
u_{1212} + u_{1221} &=& 2\alpha \\
u_{1212} - u_{1221} &=& 2\beta
\end{eqnarray*}
Thus, there are two tensors, $u_{ijij} + u_{ijji}$ contributing $2\alpha$ and $u_{ijij} - u_{ijji}$ contributing $2\beta$. Equivalently, we can say that the tensor $w_{ijkl}$ is such that
$w_{ijij}$ and $w_{ijji}$ each contribute $\alpha$ and the other choices of indices contribute zero, while the tensor $v_{ijkl}$ is such that $v_{ijij}$ contributes $\beta$, $v_{ijji}$ 
contributes $-\beta$  and the other choices of indices contribute zero.

Equations \eqref{mr12e13}, \eqref{mr12e14} and \eqref{mr12e15} suggest that there are three independent isotropic tensors described as
\begin{enumerate}
\item $u_{ijkl} = 1$ if $i=j$ and $k=l$ and zero other otherwise. It can be represented as $\gamma\delta_{ij}\delta_{kl}$.
\item $u_{ijkl} = 1$ if $i=k$ and $j=l$ or $i=l$ and $j=k$ and $i \ne j$. $u_{ijkl} = 0$ for all other choices of indices. Further, if all indices are same $u_{ijkl} = 2$. It can be 
represented as $\alpha(\delta_{ik}\delta_{jl} + \delta_{il}\delta{jk})$.
\item $u_{ijkl} = 1$ if $i=k$ and $j=l$, $u_{ijkl} = -1$ if $i=l$ and $j=k$ and $u_{ijkl} = 0$ for all other choices of indices. It can be represented as $\beta(\delta_{ik}\delta_{jl} - 
\delta_{il}\delta{jk})$.
\end{enumerate}
Therefore, a general isotropic tensor of 4th order is expressed as
\begin{equation}\label{mr12e17}
u_{ijkl} = \gamma\delta_{ij}\delta_{kl} + \alpha(\delta_{ik}\delta_{jl} + \delta_{il}\delta_{jk}) + \beta(\delta_{ik}\delta_{jl} - \delta_{il}\delta_{jk})
\end{equation}
An equivalent form is
\[
u_{ijkl} = \mu^\tp\delta_{ij}\delta_{kl} + \mu\delta_{ik}\delta_{jl} + \mu^\op\delta_{il}\delta_{jk}
\]
or,
\begin{equation}\label{mr12e18}
u_{ijkl} = \mu\delta_{ik}\delta_{jl} + \mu^\op\delta_{il}\delta_{jk} + \mu^\tp\delta_{ij}\delta_{kl}
\end{equation}
\end{itemize}

\section{Self-adjoint Operators}\label{mr13}
Consider a differential equation of the form\cite{arfken1999mathematical}
\begin{equation}\label{mr13e1}
p_0(x)u^\tp + p_1(x)u^\op + p_2(x)u(x) = 0,
\end{equation}
where $p_0, p_1$ and $p_2$ are given functions of $x$ and $u$ is the unknown function. The differential equation is defined over a neighbourhood $[a, b]$ and the functions $p_0, p_1$ and $p_2$ are 
assumed to be defined on it. We further assume that the function $p_i$ has at least $2 - i$ continuous derivatives on $[a, b]$. The function $y$ will also have certain boundary conditions to be
satisfied. The equation can be written as $Lu = 0$ where the linear operator $L$ is defined as
\begin{equation}\label{mr13e2}
L = p_0D^2 + p_1D + p_2,
\end{equation}
where $D$ denotes the differential operator $d/dx$. The inner product of functions $u$ and $v$ is defined as
\begin{equation}\label{mr13e3}
(u, v) = \int_a^b uv dx,
\end{equation}
so that
\begin{equation}\label{mr13e4}
(u, Lu) = \int_a^b u(x)\left(p_0(x)u^\tp + p_1(x)u^\op + p_2(x)u(x)\right)dx.
\end{equation}
Integrating by parts leads us to an equivalent representation
\begin{equation}\label{mr13e5}
(u, Lu) = \left[u(p_1 - p_0^\op)u\right]_a^b + \int_a^b\left(\frac{d^2}{dx^2}(p_0 u) - \frac{d}{dx}(p_1 u) + p_2 u\right) u dx.
\end{equation}
The integrals in the two preceeding equations are identical if their integrands are the same. That is, when,
\[
\left(\frac{d^2}{dx^2}(p_0 u) - \frac{d}{dx}(p_1 u) + p_2 u\right)u = u(p_0 u^\tp + p_1 u^\op + p_2 u).
\]
This equation is satisfied when $p_0^\op = p_1$, in which case the boundary terms in \eqref{mr13e5} vanishes. We now define the adjoint
operator
\begin{equation}\label{mr13e6}
L^\ast = p_0\frac{d^2}{dx^2} + (2p_0^\op - p_1)\frac{d}{dx} + (p_0^\tp - p_1^\op + p_2).
\end{equation}
It can be equivalently defined as
\begin{equation}\label{mr13e7}
(L^\ast u, u) = (u, Lu).
\end{equation}

An operator $L$ is said to be self adjoint if $L = L^\ast$. An equation of the form $Lu = 0$ is said to be in self-adjoint form if $L^\ast = L$.

It is easy to check that the Bessel equation in the form $x^2u^\tp + xu^\op + (x^2 - n^2)u = 0$ is not self-adjoint form. However, if we rewrite it
as 
\begin{equation}\label{mr13e8}
xu^\tp + u^\op + \left(x - \frac{n^2}{x}\right)u = 0
\end{equation}
then we get it in a self-adjoint form.

We will now prove Lagrange's identity,
\begin{equation}\label{mr13e9}
vL(u) - uL^\ast(v) = \frac{d}{dx}\left[p_0(u^\op v - uv^\op) - (p_0^\op - p_1)uv\right].
\end{equation}
The proof is quite straight-forward when begun from the right hand side when we add and subtract the term $p_2uv$.

\section{Sturm-Liouville theory}\label{mr14}
Consider the differential equation of the form
\begin{equation}\label{mr14e1}
\frac{d}{dx}\left(p(x)\frac{du}{dx}\right) + \left(\lambda\rho(x) - q(x)\right)u(x) = 0,
\end{equation}
where the functions $p, q$ and $\rho$ are real-valued over an interval $[a, b]$. It is easy to confirm that the operator $L$ defined by
\begin{equation}\label{mr14e2}
L = D[pD] - q
\end{equation}
is self-adjoint. If we write the differential equation \eqref{mr14e1} as $L(u) - \lambda\rho(x)u(x)$ then $p_0 = p, p_1 = p^\op$ and $p_2 = \lambda\rho - q$.
Lagrange's identity \eqref{mr13e9} then becomes
\begin{equation}\label{mr14e3}
vL(u) - uL(v) = \frac{d}{dx}\left[p(u^\op v - uv^\op)\right].
\end{equation}
The equation \eqref{mr14e1} is frequently accompanied by boundary conditions of the form
\begin{eqnarray}
\alpha u(a) + \alpha^\op u^\op(a) &=& 0 \label{mr14e4} \\
\beta u(b) + \beta^\op u^\op(b) &=& 0, \label{mr14e5} 
\end{eqnarray}
where none of the constants $\alpha, \alpha^\op, \beta$ and $\beta^\op$ are zero. It is then easy to check that if $u$ and $v$ are
two solutions of \eqref{mr14e1} then
\begin{eqnarray}
u(a)v^\op(a) - u^\op(a)v(a) &=& 0 \label{mr14e6} \\
u(b)v^\op(b) - u^\op(b)v(b) &=& 0 \label{mr14e7}.
\end{eqnarray}
If $u$ and $v$ are such that
\begin{eqnarray*}
L(u) - \lambda\rho(x)u(x) &=& 0 \\
L(v) - \mu\rho(x)v(x) &=& 0 
\end{eqnarray*}
then
\[
vL(u) - uL(v) = (\lambda - \mu)\rho(x)u(x)v(x)
\]
and hence,
\[
(\lambda - \mu)\int_a^b u(x)v(x)\rho(x)dx = \int_a^b \left(vL(u) - uL(v)\right)dx
\]
Using equation \eqref{mr14e3}, we readily get
\[
(\lambda - \mu)\int_a^b u(x)v(x)\rho(x)dx = \left[p(u^\op v - uv^\op)\right]_a^b.
\]
From equations \eqref{mr14e6} and \eqref{mr14e7} we readily get
\[
(\lambda - \mu)\int_a^b u(x)v(x)\rho(x)dx = 0.
\]
Since $\lambda \ne \mu$, we get the Sturm-Liouville orthogonality condition,
\begin{equation}\label{mr14e8}
\int_a^b u(x)v(x)\rho(x)dx = 0.
\end{equation}

\section{Bessel functions}\label{mr15}
Bessel equation is 
\begin{equation}\label{mr15e1}
x^2y^\tp + xy^\op + (x^2 - \nu^2)y = 0,
\end{equation}
where $y$ is a function of $x$ and $\nu$ is a complex number. We will try to solve this equation by Frobenius method. Let
\begin{equation}\label{mr15e2}
y(x) = \sum_{n=0}^\infty a_nx^{n+k}
\end{equation}
where $a_0 \ne 0$. Substituting it in equation \eqref{mr15e1} we get
\[
\sum_{n=0}^\infty a_n\left[(n+k)(n+k-1) + (n+k) - \nu^2\right]x^{n+k} + \sum_{n=0}^\infty a_nx^{n+k+2} = 0.
\]
If we put $n=0$, we get the lowest order term
\[
a_0[k(k-1) + k + \nu^2]x^k.
\]
Note that there is no contribution from the last term because its power is $x^{k+2}$. If the equation has to be satisfied and $a_0 \ne 0$ then we must have
\begin{equation}\label{mr15e3}
k^2 - \nu^2 = 0.
\end{equation}
This is the indicial equation of Bessel equation and its solution is $k = \pm\nu$. The recurrence relation is found by investigating the coefficients of a power of $x$. A general term of the series is
\[
\left(a_{n+2}\left[(n+k+2)(n+k+1) + (n+k+2) - \nu^2\right] + a_n\right)x^{n+k+2}.
\]
Since every term of the series has to be zero, we require that 
\[
a_{n+2}\left[(n+k+2)(n+k+1) + (n+k+2) - \nu^2\right] + a_n = 0
\]
or, after using the indicial equation \eqref{mr15e3}, we get the recurrence relation
\begin{equation}\label{mr15e4}
a_{n+2} = \frac{-a_n}{(n+2)(n+2k+2)}.
\end{equation}
We examine the first few coefficients,
\begin{eqnarray*}
a_2 = \frac{-a_0}{2(2k+2)} &=& \frac{-a_0 k!}{2^2 1! (k+1)!} \\
a_4 = \frac{-a_2}{4(2k+4)} &=& \frac{+a_0 k!}{2^4 2! (k+2)!} \\
a_6 = \frac{-a_4}{6(2k+6)} &=& \frac{-a_0 k!}{2^6 3! (k+3)!} 
\end{eqnarray*}
so that the general term can be written as
\begin{equation}\label{mr15e5}
a_{2n} = (-1)^n\frac{a_0 n!}{2^{2n}n!(n+k)!}.
\end{equation}
The solution of the equation is thus,
\begin{eqnarray*}
y(x) &=& a_0\sum_{n=0}^\infty (-1)^n\frac{n!x^{2n+k}}{2^{2n}n!(n+k)!} \\
     &=& a_02^k k!\sum_{n=0}^\infty (-1)^n \frac{1}{n!(n+k)!}\left(\frac{x}{2}\right)^{2n+k} 
\end{eqnarray*}
Using one solution of the indicial equation, namely $k = \nu$ and denoting the solution by its conventional name, we get
\begin{equation}\label{mr15e6}
J_\nu(x) = \sum_{n=0}^\infty (-1)^n\frac{1}{n!(n+\nu)!}\left(\frac{x}{2}\right)^{2n+\nu},
\end{equation}
where we have chosen $a_0 = (2^\nu \nu!)^{-1}$. $J_\nu$ is called the Bessel function of the first kind of order $\nu$. Recall that $\nu$ is an arbitrary complex number. 
Therefore, it is appropriate to wtite $(n + \nu)!$ as $\Gamma$-function. Thus,
\begin{equation}\label{mr15e7}
J_\nu(z) = \sum_{n=0}^\infty (-1)^n\frac{1}{n!\Gamma(n+\nu+1)}\left(\frac{z}{2}\right)^{2n+\nu}.
\end{equation}
Before closing this section we justify the orthogonality relation for Bessel functions. Recall that Bessel equation in Sturm-Lioville form is
\[
xy^\op + y^\op + \left(x - \frac{\nu^2}{x}\right)y  = 0.
\]
Comparing it with the standard form \eqref{mr14e1} we observe that $p(x) = x, \lambda = 1, \rho(x) = x$ and $q(x) = -\nu^2/x$. Therefore, equation \eqref{mr14e8} becomes
\begin{equation}\label{mr15e8}
\int_0^1 J_n(x)J_m(x)xdx = C\delta_{mn},
\end{equation}
where $C$ is a normalization constant.
