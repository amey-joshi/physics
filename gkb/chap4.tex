\chapter{Flow of a uniform incompressible viscous fluid}\label{c4}
\section{Introduction}\label{c4s1}
\begin{itemize}
\item A fluid is considered to be incompressible if
\begin{equation}\label{c4s1e1}
\md{\rho} = 0,
\end{equation}
in which case, the equation of mass conservation \eqref{c3s6e1} reduces to
\begin{equation}\label{c4s1e2}
\Delta = \dive\vec{u} = 0
\end{equation}
Equation \eqref{c4s1e2} when used in the momentum balance equation \eqref{c3s6e2} gives
\[
\rho\md{u_i} = \rho F_i - \pdt{p}{x_i} + 2\mu\frac{\partial e_{ij}}{\partial x_j} = 
\rho F_i - \pdt{p}{x_i} + \mu\frac{\partial u_i}{\partial x_j \partial x_j} + \frac{\partial}{\partial x_i}\pdt{u_j}{x_j}
\]
The last term on the right hand side is zero so that
\begin{equation}\label{c4s1e3} 
\rho\md{\vec{u}} = \rho\vec{F} - \grad p + \mu\nabla^2\vec{u}
\end{equation}
is the equation of motion. Further, since $\rho$ is a constant, there are only four unknowns, $\vec{u}$ and $p$. If $\rho$ and $T$ are fixed, then by the equation of state, the thermodyamic
pressure is fixed. For an incompressible fluid, the thermodynamic and mechanical pressures are identical so that $p$ in \eqref{c4s1e3} is a fixed quantity. Therefore, equation \eqref{c4s1e3} 
suffices for finding the unknowns.

\item We make an assumption, stronger than incompressibility, that the density of the fluid is uniform. That is, $\rho$ is independent of $\vec{x}$. Such an assumption is valid if there
is no motion driven by buoyancy forces. Thus, we do not consider \enquote*{free convection} in this chapter.

\item Writing \eqref{c4s1e3} so that the left hand side is accelaration, 
\[
\md{\vec{u}} = \vec{F} - \frac{1}{\rho}\grad p + \nu\nabla^2\vec{u},
\]
where $\nu = \mu/\rho$ is the kinematic viscosity of the fluid. It is the \enquote*{diffusivity} of fluid velocity and like all such quantities has dimensions $[L^2 T^{-1}]$.

\item It is difficult to tell when the effects of viscosity can be ignored or when they matter, despite the small value of $\mu$ for common fluids. Therefore, we take the conservative stand
and study its effect.

\item If $\rho$ is uniform and if we write $p = p_0 + \rho\vec{g}\cdot\vec{x} + P$, where $p_0$ is a constant and $P$ is a function of $\vec{x}$ and $t$, then
\[
\grad p = 0 + \rho\vec{g} + \grad{P}
\]
or $\rho\vec{g} - \grad{p} = -\grad{P}$ and hence the equation of motion \eqref{c4s1e3} becomes 
\[
\rho\md{\vec{u}} = -\grad P + \mu\nabla^2\vec{u}
\]
The effect of gravity is then folded in the new variable $P$. Thus, we can use the quantity $P$ instead of mechanical pressure $p$ if it does not show up in the boundary conditions. For $P$
is a mathematical artefact while boundary conditions are always in terms of a physical quantity. If $\rho$ is constant then $P = p - \rho\vec{g}\cdot\vec{x}$ is indeed a known quantity.
\end{itemize}
\subsection{Exercises}
\begin{enumerate}
\item Consider a fluid moving through a cylinder and a cylindrical polar coordinate system with its $z$ axis 
coinciding with the axis of the cylinder. Then the normal component of the deviatoric stress is
\[
\mu e_{\sigma\sigma} = \mu\pdt{u_\sigma}{\sigma}.
\]
Since the fluid does not penetrate the cylinder, $u_\sigma = 0$ and the result follows immediately.

\item The expression for the stress tensor is
\[
\sigma_{ij} = -p\delta_{ij} + 2\mu e_{ij},
\]
where the rate of strain tensor 
\[
e_{ij} = \frac{1}{2}\left(\pdt{u_i}{x_j} + \pdt{u_j}{x_i}\right).
\]
We know that
\[
\pdt{u_i}{x_j} = e_{ij} + \xi_{ij}
\]
so that
\[
\sigma_{ij} = -p\delta_{ij} + \mu(2\pdt{u_i}{x_j} - 2\xi_{ij}).
\]
Using equation (2.3.9) from the book, 
\[
\sigma_{ij} = -p\delta_{ij} + \mu\left(2\pdt{u_i}{x_j} + \epsilon_{ijk}\omega_k\right)
\]
so that
\[
\sigma_{ij}n_j = -p\delta_{ij}n_j + \mu\left(2n_j\pdt{u_i}{x_j} + \epsilon_{ijk}n_j\omega_k\right).
\]
The left hand side is the force per unit area exerted across a surface element with normal $\un$. The right
hand side, written in Gibbs' notation is
\[
-p\un + \mu(2\un\cdot\grad\vec{u} + \un \times \vec{\omega}).
\]
Now consider the stress at a rigid boundary. If $n_j$ is a normal to the right surface then there is no velocity
along that direction. That is $u_j = 0$. Therefore, in this situation $e_{ij} = \xi_{ij}$ and hence
\[
\sigma_{ij} = -p\delta_{ij} + 2\mu \xi_{ij}.
\]
Once again using equation (2.3.9) from the book, we get
\[
\sigma_{ij} = -p\delta_{ij} - \mu \epsilon_{ijk}\omega_k,
\]
so that the force per unit area on the surface normal to $n_j$ is
\[
\sigma_{ij}n_j = -p\delta_{ij}n_j - \mu \epsilon_{ijk}n_j\omega_k,
\]
In Gibbs' notation, the right hand side is
\[
-p\un - \mu\un\times\vec{\omega}.
\]
\end{enumerate}

\section{Steady unidirectional flow}\label{c4s2}
\begin{itemize}
\item The equation of a steady flow along the $x$ axis is
\begin{equation}\label{c4s2e1}
\frac{\partial^2 u_x}{\partial y^2} + \frac{\partial^2 u_x}{\partial z^2} = -\frac{G}{\rho},
\end{equation}
where $G$ is the pressure gradient. $u_y$ and $u_z$ are both zero and $u_x = u_x(y, z)$.

\item In the case of flow through a long pipe of a circular cross section, we transform equation \eqref{c4s2e1}
in cylindrical polar coordinates so that it becomes
\begin{equation}\label{c4s2e2}
\frac{1}{r}\frac{\partial}{\partial r}\left(r\pdt{u}{r}\right) = -\frac{G}{\rho}.
\end{equation}
Its solution subject to the boundary conditions $u(r) = 0$ and $u$ is finite is
\begin{equation}\label{c4s2e3}
u(r) = \frac{G}{4\mu}(a^2 - r^2).
\end{equation}
The volume flux is calculated as follows. Consider two cylinders coaxial with the pipe having radii $r$ and $r + 
dr$. If their ``height" is $dx$ then the volume between them is $dV = 2\pi (r + dr) dx - 2\pi rdx = 2\pi rdrdx$.
Since $dx = udt$, $dV = 2\pi r dr u dt$ or
\[
\td{V}{t} = 2\pi r u dr.
\]
This is the rate of flow of volume between the two cylinders. Therefore, the total flux is
\begin{equation}\label{c4s2e4}
Q = \int_0^a \td{V}{t} = \frac{\pi G a^4}{8\mu}.
\end{equation}
The only non-zero components of the rate of strain tensor are
\[
e_{rx} = e_{xr} = \frac{1}{2}\pdt{u}{r}
\]
so that the tangential stress at the wall is
\[
\mu\left[e_{rx} + e_{xr}\right]_{r=a} = -\frac{1}{2}Ga
\]
The rate of dissipation is
\[
\Phi = \frac{2\mu}{\rho}e_{ij}e_{ij} = \frac{G^2\rho^2}{4\rho\mu}.
\]

\item We will now solve the Navier-Stokes equation for a steady, laminar flow through an elliptical pipe. But
before we do that, we note that curves of equal velocity in a pipe of circular cross section are circles. For
equation \eqref{c4s2e3} can be written in the form $r^2 = $constant for a fixed value of $u$. We expect that the
flow through an elliptical pipe too will have the same property.

Let the equation of the cross section be
\[
\frac{y^2}{b^2} + \frac{z^2}{c^2} = 1 
\]
or $c^2y^2 + b^2z^2 = b^2c^2$ and let the velocity profile be of the form $u(y, z) =\alpha y^2 + \beta z^2 + 
\gamma$. Substituting it in
\eqref{c4s2e1}, we get
\begin{equation}\label{c4s2e5}
\alpha + \beta = -\frac{G}{2\mu}.
\end{equation}
The velocity profile will coincide with the ellipse of $\alpha = kc^2, \beta = kb^2, \gamma = -kb^2c^2$ for some
constant $k$. Substituting these in equation \eqref{c4s2e5} we get
\begin{equation}\label{c4s2e6}
k = -\frac{G}{2(b^2 + c^2)\mu}
\end{equation}
Now that we have got $k$ in terms of the physical parameters of the problem, we immediately have
\begin{eqnarray*}
\alpha &=& -\frac{Gc^2}{2(b^2 + c^2)\mu} \\
\beta  &=& -\frac{Gb^2}{2(b^2 + c^2)\mu} \\
\gamma &=& -\frac{Gb^2c^2}{2(b^2 + c^2)\mu}
\end{eqnarray*}
so that the velocity profile is 
\begin{equation}\label{c4s2e7}
u(y, z) = \frac{G}{2\mu(b^{-2} + c^{-2})}\left(\frac{y^2}{b^2} + \frac{z^2}{c^2} - 1\right)
\end{equation}

\item Navier-Stokes equation for a steady, laminar flow through a rectangular pipe is
\begin{equation}\label{c4s2e8}
\nabla^2 u = -\frac{G}{\mu}.
\end{equation}
Supposing $u^\prime$ is a solution of the Laplace's equation then it is easy to check
that
\[
u^\prime - \frac{G}{2\mu}(b^2 - y^2)
\]
is a solution of \eqref{c4s2e8}. The goal then is to find $u^\prime$. $u(\pm b, z) = 0$ and 
$u(y, \pm c) = 0$ implies that $u^\prime(\pm b, z) = 0$ and $u^\prime(y, \pm c) = G(b^2 - y^2)/(2\mu)$.
Let us assume that $u^\prime(y, z) = Y(y)Z(z)$, so that
\begin{eqnarray*}
Y^\tp &=& -k^2 Y \\
Z^\tp &=& k^2 Z
\end{eqnarray*}
where $k^2$ is a constant. The choice of sign of $k^2$ is determined by the need to satisfy the
boundary conditions on $u^\prime$. The other choice of sign will make $Y$ a hyperbolic function,
for which we cannot satisfy the boundary conditions. The solutions of the above ordinary differential
equations are
\begin{eqnarray*}
Y &=& A\cos(ky) + B\sin(ky) \\
Z &=& C\cosh(kz) + D\sinh(kz)
\end{eqnarray*}
$Y(\pm b) = 0$ gives $k = (2n+1)\pi/(2b)$ and $B = 0$. $Z$ being an even function of $z$ requires that
$D = 0$. Thus,
\[
u^\prime = A_n\cos\left(\frac{(2n+1)\pi y}{2b}\right)\cosh\left(\frac{(2n+1)\pi z}{2b}\right)
\]
or that the general solution of the Laplace's equation is
\begin{equation}\label{c4s2e9}
u^\prime = \sum_{n \ge 0}A_n\cos\left(\frac{(2n+1)\pi y}{2b}\right)\cosh\left(\frac{(2n+1)\pi z}{2b}\right).
\end{equation}
Finally, the solution of \eqref{c4s2e8} is
\begin{equation}\label{c4s2e10}
u(y,z)=\sum_{n \ge 0}A_n\cos\left(\frac{(2n+1)\pi y}{2b}\right)\cosh\left(\frac{(2n+1)\pi z}{2b}\right) - \frac{G}{2\mu}(b^2 - y^2).
\end{equation}
Equivalently,
\begin{equation}\label{c4s2e11}
u(y,z)=\sum_{n \text{ odd}}A_n\cos\left(\frac{n\pi y}{2b}\right)\cosh\left(\frac{n\pi z}{2b}\right) - \frac{G}{2\mu}(b^2 - y^2).
\end{equation}

\item The remark above equation (4.3.13) is a bit confusing. When the flow is steady there is no
net force on it and the acceleration is zero. The force of gravity balances the viscous drag.

\item The governing equation for paint brush flow is
\begin{equation}\label{c4s2e12}
\nabla^2 u = 0,
\end{equation}
where $u$ is a function of $y$ and $z$ coordinates. The boundary conditions are
\begin{eqnarray}
u(0, z) &=& 0 \label{c4s2e13} \\
u(b, z) &=& 0 \label{c4s2e14} \\
u(y, 0) &=& U,\; 0 < y < b \label{c4s2e15} \\
\lim_{z \rightarrow \infty}u(y, z) &=& 0 \label{c4s2e16}
\end{eqnarray}
Once again, we use the method of separation of variables to write $u(y, z) = Y(y)Z(z)$ and hence
\begin{eqnarray*}
Y^\tp &=& -k^2 Y \\
Z^\tp &=& k^2 Z
\end{eqnarray*}
The choice of sign of the constant of separation is dictated by the boundary conditions. If
\[
Y(y) = A\cos(ky) + B\sin(ky)
\]
then $Y(0) = 0$ and $Y(b) = 0$ require us to have $A = 0$ and $kb = n\pi$ or
\begin{equation}\label{c4s2e17}
Y(y) = B\sin\left(\frac{n\pi y}{b}\right).
\end{equation}
The solution of the $z$-equation is
\[
Z(z) = Ce^{kz} + De^{-kz}.
\]
Equation \eqref{c4s2e16} requires $C = 0$. The solution $u$ can thus be written as
\begin{equation}\label{c4s2e18}
u(y, z) = \sum_{n=1}^\infty A_n \exp\left(-\frac{n\pi z}{b}\right)\sin\left(\frac{n\pi y}{b}\right).
\end{equation}
We still have not used the boundary condition \eqref{c4s2e15}. Putting it in the
previous equation gives
\begin{equation}\label{c4s2e19}
U = \sum_{n=1}^\infty A_n \sin\left(\frac{n\pi y}{b}\right).
\end{equation}
We now use the relation
\begin{equation}\label{c4s2e20}
\int_0^\pi\sin(mx)\sin(nx)dx = \frac{\pi}{2}\delta_{mn}
\end{equation}
or,
\begin{equation}\label{c4s2e21}
\int_0^b \sin\left(\frac{m\pi y}{b}\right)\sin\left(\frac{n\pi y}{b}\right)dy = \frac{b}{2}\delta_{mn}.
\end{equation}
Using it in equation \eqref{c4s2e22} we get
\begin{equation}\label{c4s2e22}
A_n = \begin{cases}
0 & \text{ if } n \text{ is even.} \\
\frac{4U}{n\pi} & \text{ if } n \text{ is odd.}
\end{cases}
\end{equation}
Thus, the solution of the boundary value problem is
\begin{equation}\label{c4s2e23}
u(y, z) = \frac{4U}{\pi}\sum_{n\text{ odd}} \frac{1}{n}\exp\left(-\frac{n\pi z}{b}\right)\sin\left(\frac{n\pi y}{b}\right).
\end{equation}

\item Before leaving this section, we revisit the opening remarks in the book. If the flow is uni-directional and
if the flow velocity is independent of the coordinate in the flow direction then the convective term $\vec{u}\cdot
\grad{\vec{u}}$ is zero. The equation of motion becomes
\[
\pdt{\vec{u}}{t} = -\grad{p} + \mu\nabla^2\vec{u}.
\]
Without loss of generality assume that the flow is in the $x$ direction. Since $\vec{u}$ is independent of $x$, it is
evident that so is $\grad{p}$, or $p$. The flow wise motion is thus independent of the span wise motion. The motion in
the $yz$-plane is independent of the flow. 

\item Let us now confine ourselves to the motion of a fluid in a long cylindrical pipe whose generators are parallel
to the $x$-axis. There is no pressure gradient in the $y$ or $z$ axes and hence
\begin{eqnarray*}
\pdt{v}{t} &=& -\mu\nabla^2 v \\
\pdt{w}{t} &=& -\mu\nabla^2 w 
\end{eqnarray*}
Such a motion cannot be sustained against the viscous dissipation unless sustained by continual energy supply from
the tangential stresses at the boundary. The tangential stress is $\tau_{ij} = \mu e_{ij}$ so that the force per unit
area is $f_j = n_i\tau_{ij} = \mu n_i e_{ij}$ and the power supplied by it is $f_j u_j = \mu n_i e_{ij} u_j$. In the
span-wise plane, $n_1 = 0, u_1 = 0$ and this expression becomes
\[
\mu\left\{n_2 (e_{22}u_2 + e_{23}u_3) + n_3(e_{32}u_2 + e_{33}u_3)\right\}.
\]
Integrating this expression along a curve in the span-wise plane gives
\[
\mu\oint\left\{n_2 (e_{22}u_2 + e_{23}u_3) + n_3(e_{32}u_2 + e_{33}u_3)\right\}dl
\]
This expression is the power supplied by the boundary via tangential stresses. In the motion of a fluid through a
pipe, there are no tangential stresses and therefore there is no motion in the span-wise plane.
\end{itemize}

\section{Unsteady unidirectional flow}\label{c4s3}
\begin{itemize}
\item In this section we consider motion triggered by an unsteady movement of a boundary keeping the pressures
far upstream and far downstream equal. We can control only the pressures far upstream and far downstream in an
experiment. When we describe the problem statement we restrict ourselves in keeping them equal and we make no
statement about the pressure midway.

\item Navier-Stokes equation when $\grad{p} = 0$ and when the motion is in a single direction, say $x$, is
\begin{equation}\label{c4s3e1}
\pdt{u}{t} = \nu\left(\frac{\partial^2 u}{\partial y^2} + \frac{\partial^2 u}{\partial z^2}\right),
\end{equation}
where we wrote $\vec{u} = (u, v, w)$. We solve this equation using Fourier transforms. We define them as
\begin{eqnarray}
\hat{u}(k_y, k_z, t) &=&  \iint_{\mathbb{R}^2}u(y, z, t)e^{-i\vec{k}\cdot\vec{x}}dydz \\
u(y, z, t) &=& \frac{1}{4\pi^2}\iint_{\mathbb{R}^2} \hat{u}(k_y, k_z, t)e^{i\vec{k}\cdot\vec{x}}dk_ydk_z,
\end{eqnarray}
where $\vec{k} = k_y\uvec{y} + k_z\uvec{z}$ and $\vec{x} = y\uvec{y} + z\uvec{z}$. If we take the Fourier 
transform of \eqref{c4s3e1} we get
\[
\pdt{\hat{u}}{t} = -\nu k^2 \hat{u}.
\]
Its solution is 
\begin{equation}\label{c4s3e4}
\hat{u}(k_y, k_z, t) = \phi(k_y, k_z)e^{-\nu k^2 t}.
\end{equation}
From the above equation it is immediately clear that $\phi(\vec{k}) = \hat{u}(\vec{k}, 0)$, the initial value of
the Fourier transform of $u$, that is,
\[
\phi(\vec{k}) = \iint_{\mathbb{R}^2} u(y, z, 0)e^{-i\vec{k}\cdot\vec{x}}dydz.
\]
Substituting this in equation \eqref{c4s3e4} we get
\[
\hat{u}(\vec{k}, t) = \iint_{\mathbb{R}^2} u(y, z, 0)e^{-i\vec{k}\cdot\vec{x} - \nu k^2t}dydz.
\]
Therefore,
\[
u(\vec{x},t)=\frac{1}{4\pi^2}\iint_{\mathbb{R}^2} dk_ydk_z e^{i\vec{k}\cdot\vec{x}}\iint_{\mathbb{R}^2} dy^\op dz^\op u(y, z, 0) e^{-i\vec{k}\cdot\vec{x}^\op - k^2\nu t}
\]
Interchanging the order of integration,
\begin{equation}\label{c4s3e5}
u(\vec{x},t)=\frac{1}{4\pi^2}\iint_{\mathbb{R}^2} dy^\op dz^\op \iint_{\mathbb{R}^2} dk_ydk_z u(y, z, 0) e^{i\vec{k}\cdot(\vec{x}-\vec{x}^\op) - k^2\nu t}.
\end{equation}
Let us write the $\vec{k}-$integral as
\begin{equation}\label{c4s3e6}
K(\vec{x}, \vec{x}^\op) = \iint_{\mathbb{R}^2} dk_ydk_z u(y, z, 0) e^{i\vec{k}\cdot(\vec{x}-\vec{x}^\op) - k^2\nu t}.
\end{equation}
The integral readily factorizes as
\begin{equation}\label{c4s3e7}
K(\vec{x}, \vec{x}^\op) = u(y, z, 0) \int_{\mathbb{R}} dk_y e^{ik_y(y-y^\op)-k_y^2\nu t}\int_{\mathbb{R}} dk_z e^{ik_z(y-y^\op)-k_z^2\nu t}
\end{equation}
The two integrals are identical and we can as well focus only on one of them
\[
I(y, y^\op) = \int_{\mathbb{R}} dk_y e^{ik_y(y-y^\op)-k_y^2\nu t}
\]
After `completing the square' of the arguments of the exponential,
\[
I(y, y^\op) = \exp\left(-\frac{(y - y^\op)^2}{4\nu t}\right)\int_{\mathbb{R}}\exp\left(-\left(k_y\sqrt{\nu t} + i\frac{y^\op - y}{2\sqrt{\nu t}}\right)\right)dk_y
\]
or,
\[
I(y, y^\op) = \exp\left(-\frac{(y - y^\op)^2}{4\nu t}\right)\frac{1}{\sqrt{\nu t}}\int_{\mathbb{R}}\exp\left(-\left(k_y\sqrt{\nu t} + i\frac{y^\op - y}{2\sqrt{\nu t}}\right)\right)d(k_y\sqrt{\nu t}).
\]
The integral is of the form
\[
\int_{\mathbb{R}} e^{-(x + ia)^2}dx,
\]
where $a$ is a constant. It can be shown that
\begin{equation}\label{c4s3e8}
\int_{\mathbb{R}} e^{-(x + ia)^2}dx = \int_{\mathbb{R}}e^{-x^2}dx = \sqrt{\pi}
\end{equation}
so that
\begin{equation}\label{c4s3e9}
I(y, y^\op) = \exp\left(-\frac{(y - y^\op)^2}{4\nu t}\right)\sqrt{\frac{\pi}{\nu t}}
\end{equation}
and hence
\begin{equation}\label{c4s3e10}
K(\vec{x}, \vec{x}^\op) = \frac{\pi}{\nu t}u(y, z, 0) \exp\left(-\frac{(y - y^\op)^2 + (z - z^\op)^2}{4\nu t}\right).
\end{equation}
From equations \eqref{c4s3e5}, \eqref{c4s3e6} and \eqref{c4s3e10} we get
\begin{equation}\label{c4s3e11}
u(\vec{x}, t) = \frac{1}{4\pi\nu t}\iint_{\mathbb{R}^2}dy^\op dz^\op u(y^\op, z^\op, 0)\exp\left(-\frac{(y - y^\op)^2 + (z - z^\op)^2}{4\nu t}\right).
\end{equation}

\item Equation \eqref{c4s3e11} is a solution of the partial differential equation \eqref{c4s3e1} whose exact form
depends on the initial value $u(y, z, 0)$. We will now apply it to various cases of the initial values.

\item Supposing there are two streams of fluid moving with different velocities and suppose that there is
initially a sharp boundary between them. If we choose our reference frame as the one moving with a speed
equal to the mean of the two streams then with respect to it, one field moves with velocity, say, $U$ and 
the other one with velocity $-U$. Further, let the initial boundary between the streams coincide with
the $xz$ plane so that
\begin{eqnarray*}
u(y^\op, z^\op, 0) &=& U \text{ if } y^\op > 0 \\
u(y^\op, z^\op, 0) &=& -U \text{ if } y^\op < 0.
\end{eqnarray*}
Putting these values in equation \eqref{c4s3e11} we get
\[
u(\vec{x},t) = \frac{1}{4\pi\nu t}\int_{\mathbb{R}}e^{-(z-z^\op)^2/(4\nu t)}dz^\op\left[\int_0^\infty U e^{-(y-y^\op)^2/(4\nu t)}dy^\op - \int_{-\infty}^0 U e^{-(y-y^\op)^2/(4\nu t)}dy^\op\right].
\]
The $z^\op$ integral can be readily evaluated so that
\[
u(\vec{x},t) = \frac{1}{\sqrt{4\pi\nu t}}\left[\int_0^\infty U e^{-(y-y^\op)^2/(4\nu t)}dy^\op - \int_{-\infty}^0 U e^{-(y-y^\op)^2/(4\nu t)}dy^\op\right].
\]
In the second integral, make the substitution $y^\op \mapsto -y^\op$ so that
\begin{equation}\label{c4s3e12}
u(\vec{x},t) = \frac{U}{\sqrt{4\pi\nu t}}\left[\int_0^\infty e^{-(y-y^\op)^2/(4\nu t)}dy^\op - \int_0^\infty e^{-(y+y^\op)^2/(4\nu t)}dy^\op\right].
\end{equation}
In the first integral, put $\xi = y - y^\op$ so that it becomes
\[
-\int_y^{-\infty}\exp\left(-\frac{\xi^2}{4\nu t}\right)d\xi = \int_{-y}^\infty\exp\left(-\frac{\xi^2}{4\nu t}\right)d\xi.
\]
Similarly, in the second integral put $\xi = y + y^\op$ so that it becomes
\[
\int_y^\infty\exp\left(-\frac{\xi^2}{4\nu t}\right)d\xi.
\]
Equation \eqref{c4s3e12} thus becomes,
\[
u(\vec{x}, t) = \frac{U}{\sqrt{4\pi\nu t}}\int_{-y}^y\exp\left(-\frac{\xi^2}{4\nu t}\right)d\xi = \frac{U}{\sqrt{\pi\nu t}}\int_0^y\exp\left(-\frac{\xi^2}{4\nu t}\right)d\xi.
\]
Finally, make the substitution $\chi = \xi/\sqrt{4\nu t}$ to
\[
u(\vec{x}, t) = \frac{2U}{\sqrt{\pi}}\int_0^{y/\sqrt{4\nu t}} e^{-\chi^2}d\chi.
\]
We now use the definition of the error function,
\begin{equation}\label{c4s3e13}
\text{erf}(x) = \frac{2}{\sqrt{\pi}}\int_0^x e^{-t^2}dt.
\end{equation}
Therefore, 
\begin{equation}\label{c4s3e14}
u(\vec{x}, t) = U\text{erf}\left(\frac{y}{\sqrt{4\nu t}}\right).
\end{equation}
The velocity profile is shown in figure \ref{c4f1}
\begin{figure}
\centering
\includegraphics[scale=0.6]{c4f1}
\caption{Transition layer between two parallel streams}\label{c4f1}
\end{figure}
This figure was produced using the following code.
\begin{verbatim}
x <- seq(from = 0, to = 3, len = 100)
y <- pnorm(x) - pnorm(0)
X <- c(-rev(x), x)
# Recall that we are in the frame of reference such that upper 
# fluid moves with velocity U and the lower one with velocity -U.
Y <- c(-rev(y), y)
plot(
  Y,
  X,
  type = 'l',
  main = 'Transition layer',
  xlab = 'u/U',
  ylab = expression(y / sqrt(4 * nu * t))
)
abline(h = 0, lty = 2)
abline(v = 0, lty = 2)
\end{verbatim}

The distance between the points where $u/U = \pm 0.99$ is computed using the following code.
\begin{verbatim}
d1 <- qnorm(0.99)
d2 <- qnorm(0.1)
delta_u_by_U <- d1 - d2
\end{verbatim}
It is $4.652696$ so that 
\[
\delta y = \sqrt{\nu t} \times 2 \times 4.652696 = 9.3053.
\]
The coefficient will be $8$ if we use $u/U = \pm 0.9772499$.

\item We next show that the asymptotic behaviour of the solution will remain unchanged if instead of the
sharp discontinuity that we assumed previously we have 
\begin{equation}\label{c4s3e15}
u(y) = \frac{y}{|y|}U + F(y),
\end{equation}
where the function $F$ goes to zero rapidly as $y \rightarrow \pm\infty$ and 
\begin{equation}\label{c4s3e16}
\int_\mathbb{R} F(y)dy = 0.
\end{equation}
Note that the first term in equation \eqref{c4s3e15} is just $\pm U$. When the $u$ of \eqref{c4s3e15} 
is used in equation \eqref{c4s3e11}, the first term will give the right hand side of equation
\eqref{c4s3e14} and the second term will give
\[
I = \frac{1}{4\pi\nu t}\iint_{\mathbb{R}^2}F(y^\op)\exp\left(-\frac{(y - y^\op)^2 + (z - z^\op)^2}{4\nu t}\right)dy^\op dz^\op.
\]
The $z^\op$ integral readily evaluates to $\sqrt{4\pi\nu t}$ so that
\begin{eqnarray*}
I &=& \frac{1}{\sqrt{4\pi\nu t}}\int_\mathbb{R}F(y^\op)\exp\left(-\frac{(y - y^\op)^2}{4\nu t}\right)dy^\op \\
&=& \frac{e^{-y^2/(4\nu t)}}{\sqrt{4\pi\nu t}}\int_\mathbb{R}F(y^\op)\exp\left[\frac{2yy^\op - {y^\op}^2}{4\nu t}\right]dy^\op
\end{eqnarray*}
Let us approximate the exponential as $e^x \approx 1 + x$ for small values of $x$ so that
\[
I = \frac{e^{-y^2/(4\nu t)}}{\sqrt{4\pi\nu t}}\left\{\int_\mathbb{R}F(y^\op)dy^\op + \frac{y}{2\nu t}\int_\mathbb{R}F(y^\op)y^\op dy^\op - \int_\mathbb{R}F(y^\op)\frac{{y^\op}^2}{4\nu t}dy^\op\right\}.
\]
From \eqref{c4s3e16} it is clear that $F$ is an odd function so that the first and the third integrals
are zero and
\[
I = \frac{1}{4}\frac{1}{\nu t}\frac{y}{\sqrt{\pi\nu t}}\exp\left(-\frac{y^2}{4\nu t}\right)\int_\mathbb{R}F(y^\op)y^\op dy^\op 
\]
For a fixed $y/\sqrt{\nu t}$, $I = O(t^{-1})$.

\item Before proceeding we show that the condition on $F$ in equation \eqref{c4s3e15} can be
relaxed. If $F(y) = \exp(iay)$ for a real constant $a$ then 
\[
I = \frac{1}{\sqrt{4\pi\nu t}}\int_{\mathbb{R}}e^{iay^\op}\exp\left(-\frac{(y - y^\op)^2}{4\nu t}\right)dy^\op.
\]
It can be simplified as
\[
I = \frac{e^{iay}}{\sqrt{4\pi\nu t}}\int_{\mathbb{R}}\exp\left(-ia(y - y^\op) - \frac{(y - y^\op)^2}{4\nu t}\right)dy^\op.
\]
Under the transformation $y^\op \mapsto y - y^\op$, the integral becomes
\begin{eqnarray*}
I &=& \frac{e^{iay}}{\sqrt{4\pi\nu t}}\int_{\mathbb{R}}\exp\left(-iay^\op - \frac{{y^\op}^2}{4\nu t}\right)dy^\op \\
 &=& \frac{e^{iay - a^2\nu t}}{\sqrt{4\pi\nu t}}\int_{\mathbb{R}}\exp\left(-\left(\frac{y^\op}{\sqrt{4\pi\nu t}} + ia\sqrt{\nu t}\right)^2\right) dy^\op.
\end{eqnarray*}
Under yet another transformation $y^\op/\sqrt{4\pi\nu t} \mapsto y^\op$, we have
\[
I = e^{iay - a^2\nu t}\int_{\mathbb{R}}e^{-({y^\op} + ia\nu t)^2}dy^\op = \sqrt{\pi}e^{iay - a^2\nu t}.
\]
As $t \rightarrow \infty$, $I \rightarrow 0$. Thus any function that can be written as a sum of normal modes $e^{iay}$ 
will decay to zero as time goes by.

This analysis demonstrates that an instability at the interface of fluids moving with differing velocities
should decay rapidly with time. It seems to contradict the standard analysis of Kelvin-Helmholtz instability.
However, one must remember that the standard analysis assumes that the fluids are ideal while we work
with real fluids.

\item In the case of a plane boundary moved suddenly in a fluid initially at rest, the boundary value
problem is $u_t = \nu u_{yy}$, subject to
\begin{eqnarray*}
u(y, 0) &=& 0, \text{ for } y > 0 \\
u(0, t) &=& U, \text{ for } t > 0.
\end{eqnarray*}
If we move the to the frame of reference of the moving plane, the initial and boundary conditions become
\begin{eqnarray*}
u(y, 0) &=& -U, \text{ for } y > 0 \\
u(0, t) &=& 0, \text{ for } t > 0.
\end{eqnarray*}
In this form the problem is identical to the upper half-space of the previous item. The solution is, following
\eqref{c4s3e14},
\begin{equation}
u(y, t) = -U\text{erf}\left(\frac{y}{4\nu t}\right)
\end{equation}\label{c4s3e17}
which in the laboratory frame of reference becomes
\begin{equation}\label{c4s3e18}
u(y, t) = U - U\text{erf}\left(\frac{y}{4\nu t}\right)
\end{equation}
We cannot use equation \eqref{c4s3e11} in this case because we do not know that $u$ is in the lower half-space.
We cannot put it zero because it is not the case that there is fluid at rest in the lower half-space.

\item We now consider the motion of a fluid between two parallel plates of which one is at rest and the other
moves with a uniform velocity $U$ in its own plane. The boundary value problem for this case is $u_t = \nu u_{yy}$
subject to
\begin{eqnarray*}
u(0, t) &=& U, t > 0 \\
u(d, t) &=& 0, t > 0 \\
u(y, 0) &=& 0, 0 \le y \le d.
\end{eqnarray*}
The boundary conditions for this problem are inhomogeneous so that we cannot use the separation of variables to
solve the problem. We introduce a new variable,
\begin{equation}\label{c4s3e19}
w(y, t) = U\left(1 - \frac{y}{d}\right) - u(y, t).
\end{equation}
It is easy to verify that $w_t = \nu w_{yy}$ and
\begin{eqnarray*}
w(0, t) &=& 0, t > 0 \\
w(d, t) &=& 0, t > 0 \\
w(y, 0) &=& U\left(1 - \frac{y}{d}\right).
\end{eqnarray*}
If $w(y, t) = Y(y)T(t)$ then
\[
\frac{1}{T}\td{T}{t} = \frac{\nu}{Y}\frac{d^2Y}{dy^2} = -k^2,
\]
where $-k^2$ is a constant of separation. We chose its sign keeping in mind that the solution cannot blow up
as time goes by. Thus,
\begin{equation}\label{c4s3e20}
T(t) = \alpha e^{-k^2 t}
\end{equation}
and
\begin{equation}\label{c4s3e21}
Y(y) = A\exp\left(\frac{ik}{\sqrt{\nu}}y\right) + B\exp\left(-\frac{ik}{\sqrt{\nu}}y\right).
\end{equation}
The solution $w$ is
\begin{equation}\label{c4s3e22}
w(y, t) = Ae^{-k^2 t}\exp\left(\frac{ik}{\sqrt{\nu}}y\right) + Be^{-k^2 t}\exp\left(-\frac{ik}{\sqrt{\nu}}y\right)
\end{equation}
where the constant $\alpha$ has been `absorbed' in $A$ and $B$. The condition $w(0, t) = 0$ gives $B = -A$ so that
\begin{equation}\label{c4s3e23}
w(y, t) = 2Aie^{-k^2 t}\sin\left(\frac{ky}{\sqrt{\nu}}\right).
\end{equation}
The condition $w(d, t) = 0$ restricts $k$ to
\begin{equation}\label{c4s3e24}
k = \frac{n\pi\sqrt{\nu}}{d}, n \in \mathbb{Z}.
\end{equation}
Thus,
\begin{equation}\label{c4s3e25}
w(y, t) = A_n\exp\left(-n^2\pi^2\frac{\nu t}{d^2}\right)\sin\left(\frac{n\pi y}{d}\right).
\end{equation}
The constant is now denoted as $A_n$ to indicate the possibility that it will depend on $n$, They are determined by
the initial condition. Thus, we have
\[
\sum_{n=1}^\infty A_n\sin\left(\frac{n\pi y}{d}\right) = U\left(1 - \frac{y}{d}\right)
\]
so that
\[
\sum_{n=1}^\infty A_n\int_0^d\sin\left(\frac{m\pi y}{d}\right)\sin\left(\frac{n\pi y}{d}\right)dy = U\int_0^d\left(1 - \frac{y}{d}\right)\sin\left(\frac{m\pi y}{d}\right)dy
\]
or
\[
\sum_{n=1}^\infty A_n \frac{d}{2}\delta_{mn} = \frac{Ud}{\pi m}
\]
or
\begin{equation}\label{c4s3e26}
A_m = \frac{2U}{\pi m}.
\end{equation}
The solution to the initial, boundary value problem is thus,
\begin{equation}\label{c4s3e27}
u(y, t) = U\left(1 - \frac{y}{d}\right) - \frac{2U}{\pi}\sum_{n=1}^\infty \frac{1}{n}\exp\left(-n^2\pi^2\frac{\nu t}{d^2}\right)\sin\left(\frac{n\pi y}{d}\right).
\end{equation}

\item We next consider the motion of a fluid due to an oscillating plate. Suppose that the upper half of the
$xy$-plane is filled with a fluid and is bounded by a plate lying in the $xy$-plane. If the plate oscillates
with a velocity $U\cos(nt)$ then we can assume that in the steady state the velocity of the fluid too will
have a harmonic component of frequency $n$. Thus, we can assume that
\[
u(y, t) = \Re(e^{int}F(y)),
\]
for a function $F$. Substituting this form in the differential equation of the fluid we get $\nu F^{\tp} =
inF$. If we write $in/\nu = \alpha^2$ then the solution to this equation is
\[
F(y) = A e^{-\alpha y} + Be^{\alpha y}.
\]
From the definition of $\alpha^2$ it follows that
\[
\alpha = (1 + i)\sqrt{\frac{n}{2\nu}}
\]
so that the requirement that $F$ stays finite as $y \rightarrow \infty$ needs $B = 0$. Note that the fluid
is confined to the upper half of the $xy$-plane so that the condition $y \rightarrow -\infty$ is irrelevant.
Thus,
\[
u(y, t) = A\exp\left(-\sqrt{\frac{n}{2\nu}}y\right)\cos\left(nt - \frac{n}{2\nu}y\right).
\]
Since $u(0, t) = U\cos(nt)$, we have $A = U$ and the solution of the problem is
\begin{equation}\label{c4s3e28}
u(y, t) = U\exp\left(-\sqrt{\frac{n}{2\nu}}y\right)\cos\left(nt - \frac{n}{2\nu}y\right).
\end{equation}

\item The differential equation for starting flow is
\begin{equation}\label{c4s3e29}
\pdt{u}{t} = \frac{G}{\rho} + \nu\left(\frac{\partial^2 u}{\partial r^2} + \frac{1}{r}\pdt{u}{r}\right),
\end{equation}
where $G$ the constant pressure gradient. The boundary and initial conditions on $u$ are
\begin{eqnarray}
u(a, t) &=& 0, \text{ for } t \ge 0 \label{c4s3e30} \\
u(r, 0) &=& 0, \text{ for } 0 \le r \le a.\label{c4s3e31}
\end{eqnarray}
In this problem the equation is inhomogeneous but the boundary condition is homogeneous. We can make the
equation homogeneous by writing $u$ as a sum of the steady state solution plus the transient solution. Let
\begin{equation}\label{c4s3e32}
u(r, t) = \frac{G}{4\nu}(a^2 - r^2) + w(r, t).
\end{equation}
It is easy to confirm that $w$ satisfies the equation
\begin{equation}\label{c4s3e33}
\pdt{w}{t} = \nu\left(\frac{\partial^2 w}{\partial r^2} + \frac{1}{r}\pdt{w}{r}\right),
\end{equation}
subject to
\begin{eqnarray}
w(a, t) &=& 0, \text{ for } t \ge 0 \label{c4s3e34} \\
w(r, 0) &=& \frac{G}{4\mu}(a^2 - r^2), \text{ for } 0 \le r \le a.\label{c4s3e35}
\end{eqnarray}
This is a homogenous partial differential equation with homogeneous boundary conditions. We can solve it
using the `separation of variables' method. If we write $w(r, t) = R(r)T(t)$ then
\[
\frac{1}{T}\td{T}{t} = \frac{\nu}{R}\left(\frac{d^2R}{dr^2} + \frac{1}{r}\td{R}{r}\right) = -k^2\nu,
\]
where $-k^2\nu$ is the constant of separation. Its sign is chosen to ensure that the solution does not
diverge as $t \rightarrow \infty$. We thus get two ordinary differential equations
\begin{eqnarray}
T^\op + k^2\nu T &=& 0 \label{c4s3e36} \\
r^2R^\tp + rR^\op + k^2r^2R &=& 0 \label{c4s3e37}
\end{eqnarray}
Introduce a new variable $x = kr$ in equation \eqref{c4s3e37} to get
\begin{equation}\label{c4s3e38}
x^2R^\tp + xR^\op + x^2R = 0.
\end{equation}
Bessel equation of order $n$ is $z^2w^\tp + zw^\op + (z^2 - n^2)w = 0$, where $w$ is a function of $z$. 
Comparing it with equation \eqref{c4s3e38} suggests that it is Bessel equation of order $0$ and its
solution is
\begin{equation}\label{c4s3e39}
R(x) = \alpha J_0(x) + \beta Y_0(x),
\end{equation}
where $J_0$ is the Bessel function of first kind order $0$ and $Y_0$ is Bessel function of second kind
of order $0$. Since $Y_0 \rightarrow \infty$ as $x = kr \rightarrow 0$ we must have $\beta = 0$ and
\begin{equation}\label{c4s3e40}
R(x) = \alpha J_0(x).
\end{equation}
Using the boundary condition $R(r = a) = 0$ or $R(x = ka) = 0$ in the above equation gives
\[
\alpha J_0(ra) = 0.
\]
If we want a non-trivial solution then $ra$ should be a zero of the function $J_0$, say $\lambda_n$.
Thus,
\[
R(kr) = \alpha J_0\left(\lambda_n\frac{r}{a}\right).
\]
Since $ka = \lambda_n$, $k = \lambda_n/a$ so that the solution to \eqref{c4s3e36} is
\[
T(t) = \alpha^\op\exp\left(-\lambda_n^2\frac{\nu t}{a^2}\right)
\]
so that
\begin{equation}\label{c4s3e41}
w(r, t) = \frac{G}{4\mu}\sum_{n=1}^\infty A_n J_0\left(\lambda_n\frac{r}{a}\right)\exp\left(-\lambda_n^2\frac{\nu t}{a^2}\right),
\end{equation}
where we pulled out $G/(4\mu)$ outside the sum for convenience. Using the initial condition in 
\eqref{c4s3e35} we get
\begin{equation}\label{c4s3e42}
a^2 - r^2 = \sum_{n=1}^\infty A_n J_0\left(\lambda_n\frac{r}{a}\right)
\end{equation}

\end{itemize}
