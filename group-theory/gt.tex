\documentclass{article}
\usepackage{amsmath, amsthm, amssymb, amsfonts}
\usepackage{mathtools, physics, xcolor}
\usepackage{hyperref}

\newcommand{\sor}{\mathbf{R}}
\DeclareMathOperator{\aut}{Aut}
\DeclareMathOperator{\card}{card}

\theoremstyle{plain}
\newtheorem{thm}{Theorem}
\numberwithin{thm}{section}

\theoremstyle{plain}
\newtheorem{prop}{Proposition}
\numberwithin{prop}{section}

\theoremstyle{definition}
\newtheorem{defn}{Definition}
\numberwithin{defn}{section}

\theoremstyle{remark}
\newtheorem*{rem}{Remark}

\newtheorem*{cor}{Corollary}

\numberwithin{equation}{section}

\title{Group Theory}
\author{Amey Joshi}
\date{24-Dec-2020}
\begin{document}
\maketitle
\section{Basic definitions}\label{s1}
\begin{defn}\label{s1d1}
A group is a set $G$ and a binary operation $\cdot$ defined on it with the
following properties:
\begin{enumerate}
\item For all $a, b \in G$, $a \cdot b \in G$;
\item For all $a, b, c \in G$, $a \cdot (b \cdot c)= (a \cdot b) \cdot c$;
\item There exists an element $e \in G$ such that for all $a \in G$, $a \cdot e
= e \cdot a = a$. It is called the identity element.
\item For all $a \in G$ there exists an element $x_1$ such that $a \cdot x_1 = 
e$ and an element $x_2$ such that $x_2 \cdot a = e$. The elements $x_1$ and 
$x_2$ are called the right and left inverse of $a$.
\end{enumerate}
\end{defn}
\begin{rem}
We frequently abbreviate $a \cdot b$ as $ab$.
\end{rem}

A few propositions follow immediately.
\begin{prop}\label{s1p1}
Identity element is unique.
\end{prop}
\begin{proof}
Let, if possible, there be two identity elements $e_1$ and $e_2$. Since $e_1$
is an identity $a e_1 = a$ for all $a \in G$. Choose $a = e_2$ so that $e_2
e_1 = e_2$. But $e_2$ is also an identity so that $e_1 = e_2$.
\end{proof}

\begin{prop}\label{s1p2}
The right and left inverse of an element are identical.
\end{prop}
\begin{proof}
Let is possible there be elements $x, y \in G$ such that $xa = e$ and $ay = e$.
Then $(xa)y = ey \Rightarrow x(ay) = y \Rightarrow xe = y \Rightarrow x = y$.
\end{proof}

\begin{rem}
We no longer have to differentiate between the right and the left inverse of 
an element and call it just the inverse.
\end{rem}

\begin{prop}\label{s1p3}
The inverse of an element is unique.
\end{prop}
\begin{proof}
Let if possible there be elements $x, y \in G$ such that $ax = e$ and $ay = e$.
Then $y(ax) = ye \Rightarrow (ya)x = y \Rightarrow ex = y \Rightarrow x = y$.
\end{proof}

\begin{rem}
The unique inverse of an element $a$ is denoted by $a^{-1}$.
\end{rem}

\begin{prop}\label{s1p4}
If $x, y \in G$ then $(xy)^{-1} = y^{-1}x^{-1}$.
\end{prop}
\begin{proof}
It is easy to check that $y^{-1}x^{-1}$ is an inverse of $xy$. From proposition
\ref{s1p3} it is the only inverse.
\end{proof}

\begin{prop}\label{s1p5}
For all $x \in G$, $(x^{-1})^{-1} = x$.
\end{prop}
\begin{proof}
Follows from the fact that $xx^{-1} = e$ and the uniqueness of inverse of 
$x^{-1}$.
\end{proof}

\begin{defn}\label{s1d2}
A group $G$ is called abelian if $xy = yx$ for all $x, y \in G$.
\end{defn}

\section{Conjugate elements}\label{s2}
\begin{defn}\label{s2d1}
Elements $a$ and $b$ of a group $G$ are said to be conjugate to each other
if there exists an element $c \in G$ such that $a = cbc^{-1}$. 
\end{defn}

If $a$ is conjugate to $b$ it is written as $a \sim b$. Clearly, conjugacy is
a binary relation on $G$. Here are a few examples of conjugate elements.

\begin{enumerate}
\item
Let $a$ denote a rotation of a rigid body by an angle $\phi$ about the $x$-axis
and $b$ be the rotation by an angle $\phi$ about the $y$ axis. Then,
\begin{eqnarray}
a &=& \begin{pmatrix}1 & 0 & 0 \\ 0 & \cos\phi & \sin\phi \\
0 & -\sin\phi & \cos\phi\end{pmatrix} \label{s2e1} \\
b &=& \begin{pmatrix} \cos\phi & 0 & \sin\phi \\ 0 & 1 & 0 \\
-\sin\phi  & 0 & \cos\phi\end{pmatrix} \label{s2e2} 
\end{eqnarray}
Choose
\begin{equation}\label{s2e3}
c = \begin{pmatrix}0 & 1 & 0 \\ -1 & 0 & 0 \\ 0 & 0 & 1\end{pmatrix}
\end{equation}
so that
\begin{equation}\label{s2e4}
c^{-1} = \begin{pmatrix}0 & -1 & 0 \\ 1 & 0 & 0 \\ 0 & 0 & 1\end{pmatrix}.
\end{equation}
We can readily confirm that $cac^{-1} = b$. The matrix $c$ is just the
rotation around the $z$-axis by $\pi/2$. Therefore the operation $cbc^{-1}$ 
means
\begin{enumerate}
\item Apply $c^{-1}$, that is, rotate around the $z$ axis by $-\pi/2$. This
brings the $y$ axis to the where the $x$ axis was prior to rotation;
\item Apply $b$, that is, rotate around the `rotated $y$ axis' by $\phi$;
\item Apply $c$, that is, rotate around the $z$ axis by $\pi/2$. This restores
the $x$ and $y$ axes to their original positions.
\end{enumerate}

\item Consider the elements of the symmetric group $S_3$ which involve 
transposing two elements. They are written in Cayley's two line notation as
\begin{eqnarray}
x &=& \begin{pmatrix}1 & 2 & 3 \\ 1 & 3 & 2 \end{pmatrix} \label{s2e5} \\
y &=& \begin{pmatrix}1 & 2 & 3 \\ 3 & 2 & 1 \end{pmatrix} \label{s2e6} \\
z &=& \begin{pmatrix}1 & 2 & 3 \\ 2 & 1 & 3 \end{pmatrix} \label{s2e7}
\end{eqnarray}
It is easy to check that the elements $x, y, z$ are their own inverses. This
is not a surprising observation because carrying out the same transposition
twice gives the original arrangement. It is easy to confirm that $zxz^{-1} = y$.
\end{enumerate}

In some sense, conjugate elements are `similar' to each other. The matrices
$a$ and $b$ in equations \eqref{s2e1} and \eqref{s2e2} are both rotations 
by the same angle but about different axes. This notion of similarity is 
captured in
\begin{prop}\label{s2p1}
Conjugacy is an equivalence relation.
\end{prop}
\begin{proof}
Choose $c = a, b = a$ in the equation $a = cbc^{-1}$ to show that $a \sim a$.
If $a = cbc^{-1}$ then $c^{-1}a = bc^{-1} \Rightarrow c^{-1}ac = b$ so that
$b \sim a$. Finally, let $a \sim b$ and $b \sim c$. Thus, there exist $x, y
\in G$ such that $a = xbx^{-1}$ and $b = ycy^{-1}$ so that $a = xycy^{-1}x^{-1}
= (xy)c(xy)^{-1}$. Thus, $a \sim c$.
\end{proof}

Therefore, the conjugacy relation partitions the groups into equivalence 
classes. The members of the equivalence classes are indeed `similar' to each
other in the sense illustrated by the two examples above.

\section{Subgroups}\label{s3}
\begin{defn}\label{s3d1}
A subset $H$ of a group $G$ is called a subgroup if $H$ is a group under
the same binary operation.
\end{defn}

\begin{rem}
A subgroup must contain the identity element.
\end{rem}

\begin{rem}
$H = G$ and $H = \{e\}$ are called trivial subgroups of $G$.
\end{rem}

\begin{prop}\label{s3p1}
$H$ is a subgroup of a group $G$ if and only if for all $h_1, h_2 \in H$,
$h_1^{-1}h_2 \in H$.
\end{prop}
\begin{proof}
The `only if' part follows immediately from the closure and inverse properties
of $H$.

Let $H \subset G$ such that for all $h_1, h_2 \in H$, $h_1^{-1}h_2 \in H$.
Choose $h_2 = h_1$ to conclude that $e \in H$. Choose $h_2 = e$ to conclude
that $h_1^{-1} \in H$. Since $h_1^{-1} \in H$, $(h_1^{-1})^{-1}h_2 = h_1h_2
\in H$. Associatity of the binary operation is valid for all elements of $G$
and therefore is valid for $H \in G$.
\end{proof}

A subgroup $H$ of a group $G$ is a set from which one cannot escape by carrying
out group operations among its elements. The only way to move out of $H$ is
by operating an $h \in H$ with a $g \in G$ such that $g \notin H$.

\begin{defn}\label{s3d2}
Given a subgroup $H$ of a group $G$, the left coset of an element $a \in G$
is the set $aH = \{ah | h \in H\}$.
\end{defn}

\begin{prop}\label{s3p2}
$\card{aH} = \card{H}$.
\end{prop}
\begin{proof}
If $ah_i = ah_j$ then $a^{-1}ah_i = a^{-1}ah_j \Rightarrow h_i = h_j$. Thus
all elements of $aH$ are distinct.
\end{proof}

\begin{prop}\label{s3p3}
$a \in aH$. If $b \in aH$ then $aH = bH$.
\end{prop}
\begin{proof}
Since $H$ is a subgroup, $e \in H$. Therefore, the set $aH$ contains $ae = a$.

Let $b \in aH$. Therefore, there exists $h \in H$ such that $b = ah$. Therefore
any element of $bH$ is of the form $bh^\prime = ahh^\prime$. Thus $bH \subset
aH$. Since $b = ah$, $a = bh^\prime$ and any element of $aH$ can be written
as $ah = bh^\prime h$, which is an element of $bH$. Thus $aH \subset bH$.
\end{proof}

Given a subgroup $H$ of $G$, the left cosets of $H$ partition the set $G$.
Therefore, the relation $a R b$ if $b \in aH$ is an equivalence relation. We
can prove this fact independently of proposition \ref{s3p2}.

\begin{prop}\label{s3p4}
Let $H$ be a subgroup of a group $G$ and $aH$ be one of its left cosets. Define
a relation $R$ on $G$ as $a R b$ if $b \in aH$. Then $R$ is an equivalence
relation.
\end{prop}
\begin{proof}
Since $H$ is a subgroup of $G$, $e \in H$ and hence $a \in aH$. We showed
in the proof of proposition \ref{s3p2} that if $b \in aH$ then $a \in bH$.
Finally, assume $a \in bH$ and $b \in cH$. Then $a = bh_1$ and $b = ch_2$ 
so that $a = c(h_1h_2)$. Thus $a \in cH$.
\end{proof}

\begin{prop}\label{s3p5}
Left coset $aH$ of a subgroup $H$ of a group $G$ is not a subgroup of $G$
unless $a = e$.
\end{prop}
\begin{proof}
If $a \ne e$ then $e \notin aH$.
\end{proof}

\begin{rem}
Note that $a \notin H$ when we talk about the coset $aH$. Therefore, $a^{-1}$
too is not a member of $H$. For if it did then $a$ too would be a member of 
the subgroup $H$.

Therefore, we are guaranteed not to have $e$ in the set $aH$.
\end{rem}

Since all left cosets of a subgroup $H$ of a group $G$ have the same cardinality
and since they partition $G$, we have
\begin{thm}\label{s3t1}
$\card H | \card G$.
\end{thm}

\begin{defn}\label{s3d3}
Given a subgroup $H$ of a group $G$, the right coset of an element $a \in G$
is the set $Ha = \{ha | h \in H\}$.
\end{defn}

Right cosets have properties similar to that of left cosets. In particular,
they obey propositions analogous to \ref{s3p2}, \ref{s3p3}, \ref{s3p4} and
\ref{s3p5}.

Although cosets of a subgroup are not themselves subgroups, the set
\begin{equation}\label{s3e1}
H_a = \{aha^{-1} | h \in H\}
\end{equation}
for a fixed element $a \in G$ is indeed a group. We prove it as 
\begin{prop}\label{s3p6}
$H_a$ is a subgroup of $G$.
\end{prop}
\begin{proof}
Consider two elements $x_1 = ah_1a^{-1}$ and $x_2 = ah_2a^{-1}$. Then $x_1^{-1}
x_2 = ah_1^{-1}a^{-1}ah_2 a^{-1} = a(h_1^{-1}h_2) a^{-1}$. From proposition
\ref{s3p1}, $h_1^{-1}h_2 \in H$ so that $x_1^{-1}x_2 \in H_a$.
\end{proof}

We also have
\begin{prop}\label{s3e7}
If $a \in H$, $H_a = H$.
\end{prop}
\begin{proof}
From the closure properties of $H$ it is clear that $H_a \subset H$. Let $x$
be an element of $H$ and $y = axa^{-1} \in H$. But $y$ is also a member of
$H_a$ so that $H \subset H_a$.
\end{proof}

\begin{rem}
The set $H_a$ contains elements conjugate to elements of $H$. We observed
previously that conjugate elements are `similar' in some sense. Therefore, it
should be no surprise that the `similar elements' also form a subgroup.
\end{rem}

\begin{defn}\label{s3d4}
If $H = H_a$ for all $a \in G$ then $H$ us called an invariant subgroup of $G$.
\end{defn}

\begin{rem}
Invariant subgroup is also called a normal subgroup. An invariant subgroup
remains unchanged under the conjugacy operation.
\end{rem}

\begin{prop}\label{s3e8}
If $H$ is an invariant subgroup of $G$ then $aH = Ha$ for all $a \in G$.
\end{prop}
\begin{proof}
$aha^{-1}$ must be equal to an element $h^\prime$ of $H$. Therefore, $ah =
h^\prime a$. Since this is true for all $h \in H$, $aH = Ha$.
\end{proof}

If $H$ is an invariant subgroup of $G$ then we can define the product of its
cosets as
\begin{equation}\label{s3e2}
aHbH  = \{ahbh^\prime | h, h^\prime \in H\}.
\end{equation}

\begin{prop}\label{s3e9}
The set of all cosets of an invariant group is closed under coset composition
defined by equation \eqref{s3e2}.
\end{prop}
\begin{proof}
$ahbh^\prime = a(hb)h^\prime$. Since $H$ is an invariant subgroup $bH = Hb$.
Therefore, we can write $hb = bh^{\prime\prime}$ and hence $ahbh^\prime = 
ab(h^{\prime\prime}h^\prime)$, which is a member of $abH$.
\end{proof}

If $H$ is an invariant subgroup then the set $G/H$ of its cosets with a 
composition law defined by equation \eqref{s3e2} forms a group. It is called
the factor group. Its closure is proved in proposition \eqref{s3e9}. The set
$eH = H$ is its identity. The set $a^{-1}H$ is the inverse of $aH$ and the
associativity of coset composition follows immediately from the associativity
of the elements of $G$.

\begin{defn}\label{s3d5}
The set $G/H$ of all cosets of an invariant subgroup $H$ of a group $G$ is
called the factor group.
\end{defn}

\begin{defn}\label{s3d6}
$G$ is called a simple group if it has no nontrivial invariant subgroups.
\end{defn}

\begin{defn}\label{s3d7}
$G$ is called a semisimple group if it has no nontrivial abelian invariant 
subgroups.
\end{defn}

From their definitions it is clear that
\begin{prop}\label{s3p10}
If $G$ is simple then it is semisimple.
\end{prop}

\begin{rem}
The converse need not be true.
\end{rem}

\section{Some important groups}\label{s4}
In this section we list a few groups that are of importance in physics.
\begin{enumerate}
\item $O(n)$, the set of all orthogonal $n \times n$ matrices. Recall that an
$n \times n$ matrix $M$ is orthogonal if $MM^T = M^T M = I_n$.
\item $SO(n)$, a subgroup of $O(n)$ of matrices whose determinant is $1$.
\item $U(n)$, the set of all unitary matrices. $M$ is a unitary matrix if
$MM^\dagger = M^\dagger M = I_n$.
\item $SU(n)$, a subgroup of $U(n)$ of matrices whose determinant is $1$.
\item $GL(n, \sor)$, the set of all invertible, real $n \times n$ matrices.
\item $SL(n, \sor)$ is a subgroup of $GL(n, \sor)$ of matrices with determinant
equal to $1$.
\item A $(2n) \times (2n)$ matrix $M$ is called a symplectic matrix if
\begin{equation}\label{s4e1}
M^T\Omega M = \Omega,
\end{equation}
where
\begin{equation}\label{s4e2}
\Omega = \begin{pmatrix}0 & I_n \\ -I_n & 0 \end{pmatrix}.
\end{equation}
The set of all symplectic matrices also forms a group. It is called $Sp(2n,
\sor)$.
\end{enumerate}

\section{Commutator subgroup}\label{s5}
\begin{defn}\label{s5d1}
A function $q: G \times G \rightarrow G$ defined as
\[
q(a, b) = aba^{-1}b^{-1}
\]
is called the commutator of $a$ and $b$.
\end{defn}

\begin{prop}\label{s5p1}
If $G$ is abelian then $q(a, b) = e$ for all $a, b \in G$.
\end{prop}
\begin{proof}
Since $G$ is abelian, $aba^{-1}b^{-1} = abb^{-1}a^{-1} = e$.
\end{proof}

\begin{prop}\label{s5p2}
The commutator function has the following properties:
\begin{itemize}
\item $q(a, b)^{-1} = q(b, a)$.
\item $cq(a, b)c^{-1} = q(cac^{-1}, cbc^{-1})$.
\end{itemize}
\end{prop}
\begin{proof}
$q(a, b)q(b, a) = aba^{-1}b^{-1}bab^{-1}a^{-1} = e$ implies that $q(b, a)$
is an inverse of $q(a, b)$. From proposition \eqref{s1p3} it is the only
inverse.

The second property follows from 
\begin{eqnarray*}
cq(a, b)c^{-1} &=& caba^{-1}b^{-1}c^{-1} \\
 &=& (cac^{-1})(cbc^{-1})(ca^{-1}c^{-1})(cb^{-1}c^{-1}) \\
 &=& (cac^{-1})(cbc^{-1})(cac^{-1})^{-1}(cbc^{-1})^{-1} \\
 &=& q(cac^{-1}, cbc^{-1}).
\end{eqnarray*}
\end{proof}

\begin{defn}\label{s5d2}
$Q(G, G) = \{\prod_{i=0}^nq(a_i, b_i) | a_i, b_i \in G, i \in \mathbf{N}\}$.
\end{defn}

\begin{rem}
$Q(G, G)$ is the set of arbitrary products of commutators of $G$.
\end{rem}
\begin{rem}
$Q(G,G) \subset G$.
\end{rem}

\begin{prop}\label{s5p3}
$Q(G, G)$ is an invariant subgroup of $G$.
\end{prop}
\begin{proof}
Let $x, y \in Q(G, G)$. Then $x = q(a, b)$ and $y = q(c, d)$ for some $a, b,
c, d \in G$. Consider $q(a, b)^{-1}q(c, d) = q(b, a)q(c, d) \in Q(G, G)$, by
definition. Therefore, by proposition \ref{s3p1}, $Q(G, G)$ is a subgroup of
$G$.

Let $c \in G$. Now $cq(a, b)c^{-1} = q(cac^{-1}, cbc^{-1}) \in Q(G, G)$. 
Therefore, $Q(G, G)_c = Q(G, G)$. Since $c$ was an arbitrary element of $G$,
$Q(G, G)_c = Q(G, G)$ for all $c \in G$. This makes is an invariant subgroup.
\end{proof}

\begin{prop}\label{s5p4}
The factor group $G/Q(G, G)$ is abelian.
\end{prop}
\begin{proof}
Recall that $G/Q(G, G)$ is a set of all cosets of $Q(G, G)$ with the coset
composition law defined in equation \eqref{s3e2}. By definition of $Q(G, G)$,
$x^{-1}y^{-1}xyQ(G, G) = Q(G, G)$. Therefore, $y^{-1}xyQ(G, G) = xQ(G, G)$ and
hence $xyQ(G, G) = yxQ(G, G)$ or $xQ(G, G) yQ(G, G) = yQ(G, G) xQ(G, G)$.
\end{proof}

\begin{prop}\label{s5p5}
Let $H$ be an invariant subgroup of $G$. $G/H$ is abelian if and only if $H
\supset Q(G, G)$.
\end{prop}
\begin{proof}
Let $G/H$ be abelian. Therefore, $abH = baH$ for all $a, b \in G$. In other
words, $H = b^{-1}a^{-1}baH$ or that $b^{-1}a^{-1}ba = q(b^{-1}, a^{-1}) \in
H$ for all $a, b \in G$. Therefore, $Q(G, G) \subset H$.

Now consider a normal subgroup $H \supset Q(G, G)$ and its coset $abH$.
Since $H \supset Q(G, G)$, $x = b^{-1}a^{-1}ba \in H$. Therefore, $abxH = abH$,
or $abb^{-1}a^{-1}ba H = abH$ or $baH = abH$.
\end{proof}

A statement equivalent to proposition \eqref{s5p5} is
\begin{prop}\label{s5p6}
$Q(G, G)$ is the smallest subgroup of $G$ such that its factor group is
abelian.
\end{prop}

Let $G_0 = G$ and $G_1 = Q(G, G) = Q(G_1, G_1)$. Then we saw that $G_1$ is an
invariant subgroup of $G_0$. In some sense $G_1$ is smaller than $G_0$. We can
define a sequence of groups
\begin{equation}\label{s5e1}
G_j = \begin{cases}
G & \text{ if } j = 0 \\
Q(G_{j-1}, G_{j-1}) & \text{ if } j \ge 1.
\end{cases}
\end{equation}
Clearly, each $G_{j}$ is an invariant subgroup of $G_{j-1}$ and is in some
sense smaller than $G_{j-1}$. 

\begin{defn}\label{s5d3}
A group $G$ is solvable if $G_j = {e}$ for a finite $j \in \mathbf{N}$, where
$G_j$ is defined by equation \eqref{s5e1}.
\end{defn}

If $G$ is abelian then $G_1 = Q(G, G) = \{e\}$. Therefore, an abelian group
is (trivially) solvable. Solvability in thus a generalisation of commutativity.

Recall definitions \ref{s3d6} of a simple group. From definition \ref{s5d3}
it immediately follows that
\begin{prop}\label{s5e7}
A solvable group is not simple.
\end{prop}
An equivalent, contrapositive statement is
\begin{prop}\label{s5e8}
A simple group is not solvable.
\end{prop}

\section{Mappings between groups}\label{s6}
\begin{defn}\label{s6d1}
A homomorphism from a group $G$ to a group $G^\prime$ is a mapping $\phi: G
\mapsto G^\prime$ such that $\phi(a)\phi(b) = \phi(ab)$ for all $a, b \in G$.
\end{defn}
\begin{rem}
A homomorphism, as its name suggests, indicates that $G$ and $G^\prime$ have
similar structure. The elements $a, b, ab$ of $G$ correspond to the elements
$\phi(a), \phi(b), \phi(a)\phi(b) = \phi(ab)$ of $G^\prime$. 
\end{rem}

\begin{prop}\label{s6p1}
If $\phi: G \mapsto G^\prime$ is a homomorphism then $\phi(e)$ is an identity
of $G^\prime$.
\end{prop}
\begin{proof}
$\phi(a)\phi(e) = \phi(ae) = \phi(a)$ for all $a \in G$.
\end{proof}

Similarly, one can show that
\begin{prop}\label{s6p2}
If $\phi: G \mapsto G^\prime$ is a homomorphism then $\phi(a^{-1}) = 
(\phi(a)^{-1})$ for all $a \in G$.
\end{prop}
\begin{proof}
Put $b = a^{-1}$ in the definition \ref{s6d1} to conclude that $\phi(a)
\phi(a^{-1}) = \phi(aa^{-1}) = \phi(e)$. From the previous proposition we 
know that $\phi(e)$ is the identity of $G^\prime$.
\end{proof}

\begin{prop}\label{s6p3}
If $\phi: G \mapsto G^\prime$ is a homomorphism then $\phi(G) \subset G^\prime$.
\end{prop}
\begin{proof}
Choose $b = e$ in the definition \ref{s6d1} to conclude that $\phi(a) \in
G^\prime$ for all $a \in G$.
\end{proof}

In fact, we can also show that
\begin{prop}\label{s6p4}
If $\phi: G \mapsto G^\prime$ is a homomorphism then $\phi(G)$ is a subgroup
of $G^\prime$.
\end{prop}
\begin{proof}
Let $y_1 = \phi(x_1)$ and $y_2 = \phi(x_2)$. From proposition \ref{s6p2}
$y_1^{-1} = \phi(x_1^{-1})$ so that $y^{-1}_1y_2 = \phi(x_1^{-1})\phi(x_2)
= \phi(x_1^{-1}x_2) \in \phi(G)$.
\end{proof}

\begin{defn}\label{s6d2}
If $\phi: G \mapsto G^\prime$ is a homomorphism then $\ker(\phi) = \{a \in G
| \phi(a) = e^\prime\}$ where $e^\prime$ is the identity of $G^\prime$, is
called the kernel of $\phi$.
\end{defn}

\begin{prop}\label{s6p5}
If $\phi: G \mapsto G^\prime$ is a homomorphism then $\ker(\phi)$ is a subgroup 
of $G$.
\end{prop}
\begin{proof}
Let $a, b \in \ker(\phi)$ then $\phi(a) = \phi(b) = e^\prime$. Now, 
$(\phi(a))^{-1} = (e^\prime)^{-1} \Rightarrow \phi(a^{-1}) = e^\prime$
so that $a^{-1} \in \ker(\phi)$. Therefore $\phi(a^{-1})\phi(b) = e^\prime
e^\prime = e^\prime \Rightarrow \phi(a^{-1}b) = e^\prime \Rightarrow a^{-1}b
\in \ker(\phi)$ making $\ker(\phi)$ is subgroup of $G$.
\end{proof}

\begin{prop}\label{s6p6}
If $\phi: G \mapsto G^\prime$ is a homomorphism then $\ker(\phi)$ is an
invariant subgroup of $G$.
\end{prop}
\begin{proof}
Let $a \in G$ then, using definition \ref{s3d1} $\ker(\phi)_a = \{a^{-1}ha|
h \in \ker(\phi)\}$. Now $\phi(a^{-1}ha) = e^\prime$ for all $a \in G$. 
Therefore $\ker(\phi)_a = \ker(\phi)$ for all $a \in G$.
\end{proof}

\begin{defn}\label{s6d3}
A homomorphism $\phi: G \mapsto G^\prime$ is an isomorphism if $\phi$ is 
bijective.
\end{defn}

\begin{prop}\label{s6p7}
If $\phi:G \mapsto G^\prime$ is a homomorphism then $G/\ker(\phi)$ is 
isomorphic to $\phi(G)$.
\end{prop}
\begin{proof}
$G/\ker(\phi) = \{a\ker(\phi)| a \in G\}$. Define a mapping $\psi:G/\ker(\phi)
\mapsto \phi(G)$ such that $\psi(a\ker(\phi)) = \phi(a)$. Now $\psi(a\ker(\phi)
b\ker(\phi)) = \psi(ab\ker(\phi)) = \phi(ab) = 
\psi(a\ker(\phi))\psi(b\ker(\phi)) $ so that $\psi$ is indeed a homomorphism.

Let $g^\prime \in \phi(G)$. Therefore, there exists $g \in G$ such that 
$\phi(g) = g^\prime$. Therefore $\psi(g\ker(\phi)) = g^\prime$. Thus, $\psi$
is surjective.

Let, if possible, $\psi(a\ker(\phi)) = \psi(b\ker(\phi)) = g^\prime$. Therefore
there exists $x, y \in \ker(\phi)$ such that $\phi(ax) = \phi(by)$, or $\phi(a)
= (\phi(x))^{-1}\phi(b)\phi(y) = \phi(b)$, because $x$ and $y$ belong to 
$\ker(\phi)$ and therefore map to $e^\prime$, the identity in $G^\prime$. This
shows that $\psi$ is injective.
\end{proof}

\begin{defn}\label{s6d4}
An automorphism on a group $G$ is an isomorphism $G \mapsto G$.
\end{defn}

\section{Direct and semidirect products}\label{s7}
\begin{defn}\label{s7d1}
The direct product $G_1 \times G_2$ of two groups $G_1$ and $G_2$ is the set
$\{(g_1, g_2) | g_1 \in G_1, g_2 \in G_2\}$ with the binary operation
\[
(a_1, a_2)(b_1, b_2) = (a_1b_1, a_2b_2).
\]
\end{defn}
\begin{rem}
Note that $a_1, b_1 \in G_1$ so that $a_1b_1$ is a well-defined operation. 
Same holds good for $a_2b_2$.
\end{rem}

\begin{rem}
The difference between direct product and direct sum is explained in 
\href{https://math.stackexchange.com/questions/2412232/direct-product-vs-direct-sum-of-infinite-dimensional-vector-spaces}{Wikipedia} article.
\end{rem}

It is easy to confirm that
\begin{thm}\label{s7t1}
$G_1 \times G_2$ is a group.
\end{thm}
\begin{proof}
The operation is clearly closed. $(e_1, e_2)$ is the identity, $(a^{-1}_1, 
b_1^{-1})$ is the inverse of $(a, b)$ and the associativity of the binary
operation follows from the associativity of components. 
\end{proof}

\begin{defn}\label{s7d2}
Let $N$ and $H$ be two groups, $\aut{N}$ be the set of all automorphisms on $N$ 
and let $\varphi : H \rightarrow \aut{N}$ be a homomorphism. Then a semidirect 
product of $N$ and $H$ with respect to $\varphi$, denoted $N \rtimes_\varphi H$,
is defined the set $N \times H$ with a binary operation $\ast$ defined as
\[
(n_1, h_1)\ast(n_2, h_2) = (n_1 + \varphi(h_1)(n_2), h_1 \cdot h_2),
\]
where $+$ is the group operation of $N$ and $\cdot$ that of $H$.
\end{defn}

\begin{prop}\label{s7t2}
$(N \rtimes_\varphi H, \ast)$ is a group.
\end{prop}
\begin{proof}
Closure of $\ast$ follows from the definition. 

Let $e_N$ and $e_H$ be identities of $N$ and $H$. Therefore, $(e_N, e_H)\ast
(n, h) = (e_N + \varphi(e_H)(n), e_H \cdot h)$. Since $\varphi$ is a 
homomorphism $\varphi(e_H) = I$, where $I$ is the identity isomorphism, so that 
$(e_N, e_H)\ast(n, h) = (e_N + I(n), h) = (n, h)$. Similarly $(n, h)\ast
(e_N, e_H) = (n + \varphi(h)(e_N), e_H \cdot h) = (n + e_N, h) = (n, h)$. Thus, 
$(e_N, e_H)$ is an identity.

Consider 
\[
(n, h) \ast (\varphi(h^{-1})(n^{-1}), h^{-1}) = 
(n + \varphi(h)\varphi(h^{-1})(n^{-1}), h\cdot h^{-1})
\]
Since $\varphi$ is a homomorphism, $\varphi(h)\varphi(h^{-1}) = \varphi(hh^{-1})
 = \varphi(e_H) = I$ so that,
\[
(n, h) \ast (\varphi(h^{-1})(n^{-1}), h^{-1}) = 
(n + I(n^{-1}), e_H) = (n + n^{-1}, e_H) = (e_N, e_H)
\]
Similarly, $(\varphi(h^{-1})(n^{-1}), h^{-1}) \ast (n, h) = (e_N, e_H)$. Thus 
$(\varphi(h^{-1})(n^{-1}), h^{-1})$ is an inverse.

Lastly 
\begin{eqnarray*}
(n_1, h_1) \ast ((n_2, h_2) \ast (n_3, h_3)) &=& 
(n_1, h_1) \ast (n_2 + \varphi(h_2)(n_3), h_2 \cdot h_3) \\
 &=& (n_1 + \varphi(h_1)(n_2 + \varphi(h_2)(n_3)), h_1 \cdot (h_2 \cdot h_3)) \\
 &=& (n_1 + \varphi(h_1)(n_2) + \varphi(h_1)\varphi(h_2)(n_3), (h_1 \cdot h_2) 
 \cdot h_3) \\
 &=& (n_1 + \varphi(h_1)(n_2) + \varphi(h_1h_2)(n_3), (h_1 \cdot h_2) \cdot h_3) \\
 &=& (n_1 + \varphi(h_1) + n_2, h_1 \cdot h_2) \ast (n_3, h_3) \\
 &=& ((n_1, h_1) \ast ((n_2, h_2)) \ast (n_3, h_3)
\end{eqnarray*}
Thus $\ast$ is an associative operation.
\end{proof}

If $\varphi$ in definition \ref{s7d2} is an identity then 
\begin{equation}\label{s7e1}
(n_1, h_1) \ast (n_2, h_2) = (n_1 + n_2, h_1h_2).
\end{equation}
Thus, the semidirect product becomes the direct product. We also observe that
\begin{enumerate}
\item $\{e_N\} \rtimes_\varphi H$ is a subgroup of $N \rtimes_{\varphi} H$ and 
it is isomorphic to $H$.
\item $N \rtimes_\varphi \{e_H\}$ is a subgroup of $N \rtimes_{\varphi} H$ and 
it is isomorphic to $N$.
\end{enumerate}

\begin{prop}\label{s7p1}
$N \rtimes_\varphi \{e_H\}$ is an invariant subgroup of $N \rtimes_\varphi H$.
\end{prop}
\begin{proof}
Let $(n, e_H)$ be an element of $N \rtimes_\varphi \{e_H\}$ and $(m, h)$ be
an element of $N \rtimes_\varphi H$. 
\[
(m,h)\ast(n,e_H)\ast(m,h)^{-1} = (m + \varphi(e_H)(n), h e_H)\ast(m, h)^{-1}.
\]
Since $\varphi(e_H)$ is an identity mapping,
\[
(m,h)\ast(n,e_H)\ast(m,h)^{-1} = (m + n, h)\ast(m, h)^{-1}.
\]
Since $(m, h)^{-1} = (\varphi(h^{-1})(m^{-1}), h^{-1})$
\begin{eqnarray*}
(m,h)\ast(n,e_H)\ast(m,h)^{-1} &=& (m + n, h)\ast
(\varphi(h^{-1})(m^{-1}), h^{-1}) \\
 &=& (m + n + \varphi(h)(\varphi(h^{-1})(m^{-1})), hh^{-1}) 
\end{eqnarray*}
Since $\varphi$ is an automorphism $\varphi(h^{-1}) = \varphi^{-1}(h)$ so that
\[
(m,h)\ast(n,e_H)\ast(m,h)^{-1} = (m + n + m^{-1}, e_H) = (n, e_H).
\]
\end{proof}

\nocite{*}
\bibliographystyle{plain}
\bibliography{gt}
\end{document}
