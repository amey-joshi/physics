\chapter{Lagrange's Equations of the Second Kind}\label{c2}
\section{Generalised coordinates}\label{c2s1}
In the previous chapter we observed that in a system of $N$ particles and $r$
constraints, the $3N$ virtual displacements $\delta x^{(n)}_i$ are not 
independent. Any subset of $f = 3N - r$ virtual displacements is independent. 
$f$ is called the \emph{number of degrees of freedom} of the system.

Consider a system of two particles connected by a rigid rod. The two particles
each need three Cartesian coordinates. But the constraint makes only five of
them independent. It is not clear which five to choose. Although the lack of
independence of virtual displacements in all coordinates is taken into account
by the Lagrange's equations of the first kind, one still has to describe the 
system in terms of $6$ coordinates. This situation gets worse in the case of
a rigid body. If it made up of $N$ particles then we need $3N$ coordinates to
describe its position when only $6$ of them are independent.

Generalised coordinates are any $f$ numbers that suffice to describe the
position of a system of $f$ degrees of freedom. For example, a rigid body
with $N$ particles can be described by three Cartesian coordinates of its
centre of mass and three Euler angles. We shall denote the $f$ generalised
coordiates by $\{q_1, \ldots, q_f\}$.

Let us therefore consider a system of $N$ particles and $f$ degrees of freedom
and let $q_1, \ldots, q_f$ be its generalised coordinates. Then the Cartesian
coordinates of its particles can be written as
\begin{eqnarray}
x_i &=& x_i(q_1, \ldots, q_f, t) \\
y_i &=& y_i(q_1, \ldots, q_f, t) \\
z_i &=& z_i(q_1, \ldots, q_f, t).
\end{eqnarray}
Note that the Cartesian coordinates of a particle are allowed to depend on 
\emph{all} generalised coordinates. Unlike their Cartesian counterparts, the
generalised coordinates are not tied to a single particle. Further, note that
the Cartesian coordinates may depend on time as well. From these equations
we have
\begin{eqnarray}
dx_i &=& \pd{x_i}{q_j}dq_j + \pd{x_i}{t}dt \\
dy_i &=& \pd{y_i}{q_j}dq_j + \pd{y_i}{t}dt \\
dz_i &=& \pd{z_i}{q_j}dq_j + \pd{z_i}{t}dt
\end{eqnarray}
We have used the summation convention in these three equations. Although
the functional forms of $x_i, y_i, z_i$ are arbitrary, the differentials
$dx_i, dy_i, dz_i$ are linear functions of $dq_j$ and $dt$.

The three Cartesian velocities are
\begin{eqnarray}
\dot{x}_i &=& \pd{x_i}{q_j}\dot{q}_j + \pd{x_i}{t} \label{c2s1e7} \\
\dot{y}_i &=& \pd{y_i}{q_j}\dot{q}_j + \pd{y_i}{t} \label{c2s1e8} \\
\dot{z}_i &=& \pd{z_i}{q_j}\dot{q}_j + \pd{z_i}{t} \label{c2s1e9}  
\end{eqnarray}
Note that the symbol $\dot{x}_i$ denotes the total time derivative of $x_i$ and
it differs from $\partial x_i/\partial t$. The generalised coordinates, however,
do not depend explicitly on time. The kinetic energy of the system is
\begin{eqnarray}
T &=& \frac{1}{2}m_i(\dot{x}_i^2 + \dot{y}_i^2 + \dot{z}_i^2) \nonumber \\
 &=& \frac{1}{2}m_i\left(\pd{x_i}{q_j}\dot{q}_j + \pd{x_i}{t}\right)
                   \left(\pd{x_i}{q_k}\dot{q}_k + \pd{x_i}{t}\right) + 
     \nonumber \\
 & & \frac{1}{2}m_i\left(\pd{y_i}{q_j}\dot{q}_j + \pd{y_i}{t}\right)
                   \left(\pd{y_i}{q_k}\dot{q}_k + \pd{y_i}{t}\right) + 
	 \nonumber \\
 & & \frac{1}{2}m_i\left(\pd{z_i}{q_j}\dot{q}_j + \pd{z_i}{t}\right)
                   \left(\pd{z_i}{q_k}\dot{q}_k + \pd{z_i}{t}\right) 
	 \label{c1s1e10} 
\end{eqnarray}
We now compute the derivative of $T$ with respect to the generalised 
velocity $\dot{q}_l$.
\begin{eqnarray}
\pd{T}{\dot{q}_l} &=& 
  m_i\left(\pd{x_i}{q_j}\dot{q}_j + \pd{x_i}{t}\right)\pd{x_i}{q_l} +
 m_i\left(\pd{y_i}{q_j}\dot{q}_j + \pd{y_i}{t}\right)\pd{y_i}{q_l} +
  \nonumber \\
& &  m_i\left(\pd{z_i}{q_j}\dot{q}_j + \pd{z_i}{t}\right)\pd{z_i}{q_l} 
  \label{c2s1e11} 
\end{eqnarray}
Using equations \eqref{c2s1e7}, \eqref{c2s1e8} and \eqref{c2s1e9} we get
\begin{equation}\label{c2s1e12}
\pd{T}{\dot{q}_l} = m_i\left(\dot{x}_i\pd{x_i}{q_l} + \dot{y}_i\pd{y_i}{q_l} + 
 \dot{z}_i\pd{z_i}{q_l}\right).
\end{equation}
We now take the total time derivative of this equation.
\begin{eqnarray}
\frac{d}{dt}\left(\pd{T}{\dot{q}_l}\right) &=& 
  m_i\left(\ddot{x}_i\pd{x_i}{q_l} + \dot{x}_i\frac{d}{dt}
  \left(\pd{x_i}{q_l}\right)\right)  + 
  m_i\left(\ddot{y}_i\pd{y_i}{q_l} + \dot{y}_i\frac{d}{dt}
  \left(\pd{y_i}{q_l}\right)\right) + \nonumber \\
& & m_i\left(\ddot{z}_i\pd{z_i}{q_l} + \dot{z}_i\frac{d}{dt}
  \left(\pd{z_i}{q_l}\right)\right) 
\end{eqnarray}
Evaluating the total time derivatives on the right hand side,
\begin{eqnarray}
\frac{d}{dt}\left(\pd{T}{\dot{q}_l}\right) &=& 
 m_i\left(\ddot{x}_i\pd{x_i}{q_l} + 
 \dot{x}_i\dot{q}_k\frac{\partial^2 x_i}{\partial q_k\partial q_l} +
 \dot{x_i}\frac{\partial}{\partial q_l}\pd{x_i}{t}\right) \nonumber \\
 & & m_i\left(\ddot{y}_i\pd{y_i}{q_l} + 
 \dot{y}_i\dot{q}_k\frac{\partial^2 y_i}{\partial q_k\partial q_l} +
 \dot{y_i}\frac{\partial}{\partial q_l}\pd{y_i}{t}\right) \nonumber \\
 & & m_i\left(\ddot{z}_i\pd{z_i}{q_l} + 
 \dot{z}_i\dot{q}_k\frac{\partial^2 z_i}{\partial q_k\partial q_l} +
 \dot{z_i}\frac{\partial}{\partial q_l}\pd{z_i}{t}\right) \label{c2s1e14}
\end{eqnarray}
We now observe that
\[
\pd{\dot{q_1}}{q_k} = \frac{\partial}{\partial t}\delta_{lk} = 0
\]
so that we can write
\begin{eqnarray}
\frac{d}{dt}\left(\pd{T}{\dot{q}_l}\right) &=& 
 m_i\left(\ddot{x}_i\pd{x_i}{q_l} + 
 \dot{x}_i\frac{\partial}{\partial q_l}
 \left(\dot{q}_k\pd{x_i}{q_k} + \pd{x_i}{t}\right)\right) \nonumber \\
 & &  m_i\left(\ddot{y}_i\pd{y_i}{q_l} + 
 \dot{y}_i\frac{\partial}{\partial q_l}
 \left(\dot{q}_k\pd{y_i}{q_k} + \pd{y_i}{t}\right)\right) \nonumber \\
 & &  m_i\left(\ddot{z}_i\pd{z_i}{q_l} + 
 \dot{z}_i\frac{\partial}{\partial q_l}
 \left(\dot{q}_k\pd{z_i}{q_k} + \pd{z_i}{t}\right)\right) \label{c2s1e15}
\end{eqnarray}
Using equations \eqref{c2s1e7}, \eqref{c2s1e8} and \eqref{c2s1e9} we get
\[
\frac{d}{dt}\left(\pd{T}{\dot{q}_l}\right) =
m_i\left[\left(\ddot{x}_i\pd{x_i}{q_l} + \dot{x}_i\pd{\dot{x}_i}{{q}_l}\right) 
 + \nonumber \\
\left(\ddot{y}_i\pd{y_i}{q_l} + \dot{y}_i\pd{\dot{y}_i}{{q}_l}\right) 
 + \nonumber \\
\left(\ddot{z}_i\pd{z_i}{q_l} + \dot{z}_i\pd{\dot{z}_i}{{q}_l}\right)\right].
\]
Rearranging it
\[
\frac{d}{dt}\left(\pd{T}{\dot{q}_l}\right) =
\pd{T}{q_l} + m_i\left(\ddot{x}_i\pd{x_i}{q_l} +
\ddot{y}_i\pd{y_i}{q_l} +\ddot{z}_i\pd{z_i}{q_l}\right)
\]
or
\begin{equation}\label{c2s1e16}
\frac{d}{dt}\left(\pd{T}{\dot{q}_l}\right) - \pd{T}{q_l} =
 m_i\left(\ddot{x}_i\pd{x_i}{q_l} +
 \ddot{y}_i\pd{y_i}{q_l} +\ddot{z}_i\pd{z_i}{q_l}\right)
\end{equation}
Multiplying both sides by $\delta q_l$,
\begin{equation}\label{c2s1e17}
\left(\frac{d}{dt}\left(\pd{T}{\dot{q}_l}\right) - \pd{T}{q_l}\right)\delta q_l
= m_i\left(\ddot{x}_i\delta x_i + \ddot{y}_i\delta y_i +
 \ddot{z}_i\delta z_i\right)
\end{equation}
Comparing with equation \eqref{c1s2e13} we surmise that the left hand side
of the above equation is
\begin{equation}
\frac{d}{dt}\left(\pd{T}{\dot{q}_l}\right) - \pd{T}{q_l}\delta q_l =
X_i\delta x_i + Y_i\delta y_i + Z_i \delta z_i,
\end{equation}
where $(X_i, Y_i, Z_i)$ are components of the external force on the $i$th
particle. If this force is conservative, we can write it in terms of a 
potential $V$ as
\begin{eqnarray}
\frac{d}{dt}\left(\pd{T}{\dot{q}_l}\right) - \pd{T}{q_l}\delta q_l &=&
\pd{V}{x_i}\delta x_i + \pd{V}{z_i}\delta y_i + \pd{V}{z_i} \delta z_i \nonumber \\
&=&
\pd{V}{x_i}\pd{x_i}{q_l}\delta q_l + \pd{V}{z_i}\pd{y_i}{q_l}\delta q_l + 
\pd{V}{z_i} \pd{z_i}{q_l}\delta q_l \nonumber \\
&=& \pd{V}{q_1}\delta q_l \label{c1s1e20}
\end{eqnarray}
Note that the last term here is not $3\partial V/\partial q_l$ because $V$
depends on all $x_1, \ldots, z_N$ and each of these is a function of all 
generalised coordinates. If $V$ \emph{does not} depend on generalised 
velocities then we can write \eqref{c1s1e20} as
\begin{equation}\label{c2s1e21}
\frac{d}{dt}\pd{L}{\dot{q}_l} - \pd{L}{q_l} = 0,
\end{equation}
where the function
\begin{equation}\label{c1s1e22}
L(q_1, \ldots, q_f, \dot{q}_1, \ldots, \dot{q}_f) = T - V
\end{equation}
is the Lagrangian of the system and equations \eqref{c2s1e21} are called
Lagrange's equation of the second kind.

A few remarks about \eqref{c2s1e21}:
\begin{itemize}
\item Often times the phrase `second kind' is omitted and equations
are called Lagrange's equations.
\item They equations are presently derived only for conservative forces.
\item They do not involve the forces of constraint. Neither do they insist on
the Cartesian coordinates. 
\item They are valid in all frames of references. However $T$ depends on 
generalised velocities and will change across frames of reference. Therefore,
$T$ should always be calculated with respect to an inertial frame of 
reference \cite[p. 31]{akr}.
\end{itemize}

Problems.
\begin{enumerate}
\item A uniform rod leans against a wall \cite[Problem 1, chapter 4]{akr}.
Let $l$ be its length. Its motion is confined to a plane. Its unconstrained
motion has three degrees of freedom, the position of its centre of mass and
its orientation. In the present problem, the $x$ coordinate of its end
touching the wall is constrained to be $0$ and the $y$ coordinate of its 
end touching the floor is constrained to be $0$. Therefore, one generalised
coordinate suffices to describe the motion. 

We choose $\theta$ the (smaller) angle made by the rod with the $x$ axis. If
the rod is in the first quadrant then it makes and angle $\pi -\theta$ with
the positive $x$ axis.

The centre of the rod is at point $(0, l/2)$ when $\theta = \pi/2$ and it is
at $(0, l/2)$ when $\theta = 0$. It traces the arc of a circle of radius
$l/2$ as the rod slides down the path.

The moment of inertia of the rod abouts its centre of mass is $ml^2/12$. 
Therefore, the rotational kinetic energy is
\[
T_r = \frac{I}{2}\dot{\theta}^2 = \frac{m}{24}l^2\dot{\theta}^2
\]
The translational kinetic energy of its centre of mass is
\[
T_c = \frac{m}{2}v^2 = \frac{m}{8}l^2\dot{\theta}^2
\]
The total kinetic energy is 
\[
T = \frac{ml^2}{12}\dot{\theta}^2.
\]

We will now find the potential energy. If $dm$ is the mass element of the
rod at a height $h$ then
\[
dV = gy dm
\]
If $\lambda$ be the linear mass density of the rod, $dm = \lambda ds = mds/l$.
Thus,
\[
dV = \frac{mg}{l} yds = \frac{mg}{l}y \sqrt{(dx)^2 + (dy)^2} = 
\frac{mg}{l}ydy\sqrt{1 + \left(\frac{dx}{dy}\right)^2}.
\]
Now the equation of the straight line representing the rod is $y = y_0 -
\tan\theta x$ so that 
\[
\frac{dx}{dy} = \cot\theta
\]
so that
\[
dV = \frac{mg}{l}\csc\theta ydy
\]
or
\[
V = \int_{lsin\theta}^0 dV = -\frac{mg}{2}l\sin\theta.
\]
Therefore, the Lagrangian is
\[
L = \frac{ml^2}{12}\dot{\theta}^2 + \frac{mg}{2}l\sin\theta
\]
and the equation of motion is
\[
\frac{ml^2}{6}\ddot{\theta} - \frac{mgl}{2}\cos\theta = 0 \Rightarrow
\frac{l}{3}\ddot{\theta} - g\cos\theta = 0 \Rightarrow \ddot{x} = 
\frac{3g}{l}\cos\theta.
\]

\item An old model of the He atom consisted of a fixed nucleus and two
electrons on the opposite ends of the diameter of the circle centred at
the nucleus \cite[Problem 3, chapter 4]{akr}. Without the constraints, the
motion of the two electrons in a plane would have required four coordinates.
Their constraints are:
\begin{itemize}
\item They are always at the opposite ends of a diameter. This means that
we can consider them to be end of a rigid, weightless rod.
\item Their centre of mass is fixed at the nucleus. This fixes the two 
coordinates of their centre of mass.
\end{itemize}
This leaves the system with just one degree of freedom. If $\theta$ is the
angle made by one electron with the positive $x$ axis, the other electron
makes an angle $\theta + \pi/2$. If $m$ is the mass of the electrons, their
kinetic energy is
\[
T = \frac{m}{2}r^2\dot{\theta}^2 + \frac{m}{2}r^2\dot{\theta}^2 =
mr^2\dot{\theta}^2.
\]
Their potential energy is
\[
V = -\frac{2e^2}{r} - \frac{2e^2}{r} = -\frac{4e^2}{r}
\]
so that the Lagrangian is
\[
L = mr^2\dot{\theta}^2 + \frac{4e^2}{r^2}.
\]
The equation of motion is $\ddot{\theta} = 0$ or that $\theta = \alpha t + 
\beta$, where $\alpha$ and $\beta$ are the initial conditions.
\end{enumerate}
