\documentclass{article}
\usepackage{amsmath, amssymb, amsfonts, amsthm}
\usepackage{bm}
\usepackage{color}
\newcommand{\pd}[2]{\frac{\partial{#1}}{\partial{#2}}}

\numberwithin{equation}{section}
\let\vec\bm

\theoremstyle{plain}
\newtheorem{thm}{Theorem}
\numberwithin{thm}{section}

\theoremstyle{plain}
\newtheorem{prop}{Proposition}
\numberwithin{prop}{section}

\theoremstyle{definition}
\newtheorem{defn}{Definition}
\numberwithin{defn}{section}

\theoremstyle{remark}
\newtheorem*{rem}{Remark}

\newtheorem*{cor}{Corollary}

\title{Hamiltonian Mechanics - A Quick Recapitulation}
\author{Amey Joshi}
\date{06-Mar-2021}
\begin{document}

\maketitle
\abstract{This article summarises the main ideas of Hamiltonian mechanics and
illustrates them with a few examples.}

\section{The Hamiltonian function}\label{s1}
Consider a system of $n$ degrees of freedom described by the generalised 
coordinates $q_1, \ldots, q_n$ and a Lagrangian function $L(q_i, \dot{q}_i, t)$.
The generalised momentum is defined as
\begin{equation}\label{s1e1}
p_i = \frac{\partial{L}}{\partial\dot{q}_i}.
\end{equation}
The Legendre-Fenchel transformation of a function $f:X \rightarrow 
\mathbf{R}$, where $X \subset \mathbf{R}^n$ is defined as
\begin{equation}\label{s1e2}
f^\ast(x^\ast) = \sup_{x \in X}\left((x^\ast, x) - f(x)\right), 
\end{equation}
for all $x^\ast \in X^\ast$. The set $X^\ast$ is 
\begin{equation}\label{s1e3}
X^\ast = \left\{x^\ast \in \mathbf{R}^n : \sup_{x \in X}
         \left((x^\ast, x) - f(x)\right) < \infty\right\}.
\end{equation}
The expression $(x^\ast, x)$ is the dot product of $x^\ast$ and $x$. Let us 
assume that the function $f$ is differentiable and let us define
\begin{equation}\label{s1e4}
F(x, x^\ast) = (x^\ast, x) - f(x).
\end{equation}
Then,
\[
\frac{\partial F}{\partial x_i} = x^\ast_i - \frac{\partial f}{\partial x_i}
\]
so that the supremum of the set $(x^\ast, x) - f(x)$ is indeed the maximum
of the function $F$. In this case, the Legendre-Fenchel transform becomes
\begin{equation}\label{s1e5}
f^\ast(x^\ast) = \left((x^\ast, D_x(f)) - f(x)\right).
\end{equation}
The symbol $D_x(f)$ stands for the gradient of $f$ with respect to $x$.

The Lagrangian $L(q_i, \dot{q}_i, t)$ is assumed to be a differentiable 
throughout its domain so that its Legendre-Fenchel transformation with respect 
to the variables $\dot{q}_i^\ast$ is
\begin{equation}\label{s1e6}
L^\ast(q_i, \dot{q}_i^\ast, t) = \sum_{i=1}^n \dot{q}_i^\ast \dot{q}_i - 
L(q_i, \dot{q}_i^\ast, t),
\end{equation}
where
\begin{equation}\label{s1e7}
\dot{q}_i^\ast = \frac{\partial L}{\partial\dot{q}_i}.
\end{equation}
The right hand side is defined, in equation \eqref{s1e1}, to be the generalised 
momeumtum $p_i$. Therefore, equation \eqref{s1e6} is written as
\begin{equation}\label{s1e8}
L^\ast(q_i, p_i, t) = \sum_{i=1}^n p_i\dot{q}_i - L(q_i, \dot{q}_i, t).
\end{equation}
The function $L^\ast$ is called the Hamiltonian of the system and is denoted
by $H$.

Why did we not define the function $H$ as
\begin{equation}\label{s1e9}
H(q_i, p_i, t) = \sum_{i=1}^n p_i\dot{q}_i - L(q_i, \dot{q}_i, t)?
\end{equation}
Written in this manner, the right hand side is a function of $q_i, \dot{q}_i,
p_i$ and $t$. However, the supremum over $x$ in equation \eqref{s1e2} ensures
that the result is function of $x^\ast$ alone. The differentiability of $L$
results in the simple form of $L^\ast$, (or $H$), and we are also guaranteed
that this latter function does not depend on $\dot{q}_i$
\cite{touchette2005legendre}. 

From equation \eqref{s1e9} we have
\begin{eqnarray}
\frac{\partial{H}}{\partial q_i} &=& 
  -\frac{\partial L}{\partial q_i} \label{s1e10} \\
\frac{\partial{H}}{\partial p_i} &=& \dot{q}_i \label{s1e11}.
\end{eqnarray}
Since the Lagrangian $L$ satisfies the Euler-Lagrange equation
\begin{equation}\label{s1e12}
\frac{\partial L}{\partial q_i} = \frac{d}{dt}
\frac{\partial L}{\partial\dot{q}_i} = \dot{p}_i,
\end{equation}
the pair of equations \eqref{s1e10} and \eqref{s1e11} become
\begin{eqnarray}
\frac{\partial{H}}{\partial q_i} &=& -\dot{p}_i \label{s1e13} \\
\frac{\partial{H}}{\partial p_i} &=& \dot{q}_i. \label{s1e14}
\end{eqnarray}
These are Hamilton's canonical equations.

\section{The physical significance of the Hamiltonian function}\label{s2}
Let us assume that our system can also be described using $N$ cartesian
coordinates
\begin{equation}\label{s2e1}
x_i = x_i(q_1, \ldots, q_n), i = 1, \ldots, N.
\end{equation}
We can then write
\begin{equation}\label{s2e2}
\dot{x}_i = \sum_{j=1}^n\frac{\partial x_i}{\partial q_j}\dot{q}_j.
\end{equation}
If $m_i$ is the mass associated with the coordinate $x_i$ then its kinetic 
energy is
\begin{equation}\label{s2e3}
T = \frac{1}{2}\sum_{i=1}^N m_i\dot{x}_i^2 = 
\sum_{j,k=1}^n \alpha_{j,k}\dot{q}_j\dot{q}_k,
\end{equation}
where
\begin{equation}\label{s2e4}
\alpha_{j, k} = \sum_{i=1}^N m_i\frac{\partial x_i}{\partial q_j}
\frac{\partial x_i}{\partial q_k}
\end{equation}
is a function of $q_j, q_k$ alone.
\begin{prop}\label{s2p1}
$T$ is a homogeneous function of the generalised velocities.
\end{prop}
\begin{proof}
\[
T(q_1, \ldots, q_n, \lambda\dot{q}_1, \ldots, \lambda\dot{q}_n) = 
\lambda^2 T(q_1, \ldots, q_n, \dot{q}_1, \ldots, \dot{q}_n).
\]
\end{proof}
By Euler's theorem on homogeneous functions,
\begin{equation}\label{s2e5}
\sum_{i=1}^n \dot{q}_i\frac{\partial T}{\partial\dot{q}_i} = 2T.
\end{equation}
\begin{prop}\label{s2p2}
If the potential energy of the system is independent of the generalised
velocities and time then
\[
\sum_{i=1}^n \dot{q}_ip_i = 2T.
\]
\end{prop}
\begin{proof}
We can write the Lagrangian as
\[
L(q_i, \dot{q}_i) = T(q_i, \dot{q}_i) - V(q_i)
\]
so that
\[
\frac{\partial L}{\partial\dot{q}_i} = \frac{\partial T}{\partial\dot{q}_i}.
\]
The proposition follows from the above equation and \eqref{s1e1}.
\end{proof}
\begin{rem}
The Lagrangian of a point mass descending in a column of viscous fluid is
\begin{equation}\label{s2e6}
L = \exp\left(\frac{\gamma t}{m}\right)\left(\frac{m\dot{x}^2}{2} - U(x)\right),
\end{equation}
which is not of the form $T - V$. 

The Lagrangian of a particle of charge $q$ and mass $m$ in an electromagnetic
field is 
\begin{equation}\label{s2e7}
L = \frac{1}{2}mv^2 + q\vec{A}\cdot\vec{v} - q\phi,
\end{equation}
where $\vec{A}$ and $\phi$ are the vector and scalar potentials of the fields.
Neither is this function a difference of the kinetic energy $mv^2/2$ and 
the potential energy $q\phi$.
\end{rem}

\begin{prop}\label{s2p3}
If the potential energy of the system is independent of the generalised
velocities and time then $H = T + V$.
\end{prop}
\begin{proof}
The Lagrangian can be written as $L(q_i, \dot{q}_i) = T(q_i, \dot{q}_i)
- V(q_i)$. The proposition follows from this form of the Lagrangian, the
definition \eqref{s1e9} of the Hamiltonian and proposition \ref{s2p2}.
\end{proof}

\begin{prop}\label{s2p4}
If the Lagrangian is independent of time then the Hamiltonian is a constant
of motion.
\end{prop}
\begin{proof}
\[
\frac{dH}{dt} = \sum_{i=1}^n\dot{p}_i\dot{q}_i + \sum_{i=1}^np_i\ddot{q}_i
- \sum_{i=1}^n\frac{\partial L}{\partial q_i}\dot{q}_i 
- \sum_{i=1}^n\frac{\partial L}{\partial\dot{q}_i}\ddot{q}_i = 0.
\]
\end{proof}

From propositions \ref{s2p3} and \ref{s2p4} it follows that
\begin{thm}\label{s2t1}
If a system can be described the Lagarangian of the form
\[
L(q_i, \dot{q}_i) = T(q_i, \dot{q}_i) - V(q_i)
\]
then the Hamiltonian of the system is the total energy of the system and it
is a constant.
\end{thm}

\section{The phase space}\label{s3}
A system with $n$ degrees of freedom can be described by $n$ generalised
coordinates $q_1, \ldots, q_n$. Its behaviour can be described either by the
Lagrangian $L(q_i, \dot{q}_i, t)$ or the Hamiltonian $H(q_i, p_i, t)$. In
the former case, the system can be represented as a point in a 
$n+1$-dimensional configuration space whose coordinates as $q_1, \ldots, q_n, t$
or a $2n+1$-dimensional phase space whose coordinates are $q_1, \ldots, q_n,
p_1, \ldots, p_n, t$.

The motion of the system appears as a curve in either the configuration space
or the phase space. The benefit of the phase space is that the curves 
representing the motion of the system do not cross each other. For if they did
then at the point of intersection the system will have two possible ways of
evolution.

Consider an ensemble of systems represented by the \emph{same} Hamiltonian. 
Each of these is represented by a point in the phase space. If $\rho$ is the
number density of the systems then it satisfies the equation of continuity,
\begin{equation}\label{s3e1}
\frac{\partial\rho}{\partial t} + 
\sum_{j=1}^n\frac{\partial(\rho\dot{q}_j)}{\partial q_j} + 
\sum_{j=1}^n\frac{\partial(\rho\dot{p}_j)}{\partial p_j} = 0
\end{equation}
analogous to the equation of continuity of a fluid of density $\rho_m$ and
moving with a velocity $\vec{v}$,
\begin{equation}\label{s3e2}
\frac{\partial\rho_m}{\partial t} + \nabla\cdot(\rho_m\vec{v}) = 0.
\end{equation}
Equation \eqref{s3e1} can be written as
\begin{equation}\label{s3e3}
\frac{\partial\rho}{\partial t} + 
\sum_{j=1}^n\frac{\partial\rho}{\partial q_j} \dot{q}_j + 
\sum_{j=1}^n\frac{\partial\rho}{\partial p_j} \dot{p}_j + 
\rho\left(\sum_{j=1}^n\frac{\partial\dot{q}_j}{\partial q_j} + 
\sum_{j=1}^n\frac{\partial\dot{p}_j}{\partial p_j}\right)  = 0
\end{equation}
The last term vanishes because of Hamilton's equations \eqref{s1e13} and
\eqref{s1e14} while the first three terms can be combined as the total
time derivative of $\rho$ giving us
\begin{equation}\label{s3e4}
\frac{d\rho}{dt} = 0.
\end{equation}
Thus, the ensemble of systems all described by the same Hamiltonian moves in the
phase space like an incompressible fluid.

\section{Canonical transformation}\label{s4}
Consider the transformations
\begin{eqnarray}
q^\ast_i &=& q^\ast_i(q_1, \ldots, q_n, p_1, \ldots, p_n) \label{s4e1} \\
p^\ast_i &=& p^\ast_i(q_1, \ldots, q_n, p_1, \ldots, p_n) \label{s4e2}
\end{eqnarray}
for all $i = 1, \ldots, n$, $n$ being the number of degrees of freedom of the
system. The form of the Hamiltonian is may not remain the same as a result of
these transformations. However, $H(q_i, p_i, t) \mapsto H^\ast(q^\ast_i, 
p^\ast_i, t)$ and if
\begin{eqnarray}
\frac{\partial H^\ast}{\partial q^\ast_i} &=& -\dot{p}^\ast_i \label{s4e3} \\
\frac{\partial H^\ast}{\partial p^\ast_i} &=&  \dot{q}^\ast_i \label{s4e4}
\end{eqnarray}
then the transformations \eqref{s4e1} and \eqref{s4e2} are called \emph{
canonical transformations}. Canonically transformed variables obey canonical
equations of motions.

The simplest canonical transformations are the point transformations
\begin{eqnarray}
q^\ast_i &=& q^\ast_i(q_1, \ldots, q_n, t)\label{s4e5} \\
p^\ast_i &=& \frac{\partial L}{\partial \dot{q}^\ast_i} \label{s4e6}
\end{eqnarray}
The next simplest canonical transformation are
\begin{eqnarray}
q^\ast_i &=& p_i \label{s4e7} \\
p^\ast_i &=& q_i \label{s4e8}
\end{eqnarray}

The system can as well be described by the two Lagrangians
\begin{eqnarray}
L(q_i, \dot{q}_i, t) &=& \sum_{i=1}^n p_i\dot{q}_i - H(q_i, p_i, t)
\label{s4e9} \\
L^\ast(q^\ast_i, \dot{q}^\ast_i, t) &=& \sum_{i=1}^n p^\ast_i\dot{q}^\ast_i - 
H(q^\ast_i, p^\ast_i, t) \label{s4e10}
\end{eqnarray}
each one of which satisfies the action principle. That is
\begin{eqnarray}
\delta\int_{t_1}^{t_2} L(q_i, \dot{q}_i, t)dt &=& 0 \label{s4e11} \\
\delta\int_{t_1}^{t_2} L^\ast(q^\ast_i, \dot{q}^\ast_i, t)dt &=& 0 \label{s4e12}
\end{eqnarray}
From equations \eqref{s4e11} and \eqref{s4e12} we infer that the two
Lagrangians are related as
\begin{equation}\label{s4e13}
L(q_i, \dot{q}_i, t) = L^\ast(q^\ast_i, \dot{q}^\ast_i, t) + \frac{dF}{dt},
\end{equation}
for some function $F$. Equivalently, we can write
\begin{equation}\label{s4e14}
\sum_{i=1}^n p_i\dot{q}_i - H(q_i, p_i, t) = 
\sum_{i=1}^n p^\ast_i\dot{q}^\ast_i - H^\ast(q^\ast_i, p^\ast_i, t) +
\frac{dF}{dt}.
\end{equation}
On rearranging and writing it in terms or differentials,
\begin{equation}\label{s4e15}
\sum_{i=1}^n(p_i dq_i - p^\ast_i dq^\ast_i) = H^\ast dt - Hdt + dF.
\end{equation}

The function $F$ in \eqref{s4e14} is called the \emph{generating function}
and we can choose it to depend on one `old' and one `new' coordinate. Thus,
the four possibilities are
\begin{eqnarray}
F = F(q_i, q_i^\ast) \label{s4e16} \\
F = F(q_i, p_i^\ast) \label{s4e17} \\
F = F(p_i, q_i^\ast) \label{s4e18} \\
F = F(p_i, p_i^\ast) \label{s4e19} 
\end{eqnarray}
Assume that $F$ is of the form \eqref{s4e17} in equation \eqref{s4e15} to get
\begin{equation}\label{s4e20}
\sum_{i=1}^np_idq_i - Hdt = \sum_{i=1}^np^\ast_idq^\ast_i - H^\ast dt + 
dF(q_i, p_i^\ast).
\end{equation}
Rearrange it as
\[
\sum_{i=1}^n(p_idq_i + q_i^\ast dp^\ast_i) + (H^\ast - H)dt = 
d\left(\sum_{i=1}^n p_i^\ast q_i^\ast\right) + dF(q_i, p_i^\ast)
\]
Let
\begin{equation}\label{s4e21}
S = \sum_{i=1}^n p_i^\ast q_i + F
\end{equation}
then we have
\begin{equation}\label{s4e22}
\sum_{i=1}^n(p_idq_i + q_i^\ast dp^\ast_i) + (H^\ast - H)dt = dS.
\end{equation}
The left hand side suggest that $S$ can be considered to be a function of
$q_i, p_i^\ast, t$ and
\begin{eqnarray}
p_i &=& \pd{S}{q_i} \label{s4e23} \\
q_i^\ast &=& \pd{S}{p_i^\ast} \label{s4e24} \\
H^\ast - H &=& \pd{S}{t} \label{s4e25}
\end{eqnarray}
If we can find a transformation that can lead to $H^\ast = 0$, we get
\begin{equation}\label{s4e26}
\pd{S}{t} + H\left(q_i, p_i, t\right) = 0
\end{equation}
or, using equation \eqref{s4e23}
\begin{equation}\label{s4e27}
\pd{S}{t} + H\left(q_i, \pd{S}{q_i}, t\right) = 0.
\end{equation}
This is Hamilton-Jacobi equation.

\section{Hamilton-Jacobi equation}\label{s5}
The Hamilton-Jacobi equation is a first order partial differential equation
in the unknown $S$ considered to be a real-valued function of $n+1$ variables
$q_1, \ldots, q_n$ and $t$. However, we had assumed $S$ to be a function of
$2n + 1$ variables $q_1, \ldots, q_n, p_1^\ast, \ldots, p_n^\ast$ and $t$.
We reconcile these facts by noting that the solution of \eqref{s4e27} will
have $n + 1$ constants of which one is additive while the rest are not. We
ignore the additive constant and express the solution as $S(q_1, \ldots, q_n,
\beta_1, \ldots, \beta_n, t)$. Thus,
\begin{equation}\label{s5e1}
p_i^\ast = \beta_i.
\end{equation}
Therefore, from equations \eqref{s4e23} and \eqref{s4e24} we get
\begin{eqnarray}
p_i &=& \pd{S}{q_i} \label{s5e2} \\
q_i^\ast &=& \pd{S}{\beta_i} \label{s5e3}
\end{eqnarray}
Since $H^\ast = 0$, $q_i^\ast$ and $p_i^\ast$ are both constants. Let us set
\begin{equation}\label{s5e4}
q_i^\ast = \gamma_i,
\end{equation}
so that $q_i$ can be evaluated in terms of the constants $\beta_i, \gamma_i$ 
and $t$. We will illustrate the use of Hamilton-Jacobi equation with an example.

\subsection{Simple harmonic oscillator}
The Hamiltonian of this conservative system is
\begin{equation}\label{s5e5}
H = \frac{p^2}{2m} + \frac{kq^2}{2},
\end{equation}
where $m$ is mass of the oscillator and $k$ is the spring constant. The 
Hamilton-Jacobi equation is
\begin{equation}\label{s5e6}
\pd{S}{t} + H\left(q, \pd{S}{q}\right) = 0.
\end{equation}
Note that the Hamiltonian of a harmonic oscillator is independent of $t$.
Let us assume $S$ to be of the form $S(q, t) = S_1(q) + F(t)$. (Recall that
in the Hamilton-Jacobi equation $S$ is assumed to be a function of the $n+1$
variables $q_1, \ldots, q_n$ and $t$.) We thus have
\[
\pd{F}{t} = -H\left(q, \pd{S}{q}\right).
\]
The left hand side depends only on $t$ and the right hand side only on $q$, 
each side can be equated to a constant $-E$, the total energy of the system.
Thus,
\begin{equation}\label{s5e7}
F = -\beta t
\end{equation}
where we ignore the additive constant. Now consider
\begin{equation}\label{s5e8}
H\left(p, \pd{S_1}{q}\right) = \beta.
\end{equation}
Use the form of Hamiltonian in equation \eqref{s5e5} to get
\begin{equation}\label{s5e9}
\frac{1}{2m}\left(\pd{S_1}{q}\right)^2 + \frac{kq^2}{2} = \beta.
\end{equation}
The solution of this equation is
\begin{equation}\label{s5e10}
S_1(q) = \int\sqrt{2m\beta - kmq^2}dq 
\end{equation}
The function $S(q, t)$ is thus,
\begin{equation}\label{s5e11}
S(q, t) = \int\sqrt{2m\beta - kmq^2}dq - \beta t.
\end{equation}
Now,
\begin{equation}\label{s5e12}
q^\ast = \pd{S}{\beta} = m\int\frac{dq}{\sqrt{2m\beta - kmq^2}} - t.
\end{equation}
But $q^\ast$ is another constant, say $\gamma/\omega$, where
\begin{equation}\label{s5e13}
\omega^2 = \frac{k}{m}
\end{equation}
so that
\[
m\int\frac{dq}{\sqrt{2m\beta - kmq^2}} = \frac{\gamma}{\omega} + t.
\]
The integral can be readily evaluated to get
\begin{equation}\label{s5e14}
q = \sqrt{\frac{2\beta}{k}}\sin(\omega t + \gamma).
\end{equation}

\section{Action-angle variables}\label{s6}
If a conservative system is periodic then we define an `action variable' as
\begin{equation}\label{s6e1}
J_i = \oint p_idq_i.
\end{equation}
The time independent Hamilton-Jacobi equation is
\begin{equation}\label{s6e2}
H\left(q_i, \pd{S}{q_i}\right) = E.
\end{equation}
We can write the solution of this equation as
\begin{equation}\label{s6e3}
S(q_1, \ldots, q_n, \beta_1, \ldots, \beta_n) = 
\sum_{i=1}^n S_i(q_i, \beta_1, \ldots, \beta_n).
\end{equation}
Now,
\begin{equation}\label{s6e4}
J_i = \oint p_i dq_i = \oint \pd{S}{q_i}dq_i = \oint \frac{dS_i}{dq_i}dq_i.
\end{equation}
From \eqref{s6e3} and \eqref{s6e4} it is clear that $J_i$ are a function of
$\beta_1, \ldots, \beta_n$. Thus,
\begin{equation}\label{s6e5}
J_i = J_i(\beta_1, \ldots, \beta_n)
\end{equation}
which can be inverted so that
\begin{equation}\label{s6e6}
\alpha_i = \alpha_i(J_1, \ldots, J_n).
\end{equation}
We can then write equation \eqref{s6e3} as
\begin{equation}\label{s6e7}
S = S(q_1, \ldots, q_n, J_1, \ldots, J_n).
\end{equation}
Equations \eqref{s5e2} and \eqref{s5e3} now become
\begin{eqnarray}
p_i &=& \pd{S}{q_i} \label{s6e8} \\
q_i^\ast &=& \pd{S}{J_i} \label{s6e9}
\end{eqnarray}
From equation \eqref{s4e25}
\begin{equation}\label{s6e10}
H^\ast = H = E(J_1, \ldots, J_n).
\end{equation}
Now,
\begin{equation}\label{s6e11}
\dot{p}_i^\ast = -\pd{H^\ast}{q_i^\ast} = 0
\end{equation}
but
\begin{equation}\label{s6e12}
\dot{q}_i^\ast = \pd{H^\ast}{p_i^\ast} = \pd{H^\ast}{J_i}
\end{equation}
because we regard $\beta_i$, or equivalently $J_i$, as $p_i^\ast$. From
\eqref{s6e10},
\begin{equation}\label{s6e13}
\dot{q}_i^\ast = \pd{E}{J_i} = \nu_i,
\end{equation}
a constant, from which we readily get
\begin{equation}\label{s6e14}
q_i^\ast = \nu_i t + \alpha_i,
\end{equation}
$\alpha_i$ being a constant of integration. Traditionally, $q_i^\ast$ are
denoted by $\omega_i$ so that equation \eqref{s6e15} becomes
\begin{equation}\label{s6e15}
\omega_i = \nu_i t + \alpha_i.
\end{equation}
$\omega_i$ are called the angle variables corresponding to $J_i$, the action
variables.

\bibliographystyle{plain}
\bibliography{hm}
\end{document}
