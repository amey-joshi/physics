\chapter{d'Alembert's Principle}\label{c1}
\section{Beyond Newton's laws}\label{c1s1}
Newton's programme of classical mechanics has a few limitations.
\begin{enumerate}
\item It needs a frame of reference in which Newton's first law is valid.
\item Newton's second law
\begin{equation}\label{c1s1e1}
m\ddot{x} = F_x, m\ddot{y} = F_y, m\ddot{z} = F_z
\end{equation}
requires that we use the Cartesian coordinate system. However, the symmetry of
a few problems suggests that we use other systems. If $q_i$ are the coordinates
in the other systems then in general,
\begin{equation}\label{c1s1e2}
m\ddot{q}_i \ne F_i,
\end{equation}
where $F_i$ in this case is the $q_i$ component of the force which \emph{may
not} be the same as the $F_i$ in equation \eqref{c1s1e1}.
\item Equation \eqref{c1s1e1} requires that we know $F_i$. This is often not the
case when the motion is constrained. When a bead is constrained to move along
the wire in which it is woven we do not know the force that stops the bead
from leaving the wire.
\item Newton's third law does not hold good when electromagnetic forces are
involved.
\end{enumerate}

The first of these limitations can be overcome by introducing pseudo-forces.
The second one can be circumvented by carefully evaluating the components of
the acceleration in coordinate systems other than the Cartesian. The third one
is can be tamed by using d'Alembert's principle which we introduce next. The
fourth one requires a redefinition of momentum which can be done systematically
in Lagrange's formulation of classical mechanics.

\section{d'Alembert's principle}\label{c1s2}
Consider a few examples of constrained motion.
\begin{enumerate}
\item A beam constrained to move along a circular hoop. If the hoop is centred
at the origin and in the $xy$ plan then the constraint is
\begin{equation}\label{c1s2e1}
x^2 + y^2 = a^2,
\end{equation}
where $a$ is the radius of the hoop.
\item A particle constrained to move along an inclined plane. If $x, y,
z$ are the coordinates of the particle then they must be confined to
the plane 
\begin{equation}\label{c1s2e2}
lx + my + nz = p
\end{equation}
where $l, m, n$ are the direction cosines of the normal to the plane
and $p$ is its distance from the origin.
\item Particles of a rigid body in which the distances between the particles
is constant. If $(x_n, y_n, z_n$ denote the coordinates of the $n$th particle 
then
\begin{equation}\label{c1s2e3}
\sum_{i=1}^3\sqrt{(x_n-x_m)^2 + (y_n-y_m)^2 + (z_n-z_m)^2} = \text{constant}
\end{equation}
for all $n, m = 1, \ldots, N$, $N$ being the number of particles of the rigid
body.
\item Molecules of a monoatomic gas confined to a box. If the box is centred
at the origin and is a cube of side $a$ then
\begin{equation}\label{c1s2e4}
-\frac{a}{2} \le x_n, y_n, z_n \le \frac{a}{2},
\end{equation}
where $(x_n, y_n, z_n)$ are the coordinates of the $n$th gas molecule.
\item A disc rolling without slipping on a plane. Let us assume that the disc
has a radius $a$ and it rolls on the $xy$ plane. To specify the motion we 
need to know where on the plane the disc is, which point of the disc is in 
contact with the plane and how is the disc oriented. The point of contact can 
be specified by its coordinates $(x, y)$, the point of the disc by its 
angular separation $\phi$ from a fixed line on the disc and the orientation 
of the disc by another angle $\theta$ between the disc and the $x$ direction.
The velocity of the point of contact has a magnitude $a\dot{\phi}$ and a 
direction perpendicular to the plane of the disc. The plane of the disc makes 
an angle $\theta$ with the positive $x_1$ direction. Therefore, the velocity 
makes an angle $\pi/2 - \theta$ with the positive $x$ direction. Thus,
\begin{eqnarray}
v_x &=& a\sin\theta\dot{\phi} \label{c1s2e5} \\
v_y &=& a\cos\theta\dot{\phi} \label{c1s2e6} 
\end{eqnarray}
We can write these equations in terms of the coordinates $(x, y, \phi, 
\theta)$ used to specify the motion.
\begin{eqnarray}
dx &=& a\sin\theta d\phi \label{c1s2e7} \\
dy &=& a\cos\theta d\phi \label{c1s2e8} 
\end{eqnarray}
\end{enumerate}
Refer to section 1.2 of Rana and Joag's book \cite{rc} for many more examples
of contraints of various kinds. Constraints in the first three examples are
called \emph{holonomic} because they can be expressed as a (whole) function
\begin{equation}\label{c1s2e9}
f(q_1, q_1, \ldots) = 0.
\end{equation}
The constraint of equation \eqref{c1s2e4} is not holonomic because it involves
inequalities. The constraints of equations \eqref{c1s2e7} and \eqref{c1s2e8}
are not holonomic because the two equations cannot be integrated unless we
fix $\theta$ \cite{goldstein2002classical}. We further note that for all
holonomic constraints the force of constraint is perpendicular to the direction
of motion of the particles. Therefore, the forces of constraint do not do any
work on the particles. This is not the case when the holonomic constraints 
depend on time. For example, if the length of the bob changes with time, the
actual motion of the bob is not perpendicular to the tension in the string.
In order to treat such constraints like the time-independent ones, d'Alembert
introduced the idea of a \emph{virtual displacement}. A virtual displacement
happens instantaneously without violating the constraint. In the case of the
bob of a pendulum whose length varies with time, the virtual displacement is
always perpendicular to the tension in the string because it happens without
a passage of time. The work done in a virtual displacement is called the 
\emph{virtual work}. 

d'Alembert proposed the principle that infinitesimal virtual work by forces of
constraint is zero when the virtual displacement is reversible. The condition
of reversibility rules out dissipative constraints like frictional forces.

Let us now consider Newton's equation for a single particle. We can write the 
total force on the particle as sum of external forces
$F_i$ and constraint forces $F^c_i$ so that
\begin{align}
\left.
\begin{array}{ll}
m\ddot{x} &= F_x + F^c_x \\
m\ddot{y} &= F_y + F^c_y \\
m\ddot{z} &= F_z + F^c_z 
\end{array}
\right\}\label{c1s2e10}
\end{align}
If $\delta x, \delta y, \delta z$ denote an infinitesimal virtual displacement 
in each coordinate then
\begin{equation}\label{c1s2e11}
(m\ddot{x} - F_x)\delta x + (m\ddot{y} - F_y)\delta y + (m\ddot{z} - F_z)
\delta z = F^c_x\delta x + F^c_y\delta y + F^c_z\delta z
\end{equation}
d'Alembert's principle of virtual work states that the right hand side is zero
so that
\begin{equation}\label{c1s2e12}
(m\ddot{x} - F_x)\delta x + (m\ddot{y} - F_y)\delta y + (m\ddot{z} - F_z)
\delta z = 0.
\end{equation}
This is the mathematical form of d'Alembert's principle. Its chief advantage 
is that it does not involve the forces of constraint. However, it is not at
all as simple as equation \eqref{c1s1e1} because the virtual displacements
$\delta x, \delta y, \delta z$ are not independent. We can generalise it to a 
system of $N$ particles as
\begin{equation}\label{c1s2e13}
\sum_{i=1}^N\left(
(m_i\ddot{x}_i - X_i)\delta x_i + (m\ddot{y_i} - Y_i)\delta y_i + 
(m_i\ddot{z}_i - Z_i) \delta z_i\right) = 0,
\end{equation}
where $X_i, Y_i, Z_i$ are the $x, y, z$ components of the external forces on
the $i$th particle. Once again we emphasize that the virtual displacements 
$\delta x_i, \delta y_i, \delta z_i$ are not independent if the system moves 
under constraints.  Therefore, there is no way to simplify \eqref{c1s2e13} 
any further.

Examples of d'Alembert's principle.
\begin{enumerate}
\item Consider the motion of a simple pendulum in $xz$ plane. The motion
of the bob is constrained by the equation
\begin{equation}\label{c1s2e14}
x^2 + z^2 = l^2,
\end{equation}
where $l$ is the length of the string. We can express it as
\begin{equation}\label{c1s2e15}
x\delta x + z\delta z = 0,
\end{equation}
where we interpret $\delta x$ and $\delta z$ as virtual displacements.
From d'Alembert's principle of equation \eqref{c1s2e13} we get
\begin{equation}\label{c1s2e16}
(m\ddot{x} - F_x)\delta x + (m\ddot{z} - F_z)\delta z = 0.
\end{equation}
$F_x = 0$ and $F_z = -mg$ gives
\begin{equation}\label{c1s2e17}
\ddot{x} \delta x + (\ddot{z} + g)\delta z = 0.
\end{equation}
For small oscillations $z \approx -l$ (origin being at the point the string 
is held fixed) so that equation \eqref{c1s2e15} becomes
$x \delta x \approx l \delta z$ so that equation \eqref{c1s2e17} can
be written as
\begin{equation}\label{c1s2e18}
\ddot{x} \delta x + (0 + g)\left(\frac{x}{l}\right)\delta x = 0
\Rightarrow \left(\ddot{x} + \frac{g}{l}x\right)\delta x = 0.
\end{equation}
If this equation must be true for all virtual displacements,
\begin{equation}\label{c1s2e19}
\ddot{x} + \frac{g}{l}x = 0.
\end{equation}
Note that the d'Alembert's principle of equation \eqref{c1s2e17} could be
simplified only after we used the equation of constraint \eqref{c1s2e15}. This
theme will recur each time we apply d'Alembert's principle for constrained 
dynamical systems.

\item Consider Atwood's machine with masses $m_1$ and $m_2$ connected with
an ideal string wound aroung an ideal pulley. If $x_1$ and $x_2$ denote their
displacements then the equation of constraint is
\begin{equation}\label{c1s2e20}
x_1 + x_2 = l,
\end{equation}
$l$ being the length of the string connecting the two masses. In terms of
virtual displacement, the equation of constraint becomes
\begin{equation}\label{c1s2e21}
\delta x_1 + \delta x_2 = 0.
\end{equation}
d'Alembert's principle applied to the system gives
\begin{equation}\label{c1s2e22}
(m_1\ddot{x}_1 - F_1)\delta x_1 + (m_2\ddot{x}_2 - F_2)\delta x_2
= 0.
\end{equation}
Since $F_1 = -m_1g, F_2 = -m_2g$ we have
\begin{equation}\label{c1s2e23}
m_1(\ddot{x}_1 + g)\delta x_1 + m_2(\ddot{x}_2 + g)\delta x_2 = 0.
\end{equation}
From equation \eqref{c1s2e20} $\ddot{x}_1 = -\ddot{x}_2$ and from equation
\eqref{c1s2e21} we have $\delta x_1 = -\delta x_2$ so that \eqref{c1s2e23}
becomes
\begin{equation}
\left(m_1(\ddot{x}_1 + g) - m_2(-\ddot{x}_1 + g)\right)\delta x_1 = 0.
\end{equation}
As this equation is valid for all virtual displacements we have
\begin{equation}\label{c1s2e25}
\ddot{x}_1 = -\frac{m_1 - m_2}{m_1 + m_2}g.
\end{equation}

\item If a particle is constrained to move along a circle of radius $a$,
centred at the origin and if there are no external forces then d'Alembert's 
principle gives
\begin{equation}\label{c1s1e26}
m\ddot{x}\delta x + m\ddot{y}\delta y = 0.
\end{equation}
Since the particle is constrained to move on the circle, $x = a\cos\theta$ 
and $y = a\sin\theta$. Therefore,
\begin{eqnarray}
\delta x &=& -a\sin\theta \delta\theta \\
\delta y &=& a\cos\theta  \delta\theta
\end{eqnarray}
and
\begin{eqnarray}
\dot{x} &=& -a\sin\theta\dot{\theta} \\
\ddot{x} &=& -a\cos\theta\dot{\theta}^2 - a\sin\theta\ddot{\theta} \\
\dot{y} &=& a\cos\theta\dot{\theta} \\
\ddot{y} &=& -a\sin\theta\dot{\theta}^2 + a\cos\theta\ddot{\theta}
\end{eqnarray}
so that equation \eqref{c1s1e26} becomes
\[
\ddot{x}\delta x + \ddot{y}\delta y = (a^2\sin\theta\cos\theta
\dot{\theta}^2 + a^2\sin^2\theta\ddot{\theta} - a^2\sin\theta\cos\theta
\dot{\theta}^2 + a^2\cos^2\theta\ddot{\theta})\delta\theta = 0
\]
or $\ddot{\theta}\delta\theta = 0$ which implies that $\dot{\theta} = $ a
constant. In this problem we used the equation of constraint to replace
the two coordinates $x$ and $y$ with a single one $\theta$.
\end{enumerate}

Some more examples of constraints.
\begin{enumerate}
\item A rigid rod is constrained to move within a spherical bowl so that
is tends always touch the bowl's inner surface \cite[Problem 3, chapter 2]
{akr}. Without the constraint of the bowl, six degrees of freedom suffice t
describe the motion of the rod.  If $x_1, y_1, z_1$ and $x_2, y_2, z_2$ are
the coordinates of the end-points of the rod then then equations of
constraint are
\begin{eqnarray}
x_1^2 + y_1^2 + z_1^2 &=& a^2 \\
x_2^2 + y_2^2 + z_2^2 &=& a^2 \\
(x_1 - x_2)^2 + (y_1 - y_2)^2 + (z_1 - z_2)^2 &=& l^2 \\
l &\le& a
\end{eqnarray}
Thus, the motion of the rod can be described by two coordinates alone. One 
of them tells the `latitude' of the centre of the rod and the other one the
`longitude'.

\item A spherical pendulum is not restricted to move in a plane. Therefore,
we need three coordinates $x, y, z$ to describe its motion. They are
constrained to obey
\begin{equation}\label{c1s2e37}
x^2 + y^2 + z^2 = l^2.
\end{equation}
Expressing the constraint in terms of virtual displacements,
\begin{equation}\label{c1s2e38}
x\delta{x} + y\delta{y} + z\delta{x} = 0
\end{equation}
d'Alembert's principle gives
\begin{equation}\label{c1s2e39}
\ddot{x}\delta{x} + \ddot{y}\delta{y} + (\ddot{z} + g)\delta{z} = 0
\end{equation}
\end{enumerate}

\section{Lagrange's equations of the first kind}\label{c1s3}
We mentioned that the virtual displacements $\delta x_i, \delta y_i, 
\delta z_i$ are not 
independent in a system under constraints. If we restrict ourselves to
holonomic constraints then each constraint is an equation of the form
\begin{equation}\label{c1s3e1}
\varphi_i(x_1, \ldots, z_N) = 0.
\end{equation}
The index $i$ ranges from $1$ to $m$, the number of constraints. We can
write each of these equations as
\begin{equation}\label{c1s3e2}
\sum_{j=1}^{N}\left(\pd{\varphi_i}{x_j}\delta x_j 
+ \pd{\varphi_i}{y_j}\delta y_j
+ \pd{\varphi_i}{z_j}\delta z_j\right) = 0
\end{equation}
We now multiply each of these equations with a constant $\lambda_i$ and
add them to \eqref{c1s2e13} to get
\begin{align}
\sum_{j=1}^N \left(m_j\ddot{x}_j - X_j + 
\sum_{k=1}^m \lambda_k\pd{\varphi_i}{x_j} \right)\delta x_j & + \nonumber \\
\sum_{j=1}^N \left(m_j\ddot{y}_j - Y_j + 
\sum_{k=1}^m \lambda_k\pd{\varphi_i}{y_j} \right)\delta y_j & + \nonumber \\
\sum_{j=1}^N \left(m_j\ddot{z}_j - Z_j + 
\sum_{k=1}^m \lambda_k\pd{\varphi_i}{z_j} \right)\delta z_j & = 0 \label{c1s3e3}
\end{align}
This equation has $3N$ terms on the left hand side. Choose $\lambda_k$
such that for the first $m$ terms
\begin{equation}\label{c1s3e4}
\sum_{j=1}^N \left(m_j\ddot{u}_j - U_j + 
\sum_{k=1}^m \lambda_k\pd{\varphi_i}{u_j} \right)\delta u_j = 0,
\end{equation}
where $u$ and $U$ could denote $x, y, z$ or $X, Y, Z$ depending on $m$.
Equation \eqref{c1s3e3} now has only $3N - m$ terms. The virtual displacements
in each of these can be considered to be independent. Therefore, \eqref{c1s3e4}
can be considered to be true for all $3N$ terms. Equation \eqref{c1s3e4} is
called \emph{Lagrange's equation of the first kind}. In fact,
\begin{equation}
Q_i = \left(\sum_{k=1}^r\lambda_k \pd{\varphi_k}{x_i},
            \sum_{k=1}^r\lambda_k \pd{\varphi_k}{y_i},
            \sum_{k=1}^r\lambda_k \pd{\varphi_k}{z_i}\right)
\end{equation}
is the force of constraint on the $i$th particle.

As an example of this approach, consider the equation of constraint of a 
spherical pendulum given in \eqref{c1s2e38}. We can combine it with 
d'Alembert's equation \eqref{c1s2e39} to get
\begin{equation}\label{c1s3e6}
(\ddot{x} + \lambda x)\delta{x} + (\ddot{y} + \lambda y)\delta{y} + 
(\ddot{z} + g + \lambda z)\delta{z} = 0
\end{equation}
Now we consider the amplitude of oscillation to be small, that is $x_3 \approx
-l$ and $\ddot{x}_3 \approx 0$ so that the third term \eqref{c1s3e6} is
\[
0 + g - \lambda l.
\]
It vanishes if we choose $\lambda = g/l$ and we get
\begin{eqnarray}
\ddot{x} + \frac{g}{l}x &=& 0 \\
\ddot{y} + \frac{g}{l}y &=& 0 
\end{eqnarray}
are the equations of motion of the spherical pendulum.

We next use Lagrange's method to find the force of constraint in Atwood 
machine. Combining the equation of constraint \eqref{c1s2e21} with 
d'Alembert's equation \eqref{c1s2e23} we get
\begin{equation}
m_1\left(\ddot{x}_1 + g + \frac{\lambda}{m_1}\right)\delta x_1 + 
m_2\left(\ddot{x}_2 + g + \frac{\lambda}{m_2}\right)\delta x_2 = 0.
\end{equation}
Choose $\lambda$ such that
\begin{equation}
\ddot{x}_1 + g + \frac{\lambda}{m_1} = 0
\end{equation}
that is
\begin{equation}
\lambda = -m_1(\ddot{x}_1 + g).
\end{equation}
Using equation \eqref{c1s2e25} we get
\begin{equation}
\lambda = \frac{2m_1m_2}{m_1 + m_2}g
\end{equation}
as the tension in the string.

Lagrange's equations of the first kind solve the problem of involving the
forces of constraint in the problem. Yet, they are firmly rooted in the 
Cartesian coordinate system. In the next chapter we will severe this connection
with the Cartesian coordinates.

Before we close we remark that the framework of this section can be used
for time-dependent constraints. If equation \eqref{c1s3e1} is written as
\begin{equation}\label{c1s3e13}
\varphi_i(x_1, \ldots, z_N, t) = 0
\end{equation}
then equation \eqref{c1s3e2} becomes
\begin{equation}\label{c1s3e14}
\sum_{j=1}^{N}\left(\pd{\varphi_i}{x_j}\delta x_j 
+ \pd{\varphi_i}{y_j}\delta y_j
+ \pd{\varphi_i}{z_j}\delta z_j + \pd{\varphi_i}{t}\delta t\right) = 0
\end{equation}
However, when we interpret $\delta x_j, \delta y_j, \delta z_j$ as virtual 
displacements then $\delta t$ is zero and the rest of the analysis applies 
without change.
