\documentclass{article}
\usepackage{amsmath}
\title{Operators in Quantum Mechanics}
\author{Amey Joshi}
\begin{document}
\maketitle
\abstract{One of the postulates of quantum physics states that a dynamical 
variable in classical physics is represented by an operator in the quantum 
theory. It is hard to imagine what does it mean by `representing' a classical
dynamical variable with an operator. It is even harder to understand how anyone
could think of such a representation. This article attempts to shed light on
the origin of this postulate}

\section{Dynamical variables as matrices}
Heisenberg \cite{heisenberg1925quantum} laid down the foundations of quantum 
mechanics by starting to construct its theory based on the quantities that are 
observable in an experiment. The position and momentum of an electron are not 
observable.  However, the spectral lines are observed. Their frequency and 
intensity are measured in experiments. Heisenberg proposes a representation of
the position of an electron in terms of the spectral frequencies and concludes
that the only way to do so is by representing position as a matrix of numbers.
This paper of Heisenberg's is famous for being hard to understand
\cite{aitchison2004understanding}. However, Heisenberg states his thoughts more
lucidly in the appendix `Mathematical Apparatus' of his book 
\cite{heisenberg1949physical}.

Heisenberg's cautions \cite{heisenberg1949physical} that his argument for
representing dynamical variables as operators should not be viewed as a 
mathematical proof. It is a physicist's insight whose correctness should be
judged on its success in describing natural processes measured in experiments.

\section{The experimental facts}
Heisenberg takes into account the following well-established experimental facts
to build his theory:
\begin{enumerate}
\item The Rydberg-Ritz combination principle that all spectral lines of an
element may be represented as the differences of small number of terms. If
$T_1, T_2, \ldots$ are the terms then the atomic frequencies form a two
dimensional array
\begin{equation}\label{e1}
\nu_{nm} = T_n - T_m
\end{equation}
and obey the combination principle
\begin{equation}\label{e2}
\nu_{nk} + \nu_{km} = \nu_{nm}.
\end{equation}
\item The energy levels of an atom have a discrete sets of values as observed
in the Frank-Hertz experiments.
\item The spectral frequencies are related to energy levels through the 
Einstein-Bohr relation
\begin{equation}\label{e3}
\nu_{nm} = \frac{E_n - E_m}{h}.
\end{equation}
\end{enumerate}

One way to relate the dynamical variables to spectral frequencies is to write
them as Fourier series. Thus, a generalised coordinate $q_k$ of a periodic 
system of $f$ degrees of freedom is
\begin{equation}\label{e4}
q_k(t)=\sum_{n_1, \ldots, n_f}q^{(k)}_{n_1, \ldots, n_f}\exp[2\pi i(n_1\nu_1 +
\cdots + n_f\nu_f)t],
\end{equation}
where $\nu_1, \ldots, \nu_f$ are the fundamental frequencies of the system.
Heisenberg considered periodic systems because an atom is one such.

If equation \eqref{e4} is to be taken over to quantum mechanics, it should
be changed so that
\begin{enumerate}
\item The amplitudes depend on frequencies of two states of the atom and
\item When we combine quantities like $q_k(t)$, the Rydberg-Ritz combination
principle should be obeyed.
\end{enumerate}

The first condition forces us to write $q_k(t)$ in terms of the quantities
\begin{equation}\label{e5}
q^{(k)}(n_1, \ldots, n_f;m_1, \ldots, m_f)
\exp\left[2\pi i\nu(n_1,\ldots,n_f;m_1,\ldots, m_f)t\right]
\end{equation}
where the amplitude and the frequencies are the ones observed in the atom's
spectrum. Instead of summing numbers of this form in a Fourier series, 
Heisenberg postulated that these numbers together represent the generalised
coordinate. Thus, in quantum mechanics,
\begin{equation}\label{e6}
q_k(t) = \begin{bmatrix}
q^{(k)}_{1;1}e^{2\pi i\nu(1;1)t} & \ldots &q^{(k)}_{1;m_f}e^{2\pi i\nu(1;m_f)t}
\\
\vdots & \\
q^{(k)}_{n_f;1}e^{2\pi i\nu(n_f;1)t} & \ldots &q^{(k)}_{1;m_f}
e^{2\pi i\nu(n_f;m_f)t}
\end{bmatrix}
\end{equation}
We now use the classical representation of equation \eqref{e4} to build 
analogies to put constraints on the form of the representation of equation
\eqref{e6}.
\begin{enumerate}
\item The quantity $q_k(t)$ on the left hand side of equation \eqref{e4} must
be real. Therefore, the sum on the right hand side should have pairs of 
complex cojugate elements. Likewise, if the matrix must represent a dynamical 
variable and if its entries are complex then we must insist that the complex 
quantities occur in pairs of conjugates. From the Rydberg-Ritz combination
principle we know that $\nu(n, m) = -\nu(m, n)$. Therefore we must also have
\begin{equation}\label{e7}
q_k(m, n) = \overline{q_k(n, m)}.
\end{equation}

\item One can add quantities like \eqref{e4} so that amplitudes corresponding
to the same frequency add up. Likewise, we can add the array of numbers in 
equation \eqref{e6} such that the elements corresponding to same frequencies 
add up. Thus, the array of number corresponding to $q_1(t) + q_2(t)$ has 
elements
\begin{equation}\label{e8}
(q^{(1)}_{m, n} + q^{(2)}_{m, n})e^{2\pi i\nu(m, n)t}.
\end{equation}

\item Each frequency $\nu(n, m)$ in \eqref{e6} must be of the form $\Omega_n
- \Omega_n$ \cite{dirac1925fundamental}. When we multiply arrays of the form
\eqref{e6} we must remember to add together the terms that belong to the
same frequency. Consider the a combination of terms like $q^{(1)}_{l, m}
q^{(2)}_{m, n}$. It corresponds to the frequency $\Omega_m - \Omega_l + \Omega_n
- \Omega_m = \Omega_n - \Omega_l$. Therefore, we must add together terms like
$q^{(1)}_{l, m}q^{(2)}_{m, n}$ for all $m$ to get the term corresponding to
the frequency $\Omega_n - \Omega_l$.
\end{enumerate}
Thus, the arrays of numbers in equation \eqref{e6} are added and multiplied like
matrices. That is how the representation of dynamical variables as matrices
came about to be in quantum mechanics.
\bibliographystyle{plain}
\bibliography{why}
\end{document}
